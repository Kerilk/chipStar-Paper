\documentclass[lettersize,journal]{IEEEtran}
\usepackage{amsmath,amsfonts}
\usepackage{algorithmic}
\usepackage{algorithm}
\usepackage{array}
\usepackage[caption=false,font=normalsize,labelfont=sf,textfont=sf]{subfig}
\usepackage{textcomp}
\usepackage{stfloats}
\usepackage{url}
\usepackage{verbatim}
\usepackage{graphicx}
\usepackage{cite}
\usepackage{color}
\usepackage{amssymb}
\usepackage{xspace}
\usepackage{textcomp}

\usepackage{ifthen}
\newboolean{showcomments}
\setboolean{showcomments}{true}
\ifthenelse{\boolean{showcomments}}
{ \newcommand{\mynote}[3]{
     \fbox{\bfseries\sffamily\scriptsize#1}
        {\small$\blacktriangleright$\textsf{\emph{\color{#3}{#2}}}$\blacktriangleleft$}}
  \newcommand{\newtext}[1]{{\color{orange}{#1}}}}
{ \newcommand{\mynote}[3]{}
  \newcommand{\newtext}[1]{#1}}

% Please use a named note with this macro to comment the text:
\newcommand{\pj}[1]{ \mynote{PJ}{#1}{blue} }
\newcommand{\bv}[1]{ \mynote{BV}{#1}{green} }
\newcommand{\mb}[1]{ \mynote{MB}{#1}{cyan} }

\newcommand{\hiplz}{\texttt{HIPLZ}\xspace}
\newcommand{\hipcl}{\texttt{HIPCL}\xspace}
\newcommand{\hip}{\texttt{HIP}\xspace}
\newcommand{\opencl}{\texttt{OpenCL}\xspace}
\newcommand{\lz}{\texttt{L0}\xspace}
\newcommand{\sycl}{\texttt{SYCL}\xspace}
\newcommand{\cuda}{\texttt{CUDA}\xspace}
\newcommand{\hipstar}{\textit{hipstar}\xspace}

\hyphenation{op-tical net-works semi-conduc-tor IEEE-Xplore}
% updated with editorial comments 8/9/2021

\begin{document}

\title{hipstar: Enabling CUDA and HIP On an Open Cross-Vendor Standard Software Stack}

%\author{pekka.jaaskelainen }
%\date{March 2023}

\author{Pekka Jääskeläinen, Henry Linjamäki, Michal Babej, Peng Tu, Sarkar Sarbojit, Brice Videau, Colleen Bertoni, Kevin Harms, Paulius Velesko, Philip Roth, Rahulkumar Gaytri, Jisheng Zhao
        % <-this % stops a space
\thanks{Pekka Jääskeläinen, Henry Linjamäki, Michal Babej, Peng Tu and Sarbojit Sarkar are with Intel Corporation. \textit{Corresponding author: Pekka Jääskeläinen, email: pekka.jaaskelainen@intel.com}.}
\thanks{Pekka Jääskeläinen is also with Tampere University, Finland. }
\thanks{Paulius Velesko is with Pagan LC.}
\thanks{Brice Videau, Colleen Bertoni and Kevin Harms are with Argonne National Laboratory, ...}
\thanks{Philip Roth is with Oak Ridge National Laboratory, ... }
\thanks{Rahulkumar Gaytri is with National Energy Research Scientific Computing Center, ...}
\thanks{Jisheng Zhao is with Georgia Institute of Technology, Atlanta, Georgia.}
\thanks{\pj{The authors are not in any particular order. I put myself as 1st author as I'm leading the writing, and then I ordered the co-authors according to their affil.}}
%\thanks{This paper was produced by the IEEE Publication Technology Group. They are in Piscataway, NJ.}% <-this % stops a space
%\thanks{Manuscript received April 19, 2021; revised August 16, 2021.}}

% The paper headers
\markboth{IEEE Transactions on Parallel and Distributed Systems,~Vol.~X, No.~Y, Month~YEAR}%
{Shell \MakeLowercase{\textit{et al.}}: A Sample Article Using IEEEtran.cls for IEEE Journals}}

\IEEEpubid{0000--0000/00\$00.00~\copyright~2021 IEEE}
% Remember, if you use this you must call \IEEEpubidadjcol in the second
% column for its text to clear the IEEEpubid mark.

%Well it depends on the kind of paper you want to write, but if we present CHIP-SPV with it's environment, we need to have Phil (ORNL) for his work on hipBLAS (and maybe Wael, but his contributions are more limited), we need to have the work done on applications by Colleen and Rahul (NERSC), and we need to have Jisheng that wrote the prototype and Jeffrey (both GaTech) that designed the initial interrop. I think you can count on contributions from most of these.
%Of course, we'll have the whole team at intel and Paulius. I doubt AMD will want to be on.
%In the end it could very well be a Journal paper, depending on the above involvement in the text.
%Once here, we only have a couple more seniors who may want to join: Vivek (GaTech), Kevin (ANL), who were on the original hiplz paper. They both did participate in the writing.
%Since we're not talking about CUDA I assume Hyesoon and her team will want to do a separate paper

\maketitle

%%%%%%%%%%%%%%%%%%%%%%%%%%%%%%%%%%%%%%%%%%%%%%%%%%%%%%%%%%%%%%%%%%%%%%%%%

\begin{abstract}

%This document describes the most common article elements and how to use the IEEEtran class with \LaTeX \ to produce files that are suitable for submission to the IEEE.  IEEEtran can produce conference, journal, and technical note (correspondence) papers with a suitable choice of class options. 

\pj{This first paragraph is optional, we can remove it:}
Due to NVIDIA dominating the GPU market and despite its lack of cross-vendor portability, the C/C++-based application programming interface of CUDA and its related key libraries are still used in a significant fraction of software utilizing GPU-based acceleration. AMD's ROCm and its Heterogeneous-compute Interface for Portability (HIP) aims to alleviate the CUDA's lack of portability by providing a route out from the NVIDIA CUDA platform to AMD's devices.
  
In this article we describe \textit{hipstar}, an open source software platform which allows running CUDA and HIP programs on an a open cross-vendor standard software platform consisting of an OpenCL/SPIR-V backend and a carefully crafted set of standard extensions. We discuss the relevant technical aspects of \textit{hipstar} related to the feature mismatches between CUDA/HIP and OpenCL and exemplify its runtime overheads in comparison to executing CUDA/HIP applications directly with NVIDIA's CUDA software platform. The measurements show that the overheads induced by \textit{hipstar} are typically only in the order of N-M\%, thus negligible.\pj{TODO} Although being a relatively young code base, \textit{hipstar} is now considered mature enough for production use, as demonstrated by multiple application porting case studies performed for the Aurora supercomputer.

\end{abstract}

\begin{IEEEkeywords}
CUDA, HIP, OpenCL, SPIR-V, Portability, Shared Virtual Memory
\end{IEEEkeywords}

%%%%%%%%%%%%%%%%%%%%%%%%%%%%%%%%%%%%%%%%%%%%%%%%%%%%%%%%%%%%%%%%%%%%%%%%%

\section{Introduction}

\IEEEPARstart{W}{alled} garden strategy is popular within market dominating companies. Its purpose is to lock-in customers to company's products by making escaping the gates of the garden as costly as possible. NVIDIA's CUDA software platform is considered to be one of such walled gardens. It in part helps NVIDIA to expand and keep a foothold of their GPU market advantage, and at the same time maintain high innovation pace on the software APIs since there is no need to work with standardization committees that always have to aim for a consensus among multiple participating vendors.

For the end-users and other hardware vendors, naturally, the situation of a single-vendor dominating API is not optimal. End-users benefit from open standard software interfaces that enable switching the hardware without incurring significant non-recurring engineering costs required for porting all the legacy applications and libraries to a new software platform just to be able to utilize the purchased hardware optimally. Similarly, other hardware vendors aiming to get their piece of the market pie, would prefer an API that is not dictated by a single vendor.

AMD's ROCm~\cite{ROCm} software platform and its Heterogeneous-compute Interface for Portability (HIP) language~\cite{hip} helps escaping the CUDA walled garden by providing a route out from the NVIDIA CUDA platform to AMD's devices. HIP defines a subset of CUDA that is more easily portable to various hardware, thanks mainly to omitting various advanced features available in the later CUDA versions (some of these features are discussed in Section~\ref{subsection:compatgaps}). In order to enable easy automated path from CUDA applications, HIP is largely a copy of a CUDA C/C++ subset with a few minor differences, and renamed function names. It alleviates the CUDA portability problem, but doesn't solve it satisfactorily due to AMD targeting primarily their self-specified low level ROCm APIs which have not been so far ported to other than AMD platforms. An open source CUDA/HIP software platform solely based on open standards with a sincere aim for cross-vendor portability is still lacking.

In this article we propose a complete open source software platform that enables porting applications from NVIDIA-driven CUDA and AMD-driven ROCm platforms to any platform supporting the cross-vendor open standards OpenCL and SPIR-V. We evaluate the portability aspects on platforms with proprietary drivers, and as a further contribution, extend an open source OpenCL implementation to expand the scope to further device types. In comparison to the previous solutions, our proposed platform enables usually source-modification-free compilation of HIP programs along with supporting the necessary library dependencies for the most essential CUDA/ROCm APIs. 

The rest of the article is organized as follows: Section~\ref{section:runtime} discusses the technical challenges related to the runtime execution. Section~\ref{section:compilation} describes the most relevant parts of the compiler support. Section~\ref{section:libraries} identifies the most important libraries that must be supported for compatibility, and how they are supported in the proposed software platform. Section~\ref{section:expandingCompat} shows how we expanded the platform support using open source tooling. The aspects related to directly supporting CUDA programs instead of converting them first to HIP are discussed in Section~\ref{section:directCUDA}. Debugging and profiling support is outlined in Section~\ref{section:debuggingAndProfiling}. Then we describe how the software platform portability was tested (Section~\ref{section:platformPortability}), source code portability validated (Section~\ref{section:applications}) and performance evaluated (Section~\ref{section:applications}). Section~\ref{section:relatedWork} overviews the related work to the proposed software platform and its components. 
Finally, conclusions and future work plans are presented in~\ref{section:conclusions}.

%%%%%%%%%%%%%%%%%%%%%%%%%%%%%%%%%%%%%%%%%%%%%%%%%%%%%%%%%%%%%%%%%%%%%%%%%

\section{Runtime}
\label{section:runtime}

\pj{TODO: aspects related to runtime, how to make it portable, mapping to OpenCL}

\subsection{Memory Management}

CUDA and HIP support device pointers embedded into other data structures in host memory.\pj{TODO: Expand this aspect. What type of host-device memory management is supported in hipstar and which CUDA version, unified shared memory etc. is not.} Implementation of this feature without a shared address space requires either tracking the device pointers and updating them before sending data to GPU, which is complex, error-prone, and in the general case impossible to perform at compile time in a watertight manner. Therefore, \textit{hipstar} requires a shared virtual address space across the host and the device(s).

\pj{TODO: OpenCL 3.0 requires coarse grained SVM of the minimal implementation, should we state it here clearly?}

Related to memory management, a requirement \textit{hipstar} places for the supported OpenCL platforms is coarse grained Shared Virtual Memory (SVM).
\pj{Michal/Henry (?) TODO: Describe how we use it. When we do unmap/map? Before and after kernel launch?}
\mb{We map at allocation time, and also at hipMemset \& hipMemcpy time. TBH does not make sense to me, but perhaps i'm missing something. At kernel launch AFAICT only the global variables are mapped/unmapped, which i think is correct. }
\pj{unmap/map is done implicitly for coarse grained SVM at command launch time, IIRC}

\pj{We should fix \url{https://github.com/CHIP-SPV/chip-spv/issues/211} and document it here ;)}
\mb{Fixing \#211 would 1) require help from backends, and also 2) we need to find out if it's legal to pass host pointers without hipHostRegister }
\pj{we discussed this yesterday with Henry, should not be possible if not unified memory is not supported by the platform, so the worst case implementation should work -- or which backend help you refer to?}

\subsection{Task Synchronization}

...
The CUDA/HIP event API does not directly map to the OpenCL event API, since in CUDA/HIP the user is responsible for creating and recording events, while in OpenCL the implementation creates and records events when queuing commands. Recording is implemented by creating a new Marker-type cl\_event (clEnqueueMarkerWithWaitList) when hipEventRecord is called. \pj{Michal (?) TO expand} \mb{not sure what else to write here}

\pj{TODO: Asynchronous stream execution.}

\subsection{Lower Layer API Interoperability}

\pj{TODO Michal, Brice (?): Describe the HIP-OpenCL and HIP-SYCL interoperability APIs and their use cases.}

The native interoperability API can be used to initialize HIP context (with assigned device \& command queue) from a set of native (LevelZero/OpenCL) object handles, or in the opposite direction to retrieve a set of native handles from an existing HIP context. Thread-safe use of handles is currently left to the application (which should be non-issue with OpenCL since it is thread-safe). Additionally, there are two APIs that create a HIP event from native event handle, and vice-versa. These can be used for chaining HIP code with native code while maintaining asynchronous execution.

%\subsection{OpenCL-CUDA/HIP Compatibility Gaps}

%\pj{Pekka TODO: This is a verbatim copy from HIPCL, to update:}
%Most of the HIP API maps trivially to the OpenCL API, with some notable exceptions which might call for new OpenCL extensions:\pj{TODO: We should just make them extensions (proposals) to clean up the story.}

%\mb{Pekka TODO: do we also list APIs which can be implemented but aren't yet (because nobody's done the work) ? looking quickly at CHIPBindings.cc, there are >50 hip API functions which have not been implemented, things like Peer2peer, hipIPC*, hipModuleOccupancy*, hipProfiler*, hipMemPool*, hip{Malloc,Free}Async etc; some might require OpenCL extensions }

%\begin{itemize}

%\item {hipGetDeviceProperties()}: for certain device properties, there is no portable way to get the information via the OpenCL device query API.\pj{this should be an easy extension}

% Pekka> I think we can do without this as it's visible only in terms
% of latency/performance to the user, and there should be also other
% similar features which can be observed only in terms of perf., not
% functional correctness (e.g. the typical concurrency to parallelism mapping).
%\item {hipSetDeviceFlags()}: the flags to this call control how the host thread interacts with the driver thread while waiting for the device (yield the host thread to OS, or spin wait).
%
%\item{hipEventCreateWithFlags()}: provides per-event control of the synchronize behaviour (yield thread/spin wait). However, these APIs affect only performance, not correctness, thus can be implemented as no-operations.

% Pekka> Cannot we really implement this without an extension even if we had kernel metadata to traverse? \mb{possibly, if we figure out the correct alignments & padding}
%\item {hipModuleLaunchKernel()}: passing args by ``extra'' parameter requires an API for setting all kernel arguments at once.\pj{a new clEnqueueNDRange variation with a HSA-style-specified exact layout argument buffer layout might be useful in any case.}

%\item {hipGraph API}: the API to create, update & launch graphs. The existing cl_khr_command_buffer extension is not sufficient, since we need to work with SVM. (discussed below in the "opencl and spirv extensions" section).

%\item {hipHostRegister}: we'll need an OpenCL extension to implement this (unless there is something already we could use, i haven't checked).

%\end{itemize}

%%%%%%%%%%%%%%%%%%%%%%%%%%%%%%%%%%%%%%%%%%%%%%%%%%%%%%%%%%%%%%%%%%%%%%%%%

\section{Device Program Compilation}
\label{section:compilation}

% TODO: points related to compilation, how we make it portable and what type
% of LLVM transformations are needed

\subsection{Program Intermediate Representation}

A major design decision for \textit{hipstar} was to choose a \textit{portable} kernel program \textit{Intermediate Representation (IR)} format that is supported by multiple OpenCL implementations and that also has solid open source infrastructure available. While the main program of a heterogeneous application is typically compiled to a native instruction-set binary of a selected host CPU, using an IR target and ``online compilation'' for the device programs enables portability of the application binary across a diversity of co-processors. 

Overall, SPIR-V~\cite{SPIRV} seemed the best choice for the \textit{hipstar}’s ``fat binary'' format’s IR.\pj{emphasize that it's a cross-vendor committee driven standard, in good and in bad} It is supported both in later OpenCL standard as well as Vulkan. It also has good LLVM-based conversion tools available as open source. Therefore, SPIR-V support was set early on as a requirement for \textit{hipstar} supported OpenCL targets.  Unfortunately, its consumption support is not widespread in OpenCL implementations at the time of this writing, but hopefully this will change as the OpenCL implementations and input producers such as \textit{hipstar} mature. Notably, neither AMD's nor NVIDIA's OpenCL drivers support SPIR-V at the time of this writing. This should not be a significant problem in terms of HIP/CUDA portability since both platforms have direct support for them through their own implementations.

\subsection{Clang/LLVM-based Compilation Flow}

The compilation flow is based on the LLVM Project's~\cite{LLVM} Clang~\cite{Clang} frontend. The compiliation process is shown in Fig.\ref{fig:compilation}. It relies heavily on the CUDA and HIP frontend of Clang, which was updated to produce SPIR-V binaries instead of PTX or AMDIL for the device program. 

\begin{figure*}
    \centering
    \includegraphics[width=\linewidth]{figs/compilation.pdf}
    \caption{\pj{Henry (?) TODO: This is copied from HIPLZ -- redraw and adapt to the current status of \hipstar with the OpenCL in focus and Clang, LLVM tooling + the extra passes we run visible.}}
    \label{fig:compilation}
\end{figure*}

Most of the compilation related changes have been upstreamed to the LLVM project and very little compilation related functionality remains in the \hipstar project. The notable exceptions are: \pj{printf} \pj{abort} \pj{globals} \pj{what else} \pj{maybe we should implement the divergent/exited-thread barrier detection based on divergence analysis?}

\pj{Mention what was upstreamed to the LLVM Project to make hipstar work.}
\pj{Identify the hipstar-specific LLVM passes yet to upstream.}

\pj{Henry (?) TODO: Discuss the eager compilation slowness problem and how it was solved in hipstar and PoCL-level0.}

\subsection{Handling Warp-Level Primitives}

CUDA and HIP platforms have a finer grained grouping of the threads (OpenCL work-items) executing the blocks (work-groups) called a warp. In the earlier CUDA versions \pj{clarify}, the threads in a warp could be assumed to execute in lock-step, implying that the enabled threads in the same warp would execute the same instruction. This implied that in some cases explicit synchronization could be omitted: In case of a usual read-modify-update case, the programmer could trust that the warp's threads all execute the read part before any of them proceed to the update part, enabling in-place-updates without explicit synchronization. However, with the later\pj{specify} releases of CUDA relying on lock-step behavior was deprecated~\cite{NVIDIAProgrammersManual}. ...

However, the fixed size warps (32 threads for NVIDIA and usually 64 threads in AMD devices) affect the execution semantics when executing warp-level primitives that rely on the warp grouping and the mapping of the threads to the lanes of the warp.  Such primitives include the warp shuffles, which read data from a specific lane within the warp, and the explicit warp synchronization primitives. 

\pj{There could be a figure here with possible subgroup id mappings and how warps always map the threads in linear order.}
The base OpenCL specification, on the other hand, doesn't have a warp concept, but the work-items are free to make progress in any order and grouping. It has an extension called ``subgroups'' which can be used to implement the warp semantics when the kernel needs them. In contrast to  warps which have a specified form and content which allows the programmer to utilize them reliably, the basic subgroups of OpenCL are ``implementation-oriented''; they enable grouped execution in a manner that the is simplest or most efficient for the driver: The sizes of the subgroups are not fixed, but must be queried per kernel by the programmer in the basic extension. Also the way work-items are mapped to subgroup lanes (so they can be referred to when using cross-lane intrinsics) is also implementation-defined. To close the gap between subgroups and warps, an incremental specification extension \textit{cl\_intel\_reqd\_subgroup\_size} that \textit{forces} the subgroup size of the kernel to the desired size along with the linear id mapping was proposed.

% HIP doesn't support the new _sync-ones, so let's focus on it.
% Maybe also in the title of the paper.
%\pj{Pekka TODO: Non-uniform primitives.}

\subsection{Kernel Library}

\pj{Pekka TODO: This is a verbatim copy from HIPCL, to update:}
The \textit{hipstar} kernel library is an implementation of the HIP math API, by using the OpenCL C math builtins, where possible.
Many of the interfaces in the math API have an equivalent OpenCL builtin. Unfortunately, there are some functions which don't map directly, and thus require software based emulation:

\pj{TODO: About the textures implementation.}
\pj{TODO: About the CUDA intrinsics and their accuracy requirements not mapping to relaxed math limits of OpenCL and the native* OpenCL intrinsics not having any accuracy req. at all.}
\mb{TODO there is also a lot of math "intrinsics" - which i think map directly to GPU instructions in CUDA - and most of these are not implemented (a few have a native\_XYZ equivalent in OpenCL)}

\begin{itemize}

\item {Atomics on floating-points}: Do not exist in OpenCL, though they can be implemented e.g. with a CAS loop, with a performance penalty.\pj{TODO: We should. Slow is better than nothing at all.} \mb{this (CAS loop implementation) was already done in HIPCL time. There are also two SPIRV extensions: SPV\_EXT\_shader\_atomic\_float\_add \& SPV\_EXT\_shader\_atomic\_float\_min\_max, and Level Zero has an extension "ZE\_extension\_float\_atomics" so we should somehow use that. }

\end{itemize}

\subsection{OpenCL and SPIR-V Extensions}

Although the minimal OpenCL 3.0 feature set and the v1.2 of the SPIR-V OpenCL Profile\pj{?} covers most of the needs to support CUDA and HIP, some details require or improve from specification extensions. \textit{Hipstar} compilation flow is built in a way that different advanced OpenCL features and extensions are not required from the target platform's driver unless the compiled input application specifically needs them. In Table~\ref{table:extensions} we summarize the various extensions \textit{hipstar} can use and which CUDA/HIP feature triggers their need. The extensions are in different stages in the Khronos Group standardization pipeline, which is also noted in the table.~\footnote{We will update the status for the final publication.} 

\begin{table*}[ht]
    \centering

    \begin{tabular}{|p{6 cm}|p{6cm}|p{6cm}|}
    \hline
\textbf{Extension name} & \textbf{CUDA/HIP feature(s)} & \textbf{Status} \\
    \hline
cl\_ext\_alive\_only\_barrier       & A special work-group barrier for barrier calls which might not be reached by work-items that have exited the kernel. & To be proposed. \\
    \hline 
cl\_ext\_command\_buffer\_host\_data & CUDA graphs which transfer data between the host and the device. & To be proposed. \\
    \hline 
cl\_ext\_command\_buffer\_host\_sync & CUDA graphs which synchronize with the host require an extended command buffering mechanism. & To be proposed. \\
    \hline
cl\_ext\_cuda\_prec\_intrinsics     & Proposed for the future to implement math intrinsics with reduced precision requirements which do not map to the relaxed precision limits of OpenCL, thus require an extension. & To be proposed. Currently not used by \textit{hipstar}.  \\
    \hline
cl\_ext\_device\_side\_abort        & Proposed for the future to implement \_\_trap() on the low-level runtime side. & Public, potential adopters considering. Currently not used by \textit{hipstar}. \\
    \hline 
cl\_ext\_extended\_device\_properties & hipGetDeviceProperties() can be used to query more device properties than the basic OpenCL device query API supports & To be proposed. \\
    \hline                          
cl\_ext\_relaxed\_printf\_address\_space &  Proposed for the future to unify printf() behavior with non-constant address spaces. & Public, potential adopters considering. Currently not used by \textit{hipstar}. \\
    \hline
cl\_intel\_required\_subgroup\_size & When calling warp-level primitives that depend on the 32/64 warp size or the thread id ordering. & Promotion to a generalized 'khr' extension being discussed. \\ 
    \hline
cl\_khr\_command\_buffer            & Optimized implementation of CUDA graph execution. Extended with SVM commands. & Public. \\
    \hline                          
cl\_khr\_fp64                       & If double precision floating point is used. & Public. \\
    \hline 
cl\_khr\_global\_int32\_base\_atomics \newline
cl\_khr\_global\_int32\_extended\_atomics \newline
cl\_khr\_local\_int32\_base\_atomics  \newline
cl\_khr\_local\_int32\_extended\_atomics \newline
cl\_khr\_int64\_base\_atomics \newline
cl\_khr\_int64\_extended\_atomics & Atomic operations. & Public. \\
    \hline
cl\_khr\_subgroups                  & Warp-level synchronization with \_\_syncwarp(). & Public. \\
    \hline
cl\_khr\_subgroup\_ballot           & Warp-level ballot operations. & Public. \\
    \hline
cl\_khr\_subgroup\_shuffle          & Warp-level shuffle operations. & Public. \\
    \hline
    
    \end{tabular}
    \caption{OpenCL 3.0 and SPIR-V standard extensions that \textit{hipstar} might use to implement certain CUDA/HIP features. Status is the state of the extension at the time of this article's publication.\pj{TODO: double check that we didn't forget any extension.}}
    \label{table:extensions}
\end{table*}

Most of the extensions are straightforward and the explanation in the table cell suffices to grasp their purpose. However, X and Y require further discussion: \pj{...}
\pj{Note that the warp-primitives section already discussed reqd SG size}



%%%%%%%%%%%%%%%%%%%%%%%%%%%%%%%%%%%%%%%%%%%%%%%%%%%%%%%%%%%%%%%%%%%%%%%%%

\section{Libraries}
\label{section:libraries}

The CUDA software platform and as implication, ROCm, include a set of useful libraries in addition to the general purpose program input. These libraries include common routines such as BLAS (Basic Linear Algebra Subprograms) and Deep Neural Network (DNN) acceleration libraries.

In order to support nearly drop-in compile-time compatibility with various applications that utilize either the CUDA or HIP libraries, we have ported a selection of them in a fashion that they can utilize and interoperate with the \textit{hipstar} platform. We discuss the libraries in the following and highlight their essential technical aspects.

\subsection{cuBLAS / hipBLAS}

\pj{TODO Peng, Sarbojit (?)}

\subsection{cuDNN / MIOpen}

\pj{TODO: SYCL-DNN. Involve Codeplay with this paper?}

\subsection{rocPRIM for CUB compatibility}

\pj{TODO: test with cub}

\subsection{CUDA Graphs / MIGraphX}

\pj{TODO: mapping the Graph API to the command buffer API}

%%%%%%%%%%%%%%%%%%%%%%%%%%%%%%%%%%%%%%%%%%%%%%%%%%%%%%%%%%%%%%%%%%%%%%%%%

\section{Expanding Portability with Open Source OpenCL Implementations}
\label{section:expandingCompat}

Although OpenCL has again risen in popularity in the recent years, thanks to its more easily implementable OpenCL 3.0 version, the main portability gap is SPIR-V input, which is not mandated by the specification. As of this writing, only ARM and Intel support SPIR-V input to their GPU offerings in their proprietary drivers.

Fortunately there are now various active OpenCL implementations that can be used and expanded to fill up the lack of features in the proprietary implementations until they catch up. The two most vibrant ones are Rusticl~\cite{RustiCLWeb} and Portable Computing Language (PoCL~\cite{poclIJPP}). These two projects can be utilized to extend the portability to X and Y, and interestingly, back to CUDA platforms by using the PoCL-CUDA driver.

...
\pj{Discuss how we can have end-to-end testing with PoCL and support more targets thanks to its SPIR-V input.}
...
\pj{Looping Back to NV with PoCL-CUDA and/or RustiCL}
...
\pj{TODO: A portability graph which shows the layers, required extensions, and platforms that support it. Similar to https://github.com/pocl/pocl/files/10957913/sw-stack-graph.pdf}

%%%%%%%%%%%%%%%%%%%%%%%%%%%%%%%%%%%%%%%%%%%%%%%%%%%%%%%%%%%%%%%%%%%%%%%%%

\section{CUDA/HIP Feature Coverage}
\label{section:directCUDA}

\pj{Discuss how easy it is to add direct CUDA support too, but highlight the compatibility gaps in the next section.}
\pj{Cite and discuss the USA highest court legal ruling related to Java API, emphasize that not a lawyer, and not a focus.}

\subsection{Remaining Compatibility Gaps}
\label{subsection:compatgaps}

\pj{Discuss features specific to CUDA, from the doc I wrote in Parmance.}


% TODO: the OpenCL extensions identified and proposed.

%%%%%%%%%%%%%%%%%%%%%%%%%%%%%%%%%%%%%%%%%%%%%%%%%%%%%%%%%%%%%%%%%%%%%%%%%

\section{Debugging and Profiling Support}
\label{section:debuggingAndProfiling}

Thanks to using the open standard OpenCL as the portability layer, various debugging and profiling tooling options are available to use with little to no additional effort. The following discusses a set of tools that were successfully tested and adopted for $hipstar$ for profiling and debugging HIP programs.

\subsection{Profiling Tools}

\subsubsection{VTune}

\pj{TODO Paulius?}

\subsubsection{Tracing Heterogeneous APIs (THAPI)}

% https://github.com/argonne-lcf/THAPI

\pj{TODO Brice?}

\subsection{Debugging}

\pj{TODO: PoCL-CPU, GDB, Valgrind, debug info...}

%%%%%%%%%%%%%%%%%%%%%%%%%%%%%%%%%%%%%%%%%%%%%%%%%%%%%%%%%%%%%%%%%%%%%%%%%

\section{Platform Portability Experiments}
\label{section:platformPortability}

\subsection{AMD and Intel CPUs}

\pj{Using Intel OpenCL CPU and/or PoCL-CPU}

\subsection{ARM Mali GPU}

\pj{Since ARM Mali supports SPIR-V, we could run a portability experiment with CUDA/HIP running on a smartphone. I can try to ask someone from TAU to do this. But even better, Brice, could we involve Kevin Petit to help with this? It would not be much to ask given they are getting HIP/CUDA input basically for free (well, how much benefit that has on Android platforms is questionable)...}

\subsection{RISC-V}

\pj{If we manage to do the port in time, we could show execution on RISC-V CPUs.}

\subsection{CUDA}

\pj{Back to CUDA via PoCL-CUDA}

%%%%%%%%%%%%%%%%%%%%%%%%%%%%%%%%%%%%%%%%%%%%%%%%%%%%%%%%%%%%%%%%%%%%%%%%%

\section{Application Portability Validation}
\label{section:applications}

In order to validate the drop-in compatibility level of \textit{hipstar}, we ported a set of well-known applications and libraries used widely in HPC that target either the CUDA or HIP API. The applications along with any source code modifications required are outlined in the following.

\pj{TODO: also highlight which compatibility library was tested}
\pj{If there's not much to write about each ported app, we could create a table of them and have columns which HIP/hipstar features or libraries they use.}

\subsection{CP2K}

\pj{TODO who writes?}

\subsection{Exabio}

\pj{TODO who writes?}

\subsection{GAMESS}

\pj{TODO: Colleen? Evaluate how well it works in terms of functionality and compare to another platform in perf.}

\subsection{libCEED}

\pj{TODO: Colleen?}

\subsection{Pytorch-HIP}

\pj{TODO: Henry?}

%%%%%%%%%%%%%%%%%%%%%%%%%%%%%%%%%%%%%%%%%%%%%%%%%%%%%%%%%%%%%%%%%%%%%%%%%

\section{Performance Evaluation}
\label{section:performance}

In this section we evaluate the performance of \textit{hipstar}  with specific focus on overheads cause by layering HIP and CUDA on top of other API layers such as OpenCL and Level Zero.

\subsection{Execution Performance}

\pj{TODO: Which apps and which GPUs we should use here?}

\subsubsection{GeekBench}

% GeekBench 5.2.3  - 
\pj{TODO: Compare to https://github.com/vosen/ZLUDA, if possible}

\subsection{Layering Overheads}

\pj{TODO: Evaluate how much overhead on top of straight HIP/CUDA to CUDA vs. through PoCL-CUDA}
\pj{TODO: Evaluate how much overhead going through PoCL-level0 incurs on top of straight LZ BE}

\pj{Straight to CUDA vs. CUDA to HIP to hipstar to OpenCL + to OpenCL to PoCL-CUDA to CUDA}

\subsection{Graphs on Command Buffers}

\pj{Michal (?) TODO: numbers on command buffer benefits here?}

\pj{TODO: Comparisons}
\pj{TODO: Overhead analysis}

%%%%%%%%%%%%%%%%%%%%%%%%%%%%%%%%%%%%%%%%%%%%%%%%%%%%%%%%%%%%%%%%%%%%%%%%%

\section{Related Work}
\label{section:relatedWork}

The origin of \textit{chipstar} is on the HIPCL~\cite{HIPCL} prototype which first tested the concept of compiling HIP programs to fat binaries relying on OpenCL and SPIR-V. The \chipstar tool described in this article is a result of an almost a complete rewrite of the HIPCL code base and over approximately three years of continuous development work by multiple partners and HPC users. The HIPCL code base was initially forked to a separate code base to utilize the Level Zero~\cite{l0} low level API directly (HIPLZ~\cite{HIPLZ}) after which the OpenCL backend of HIPCL and the Level Zero backend of HIPLZ were merged to the same code base discussed in this article.
A large number of missing essential features have been implemented since the initial prototypes were published. This article thus significantly expands upon the original poster abstract that introduced the early-prototype-stage HIPCL and now presents a much more mature software stack usable for a wider range of real-world workloads.
%The direct Level Zero access is used as an additional backend for comparison purposes in this article, with the primary focus being on the OpenCL backend.

%However, at the time of this writing, the recommended path from CUDA/HIP to Level Zero goes through the OpenCL backend and PoCL's~\cite{poclIJPP} Level Zero backend since the OpenCL code path has matured longer and is somewhat more robust.

When comparing \chipstar to other HIP implementations, obviously the original ROCm, the AMD's official GPU software platform~\cite{ROCm}, is the baseline. ROCm consists of the general purpose programming API compilation and runtime support for HIP, and a set of libraries that support different degrees of compatibility with the CUDA platform. \chipstar is not a new backend in addition to the AMD GPU and NVIDIA GPU backends provided by the AMD's offering, but has an important technical difference: \chipstar aims to offer runtime portability by its open standard based fat binary, removing the need to recompile the input software per target vendor platform, which is the case with ROCm.
%\pj{Brice/Paulius: Can you check that this is (still) true?}

%HIP is very close to CUDA, and in fact AMD provides a source-to-source translation tool called HIPify that can automate the porting process. Interestingly, although heavily based on the NVIDIA-driven CUDA, AMD now promotes HIP as the primary C++ programming API for their GPU platforms. Since AMD GPUs have increased their market share and received major design wins in large HPC installations, HIP as such has risen in importance as an application-facing interface.

SYCLomatic~\cite{SYCLomatic} is a tool contributed by Intel Corporation for converting CUDA sources to the cross-vendor open standard SYCL~\cite{SYCL}. Similar to AMD's HIPify, but in contrast to \chipstar which aims for source-level compatibility, SYCLomatic is a source-to-source conversion tool, which has its good and bad sides. The most apparent implication of relying on source-to-source conversion is more about maintenance aspects than technical ones; it neccessitates the further development of the converted application to proceed using the SYCL API instead or in addition to CUDA. The main drawback is that in reality many code bases are difficult or impossible to convert solely to SYCL without having the CUDA version as a backup due to legacy, risk-management or technical reasons. The main benefit is that SYCL is an open standard, in constrast to CUDA, enabling more fair competition ground between hardware vendors. Thus, being able to target many platforms from a fat binary compiled from the unmodified CUDA/HIP source code base using a \chipstar-style open platform approach can have its benefits. Furthermore, since \chipstar is not a linkage-time or binary translation solution, but requires recompilation, it coincidentally also encourages utilizing and further developing the cross-vendor ecosystem APIs it relies upon. Furthermore, as of this writing SYCLomatic supports only CUDA, not HIP, while HIP has an increasing number of new applications implemented directly using it.

% https://github.com/vosen/ZLUDA
ZLUDA~\cite{ZLUDA} is a tool for running unmodified CUDA binaries on AMD GPUs. It works by reimplementing the \cuda driver API, and converting NVIDIA PTX~\cite{ptx} to the vendor-specific IRs. Since it is a ``drop-in solution'' that works at program loading/linkage time, it can execute unmodified CUDA fat binaries, which is very comfortable to the end users as it doesn't require access to the source code of the application. While we see ZLUDA as an excellent tool, it requires reverse engineering CUDA SDK's binary interfaces and keeping up-to-date with the NVIDIA PTX as it evolves. We believe, in the longer term, especially as more of the missing extensions we describe are adopted by OpenCL implementations, \chipstar can provide a more robust solution. 
%Of course only time will tell how well this turns out to be the case. 
In addition, ZLUDA also doesn't support HIP as an input and now only targets AMD GPUs, whereas a key goal of \chipstar is extensive cross-vendor portability.

%However, its developed has stalled and it only supports a limited subset of applications and only on the Intel devices supported by the Level Zero API. ZLUDA author claims in their web page that they can achieve performance benefits when running straight on top of the lower level \lz instead of the somewhat higher level OpenCL. Since \chipstar supports both, we were able to measure this difference accurately, and found it to be negligible\pj{to do actually}.
%A key benefit of skipping a cross-vendor standardized layer is that PTX has instructions which map directly to the Intel GPU instructions which are not exposed in OpenCL C.
%Although \chipstar uses the OpenCL runtime for portability, it targets SPIR-V instead of OpenCL C as the device-side programming language, thus this drawback does not appear with it. The potential overhead is first passing through LLVM IR, which might lose beneficial information, but that also is found not to be an issue according to the measurements presented in Section~\ref{TODO} \pj{to do actually}.

MCUDA~\cite{MCUDA} is the oldest tool we found for porting CUDA programs to non-NVIDIA platforms. MCUDA does source-to-source translation of kernels in a fashion that the translated kernels can execute efficiently on CPUs on a single CPU thread while respecting the barrier synchronization. In the case of \chipstar, since it uses OpenCL as its portability layer, it can similarly target also vectorized CPU execution through CPU-targeting OpenCL implementations such as the Intel OpenCL CPU driver and PoCL's CPU drivers~\cite{PoCL}. Both of them are capable of vectorizing work-items (CUDA/HIP threads) inside work-groups, which translates to implicit autovectorization of CUDA/HIP kernels across CUDA threads and provide the benefits of CPU execution such as easier kernel debugging.

Swan~\cite{Swan} is another early source-to-source tool for CUDA porting. It generates OpenCL code from CUDA, providing similar level of portability as \chipstar does. Another similar tool, CU2CL~\cite{CU2CL} was published in the same year as \cite{Swan}. Neither Swan nor CU2CL are maintained any longer.  In comparison to \chipstar, the main technical differences to these tools are that \chipstar utilizes the latest version of the OpenCL standard to support the newer CUDA/HIP features, uses SPIR-V as the intermediate language (no need to generate textual OpenCL C with its limitations) and it doesn't suffer from problems related to source-to-source translations as \chipstar provides source-level compatibility.

The closest comparable CUDA porting tool we could find is CUDA-on-CL~\cite{CUDAonCL}. Like \chipstar, it similarly compiles CUDA programs using Clang/LLVM-based compiler chain to binaries which then execute on OpenCL platforms. However, similarly to Swan and CU2L, it compiles device kernels to OpenCL C whereas \chipstar uses SPIR-V as the portable binary format. Other technical differences in \chipstar are related to the use of modern OpenCL standard features to implement some of the features of CUDA. These include using SVM to implement raw pointers and implementing warp-level primitives such as shuffles using the subgroup features. 




%%%%%%%%%%%%%%%%%%%%%%%%%%%%%%%%%%%%%%%%%%%%%%%%%%%%%%%%%%%%%%%%%%%%%%%%%

\section{Conclusions and Future Work}
\label{section:conclusions}

\pj{TODO: Drop-in/link-time replacement (CUDA binary compatibility)?}

\bibliography{IEEEabrv,hipstar}
\bibliographystyle{IEEEtran}

\end{document}
