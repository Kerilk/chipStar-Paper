\section{Unimplemented HIP APIs}
There are a total of 107 HIP API calls that are unimplemented in the current version. 
None of the following APIs have been required in the applications that have been tested which is how features are prioritized for implementation.
Some of these can be trivially implemented, others might require additional extensions to OpenCL.

\begin{itemize}

    \item Mipmapped Array Management:
    This API allows for the creation and management of mipmapped arrays in device memory.
    OpenCL does not support mipmapped arrays thus an extension would be required.
    \begin{itemize}
        \item \texttt{hipFreeMipmappedArray}: Frees a mipmapped array
        \item \texttt{hipGetMipmappedArrayLevel}: Gets a specific level of a mipmapped array
        \item \texttt{hipMallocMipmappedArray}: Allocates a mipmapped array in device memory
    \end{itemize}
    
    \item 3D Array Management:
    This API allows for the creation of 3D arrays in device memory.
    \begin{itemize}
        \item \texttt{hipArray3DCreate}: Creates a 3D array in device memory
        \item \texttt{hipGetChannelDesc}: Gets the channel descriptor for an array
    \end{itemize}
    
    \item Device Management:
    \begin{itemize}
        \item \texttt{hipDeviceGetLimit}: Gets the limit for a given device attribute
        \item \texttt{hipDeviceGetP2PAttribute}: Gets peer-to-peer attributes between devices
        \item \texttt{hipDeviceGetUuid}: Gets a UUID for a device
        \item \texttt{hipDeviceSetLimit}: Sets the limit for a given device attribute
        \item \texttt{hipGetDeviceFlags}: Gets the flags for the current device
    \end{itemize}
    
    \item Virtual Memory Management:
    This group of APIs allows for the allocation and mapping of virtual addresses. 
    This API API allows developers to explicitly manage large datasets by decoupling virtual and physical memory.
    It enables dynamic mapping and unmapping of memory, efficient reuse of limited physical memory, and fine-grained control over memory access permissions.
    The main benefits include handling datasets larger than physical memory, reducing fragmentation, and improving performance in applications with complex memory management needs. 
    This functionality is not supported in OpenCL thus an extension would be required.
    \begin{itemize}
        \item \texttt{hipMemAddressReserve}: Reserves a virtual address range
        \item \texttt{hipMemAddressFree}: Frees a virtual address range
        \item \texttt{hipMemMap}: Maps previously-allocated memory
        \item \texttt{hipMemUnmap}: Unmaps previously mapped memory
        \item \texttt{hipMemGetAllocationGranularity}: Gets memory allocation granularity
        \item \texttt{hipMemCreate}: Creates a memory allocation handle
    
    \item Memory Management:
    \begin{itemize}
        \item \texttt{hipMemAllocPitch}: Allocates pitched memory
        \item \texttt{hipMemcpyPeer}: Peer-to-peer memory copy
        \item \texttt{hipMemcpyPeerAsync}: Asynchronous peer-to-peer memory copy
    \end{itemize}
    
    \item Stream Ordered Memory Allocator:
    Stream Ordered Memory Allocator (SOMA) allows for asynchronous memory allocation and deallocation. 
    This is useful for applications that require high-throughput memory allocations and deallocations, 
    such as deep learning training and inference frameworks. Given chipStar's current focus on classical HPC, 
    this feature has not been a priority. See \section{AI/ML Support} for more details.
    \begin{itemize}
        \item \texttt{hipMallocAsync}: Asynchronous memory allocation
        \item \texttt{hipFreeAsync}: Asynchronous memory deallocation
        \item \texttt{hipDeviceGetDefaultMemPool}: Gets the default memory pool for a device
        \item \texttt{hipDeviceGetMemPool}: Gets the current memory pool for a device
        \item \texttt{hipDeviceSetMemPool}: Sets the current memory pool for a device
        \item \texttt{hipMemPoolCreate}: Creates a memory pool
        \item \texttt{hipMemPoolDestroy}: Destroys a memory pool
        \item \texttt{hipMemPoolExportPointer}: Exports a memory pool pointer
        \item \texttt{hipMemPoolExportToShareableHandle}: Exports a memory pool to a shareable handle
        \item \texttt{hipMemPoolGetAccess}: Gets memory pool access permissions
        \item \texttt{hipMemPoolGetAttribute}: Gets memory pool attributes
        \item \texttt{hipMemPoolImportFromShareableHandle}: Imports a memory pool from a shareable handle
        \item \texttt{hipMemPoolImportPointer}: Imports a memory pool pointer
        \item \texttt{hipMemPoolSetAccess}: Sets memory pool access permissions
        \item \texttt{hipMemPoolSetAttribute}: Sets memory pool attributes
        \item \texttt{hipMemPoolTrimTo}: Trims a memory pool to a specified size
        \item \texttt{hipMallocFromPoolAsync}: Asynchronously allocates from a memory pool
    \end{itemize}
    
    \item IPC (Inter-Process Communication):
    \begin{itemize}
        \item \texttt{hipIpcCloseMemHandle}: Closes an IPC memory handle
        \item \texttt{hipIpcGetEventHandle}: Gets an IPC event handle
        \item \texttt{hipIpcGetMemHandle}: Gets an IPC memory handle
        \item \texttt{hipIpcOpenEventHandle}: Opens an IPC event handle
        \item \texttt{hipIpcOpenMemHandle}: Opens an IPC memory handle
    \end{itemize}
    
    \item Stream and Event Management:
    \begin{itemize}
        \item \texttt{hipExtStreamCreateWithCUMask}: Creates a stream with compute unit mask
        \item \texttt{hipExtStreamGetCUMask}: Gets the compute unit mask for a stream
        \item \texttt{hipStreamAttachMemAsync}: Attaches memory to a stream asynchronously
        \item \texttt{hipStreamGetCaptureInfo}: Gets stream capture information
        \item \texttt{hipStreamGetDevice}: Gets the device associated with a stream
        \item \texttt{hipStreamIsCapturing}: Checks if a stream is being captured
        \item \texttt{hipStreamUpdateCaptureDependencies}: Updates stream capture dependencies
        \item \texttt{hipStreamWaitValue32/64}: Waits for a 32/64-bit value in a stream
        \item \texttt{hipStreamWriteValue32/64}: Writes a 32/64-bit value in a stream
        \item \texttt{hipThreadExchangeStreamCaptureMode}: Exchanges stream capture mode
    \end{itemize}
    
    \item Cooperative Kernel Management:
    Cooperative kernels allow for performance synchronization across the entire kernel invocation, 
    not just within a threadblock. This functionality is not supported in OpenCL thus an extension would be required.
    \begin{itemize}
        \item \texttt{hipLaunchCooperativeKernel}: Launches a cooperative kernel
        \item \texttt{hipLaunchCooperativeKernelMultiDevice}: Launches cooperative kernels on multiple devices
        \item \texttt{hipModuleLaunchCooperativeKernel}: Launches a cooperative kernel from a module
    \end{itemize}
    
    \item Kernel and Function Management:
    \begin{itemize}
        \item \texttt{hipFuncGetAttribute}: Gets function attributes
        \item \texttt{hipFuncSetAttribute}: Sets function attributes
        \item \texttt{hipFuncSetSharedMemConfig}: Sets shared memory configuration
        \item \texttt{hipLaunchHostFunc}: Launches a host function in a stream
        \item \texttt{hipModuleGetTexRef}: Gets a texture reference
        \item \texttt{hipSetupArgument}: Sets up kernel arguments (legacy API)
    \end{itemize}
    
    \item Occupancy and Performance:
    \begin{itemize}
        \item \texttt{hipModuleOccupancyMaxActiveBlocksPerMultiprocessor}: Calculates occupancy
        \item \texttt{hipModuleOccupancyMaxActiveBlocksPerMultiprocessorWithFlags}: Calculates occupancy with flags
        \item \texttt{hipModuleOccupancyMaxPotentialBlockSizeWithFlags}: Calculates potential block size with flags
        \item \texttt{hipOccupancyMaxActiveBlocksPerMultiprocessor}: Calculates maximum active blocks
        \item \texttt{hipOccupancyMaxActiveBlocksPerMultiprocessorWithFlags}: Calculates maximum active blocks with flags
        \item \texttt{hipOccupancyMaxPotentialBlockSize}: Calculates potential block size
        \item \texttt{hipProfilerStart}: Starts the profiler
        \item \texttt{hipProfilerStop}: Stops the profiler
    \end{itemize}
    
    \item Graph Management:
    Currently, chipStar provides a basic implementation of the graph API. 
    Graph operations are recorded and executed by replaying the same sequence of non-graph APIs. 
    While most unit tests pass, the naive implementation provides no performance benefits over the non-graph APIs.
    Properly implemeting the Graph API requires support for the \texttt{cl\_khr\_command\_buffer} extension. 
    Currently, only the PoCL implementation supports this extension.
    
    \begin{itemize}
        \item \texttt{hipGraphAddMemAllocNode}: Adds memory allocation node to a graph
        \item \texttt{hipGraphAddMemFreeNode}: Adds memory free node to a graph
        \item \texttt{hipGraphDebugDotPrint}: Prints graph in DOT format
        \item \texttt{hipGraphInstantiateWithFlags}: Instantiates a graph with flags
        \item \texttt{hipGraphKernelNodeCopyAttributes}: Copies kernel node attributes
        \item \texttt{hipGraphKernelNodeGetAttribute}: Gets kernel node attributes
        \item \texttt{hipGraphKernelNodeSetAttribute}: Sets kernel node attributes
        \item \texttt{hipGraphMemAllocNodeGetParams}: Gets memory allocation node parameters
        \item \texttt{hipGraphMemFreeNodeGetParams}: Gets memory free node parameters
        \item \texttt{hipGraphNodeGetEnabled}: Gets if a node is enabled
        \item \texttt{hipGraphNodeSetEnabled}: Sets if a node is enabled
        \item \texttt{hipGraphReleaseUserObject}: Releases a user object from a graph
        \item \texttt{hipGraphRetainUserObject}: Retains a user object in a graph
        \item \texttt{hipGraphUpload}: Uploads a graph to a stream
        \item \texttt{hipDeviceGetGraphMemAttribute}: Gets graph memory attributes for a device
        \item \texttt{hipDeviceGraphMemTrim}: Trims device graph memory
        \item \texttt{hipDeviceSetGraphMemAttribute}: Sets graph memory attributes for a device
    \end{itemize}
    
    \item Miscellaneous:
    \begin{itemize}
        \item \texttt{hipDrvGetErrorName}: Gets the name of a driver error
        \item \texttt{hipDrvGetErrorString}: Gets the description of a driver error
        \item \texttt{hipDrvMemcpy3D}: 3D memory copy (driver API)
        \item \texttt{hipDrvMemcpy3DAsync}: Asynchronous 3D memory copy (driver API)
        \item \texttt{hipDrvPointerGetAttributes}: Gets pointer attributes (driver API)
        \item \texttt{hipMemRangeGetAttribute}: Gets attributes of a memory range
        \item \texttt{hipMemRangeGetAttributes}: Gets multiple attributes of a memory range
        \item \texttt{hipPointerGetAttribute}: Gets a single attribute of a pointer
        \item \texttt{hipSignalExternalSemaphoresAsync}: Signals external semaphores
        \item \texttt{hipWaitExternalSemaphoresAsync}: Waits on external semaphores
        \item \texttt{hipUserObjectCreate}: Creates a user object
        \item \texttt{hipUserObjectRelease}: Releases a user object
        \item \texttt{hipUserObjectRetain}: Retains a user object
    \end{itemize}

\end{itemize}\textbf{}
\end{itemize}\textbf{}