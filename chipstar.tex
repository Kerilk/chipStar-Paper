\documentclass[Afour,sagev,times]{sagej}

% Begin IEEE Imports
\usepackage{amsmath,amsfonts}
\usepackage{algorithmic}
\usepackage{algorithm}
\usepackage{array}
\usepackage{textcomp}
\usepackage{stfloats}
\usepackage{url}
\usepackage{verbatim}
\usepackage{graphicx}
\usepackage{color}
\usepackage{amssymb}
\usepackage{xspace}
\usepackage{textcomp}
\usepackage{caption}
\usepackage{subcaption}
\usepackage{soul}

%Removed the DRAFT stamp in the hopes of this being
%the final submission ;) --Pekka
%\usepackage{draftwatermark}
%\SetWatermarkText{DRAFT}
%\SetWatermarkScale{1}
%\SetWatermarkLightness{0.9}

% https://tex.stackexchange.com/questions/326897/vertical-alignment-of-a-turned-cell
\usepackage{rotating}
\usepackage{array,makecell,multirow}

\newboolean{showhighlights}
\setboolean{showhighlights}{false}
\ifthenelse{\boolean{showhighlights}}
{ \newcommand{\myhl}[1]{\hl{#1}} }
{ \newcommand{\myhl}[1]{#1} }


\usepackage{ifthen}
\newboolean{showcomments}
\setboolean{showcomments}{true}
\ifthenelse{\boolean{showcomments}}
{ \newcommand{\mynote}[3]{
     \fbox{\bfseries\sffamily\scriptsize#1}
        {\small$\blacktriangleright$\textsf{\emph{\color{#3}{#2}}}$\blacktriangleleft$}}
  \newcommand{\newtext}[1]{{\color{orange}{#1}}}}
{ \newcommand{\mynote}[3]{}
  \newcommand{\newtext}[1]{#1}}

\usepackage[acronym]{glossaries}
\makeglossaries

\newacronym{ir}{\myhl{IR}}{\myhl{Intermediate Representation}}
\newacronym{jit}{\myhl{JIT}}{\myhl{just-in-time}}
\newacronym{isa}{\myhl{ISA}}{\myhl{Instruction Set Architecture}}
\newacronym{hsa}{\myhl{HSA}}{\myhl{Heterogeneous System Architecture}}
\newacronym{svm}{\myhl{SVM}}{\myhl{Shared Virtual Memory}}
\newacronym{cg}{\myhl{CG}}{\myhl{Coarse-grained buffer}}
\newacronym{spir}{\myhl{SPIR}}{\myhl{Standard Portable Intermediate Representation}}
\newacronym{spirv}{\myhl{SPIR-V}}{\myhl{new version of SPIR}}
\newacronym{hsail}{\myhl{HSAIL}}{\myhl{Heterogeneous system architecture intermediate language}}
\newacronym{ssa}{\myhl{SSA}}{\myhl{Static Single Assignment}}
\newacronym{dpc}{\myhl{DPC++}}{\myhl{Data Parallel C++}}
\newacronym{usm}{\myhl{USM}}{\myhl{Unified Shared Memory}}


% Please use a named note with this macro to comment the text:
\newcommand{\pj}[1]{ \mynote{PJ}{#1}{blue} }
\newcommand{\bv}[1]{ \mynote{BV}{#1}{green} }
\newcommand{\mb}[1]{ \mynote{MB}{#1}{cyan} }
\newcommand{\cb}[1]{ \mynote{CB}{#1}{magenta} }
\newcommand{\pv}[1]{ \mynote{PV}{#1}{yellow} }
\newcommand{\ba}[1]{ \mynote{BA}{#1}{brown} }
\newcommand{\kh}[1]{ \mynote{KH}{#1}{red} }


\newcommand{\hiplz}{\texttt{HIPLZ}\xspace}
\newcommand{\hipcl}{\texttt{HIPCL}\xspace}
\newcommand{\hip}{\texttt{HIP}\xspace}
\newcommand{\opencl}{\texttt{OpenCL}\xspace}
\newcommand{\lz}{\texttt{L0}\xspace}
\newcommand{\sycl}{\texttt{SYCL}\xspace}
\newcommand{\cuda}{\texttt{CUDA}\xspace}
\newcommand{\chipstar}{\textit{chipStar}\xspace}
\newcommand{\func}[1]{$#1$\xspace}
\newcommand{\type}[1]{$#1$\xspace}

% End of IEE imports

\usepackage{moreverb,url}

\usepackage[colorlinks,bookmarksopen,bookmarksnumbered,citecolor=red,urlcolor=red]{hyperref}

\newcommand\BibTeX{{\rmfamily B\kern-.05em \textsc{i\kern-.025em b}\kern-.08em
T\kern-.1667em\lower.7ex\hbox{E}\kern-.125emX}}

\def\volumeyear{2020}
\setcounter{secnumdepth}{3} 
\begin{document}

%\runninghead{TODO}

\title{\chipstar: Making HIP/CUDA Applications Cross-Vendor Portable by Building on Open Standards}

\author{
{Paulius Velesko}\affilnum{1},
{Pekka Jääskeläinen}\affilnum{4},
{Henry Linjamäki}\affilnum{4},
{Michal Babej}\affilnum{4},
{Peng Tu}\affilnum{3},
{Sarbojit Sarkar}\affilnum{3},
{Ben Ashbaugh}\affilnum{3},
{Colleen Bertoni}\affilnum{2},
{Jenny Chen}\affilnum{9},
{Philip C. Roth}\affilnum{5},
{Wael Elwasif}\affilnum{5},
{Rahulkumar Gayatri}\affilnum{6},
{Jisheng Zhao}\affilnum{7},
{Karol Herbst}\affilnum{8},
{Kevin Harms}\affilnum{2},
{Brice Videau}\affilnum{2}}

\affiliation{
\affilnum{1}PGLC Consulting
\affilnum{2}Argonne National Laboratory
\affilnum{3}Intel Corporation
\affilnum{4}Tampere University
\affilnum{5}Oak Ridge National Laboratory
\affilnum{6}National Energy Research Scientific Computing Center
\affilnum{7}Georgia Institute of Technology
\affilnum{8}Red Hat, Inc.
\affilnum{9}Purdue University
}

\corrauth{Paulius Velesko, pvelesko@pglc.io}

% Jenny requests we use jjenny0503@gmail.com for her email


%%%%%%%%%%%%%%%%%%%%%%%%%%%%%%%%%%%%%%%%%%%%%%%%%%%%%%%%%%%%%%%%%%%%%%%%%

\begin{abstract}
We describe \chipstar, an open source software stack that enables building unmodified CUDA and HIP programs into binaries that rely solely on open cross-vendor compute standards OpenCL and SPIR-V. The relevant technical aspects of \chipstar and the feature mismatches between the CUDA/HIP APIs and OpenCL are discussed along with a set of standard extension proposals to bridge the essential gaps in the future.
The key benefit of the software stack is its portability, which is demonstrated by providing performance evaluations on a diversity of less common CPU/GPU platforms including RISC-V/PowerVR and ARM Mali. A comparison against the original AMD HIP platform provides a geometric mean of 0.75, a reasonable price to pay for the enhanced portability.    
\chipstar is now considered mature enough for wider testing and even production use, which is demonstrated by successful porting and competitive performance of GAMESS-GPU-HF, a complex HPC application.

\end{abstract}

%\begin{IEEEkeywords}
\keywords{CUDA, HIP, OpenCL, SPIR-V, Portability, Shared Virtual Memory}
%\end{IEEEkeywords}

%%%%%%%%%%%%%%%%%%%%%%%%%%%%%%%%%%%%%%%%%%%%%%%%%%%%%%%%%%%%%%%%%%%%%%%%%

\maketitle

%\section{Introduction}

\noindent The walled garden strategy is popular among market-dominating companies. The idea behind walled garden strategy is to lock in customers to a company's products by making escaping the gates of the garden as costly as possible. NVIDIA's CUDA software platform is considered to be one of such walled gardens. It helps NVIDIA to expand and retain their GPU market advantage, and at the same time maintain a high innovation pace on the software APIs since there is no need to work with standardization committees that always have to aim for a consensus among multiple participating vendors.

Naturally, for end-users and the competing hardware vendors, the situation of a single-vendor dictated API is not ideal. End-users would prefer open standard software interfaces that enable switching the targeted hardware without incurring significant non-recurring engineering costs required for porting the applications and libraries to a new software platform just to be able to utilize the newly purchased hardware optimally. Similarly, other hardware vendors, aiming to get their piece of the market pie, would prefer an API that is not controlled by a single vendor.

AMD's ROCm~\cite{ROCm} software platform and its Heterogeneous-compute Interface for Portability (HIP) language~\cite{hip} helps escaping the CUDA walled garden by providing a route out from the NVIDIA CUDA platform to AMD's devices. HIP defines a subset of CUDA that is more easily portable to various hardware, thanks mainly to omitting various advanced features available in the later CUDA versions.

In order to enable an easy automated transition path from CUDA applications, HIP is largely a copy of a CUDA C/C++ API subset with a few minor differences and renamed functions. HIP alleviates the CUDA portability problem, but doesn't solve it satisfactorily due to AMD targeting their self-specified low-level ROCm APIs which are not actively supported on non-AMD platforms.
An open source HIP/CUDA software platform solely based on open standards with a sincere aim for cross-vendor portability is still lacking.

With the \chipstar software stack described in this article we aim to alleviate the CUDA/HIP application
portability problem. In contrast to previous solutions that either require source-to-source conversion from CUDA programs~\cite{SYCLomatic}, that can lead to costly multiple codebase maintenance, or aim for binary-level compatibility of existing CUDA/HIP programs~\cite{ZLUDA} that rely on questionable reverse engineering of proprietary binary interfaces -- a brittle longer-term strategy -- \chipstar chooses a middle-ground approach which enables source-level compatibility of HIP/CUDA programs by compiling them to a runtime portable ``fat binary'' (a single executable that contains multiple code objects, each compiled for a different architecture)  that utilizes solely open standards and can execute on any platform supporting the required standard features without recompilation.

With this article and the associated open source code base we make the following contributions:

\begin{enumerate}
  \item We publish internal design choices for the software platform \chipstar that enables porting applications from the NVIDIA-driven CUDA and AMD-driven ROCm platforms to any current and future platform supporting the OpenCL and \acrshort{spirv} cross-vendor open standards.
  \item We evaluated \chipstar performance in comparison to running the CUDA applications directly using the NVIDIA SDK or converting the applications to a popular open-standard-based CUDA alternative SYCL.
  %~\cite{SYCL}
  \item We demonstrate a case study for the usability of OpenCL as a portability layer to implement other languages/APIs on top. Portability is shown by providing performance numbers on a RISC-V CPU \& PowerVR GPU,  ARM CPU and GPU, as well as on discrete GPUs from NVIDIA, AMD, and Intel. 
\end{enumerate}

The rest of the article is structured as follows: Section~\ref{section:portabilityAPIs} discusses our rationale for choosing OpenCL~\cite{OpenCL} and its device-side program representation SPIR-V~\cite{SPIRV} as the core APIs to support runtime portability in \chipstar.
Section~\ref{section:implementation} details the key technical issues in implementing the HIP/CUDA runtime on these APIs, while Section~\ref{section:compilation} focuses on the compilation aspects. Performance evaluation results are shown in Section~\ref{section:performance}. 
Section~\ref{section:applications} presents the GAMESS-GPU-HF porting case study using \chipstar, and finally Section~\ref{section:conclusions} concludes the article.


%%%%%%%%%%%%%%%%%%%%%%%%%%%%%%%%%%%%%%%%%%%%%%%%%%%%%%%%%%%%%%%%%%%%%%%%%
\section{\texorpdfstring{\myhl{Rationale for the Chosen Standard APIs}}a}
\label{section:portabilityAPIs}

%As a technical background, we provide our considerations for the ``hardware abstraction layer'' API options forming the platform portability layer for \chipstar.
\myhl{As a technical background, we review the various options available for the portable runtime API and the target-independent device program representation in }\chipstar \myhl{and provide a rationale for our choice}. The discussion is split into 1) runtime APIs used to control the execution from the host side and 2) device program representations providing an abstraction for the kernel side programs in a portable manner. Due to the abundance of potential target devices available for acceleration, we consider it important to be able to embed the device programs in an open standard-based \gls{ir} and to use \gls{jit} compilation for lowering the program to the target \gls{isa} of the accelerator at deployment or launch time. This enables future-proof ``fat binaries'' which can be supported on new platforms by implementing the specification of the \gls{ir}. Another alternative would be to provide only source-level
compatibility where the application needs to be recompiled for each host and a device pair of interest, hindering binary distribution.

\subsection{Runtime APIs}

In practice, both CUDA and HIP are single-vendor supported programming models.
This is reflected, for example, in their platform property APIs which define limited
queries for device properties, highlighting features in each vendor's
GPU offerings. The goal of \chipstar is to
expand the portability of applications implemented using the CUDA/HIP APIs.
Therefore, the key requirement to the underlying ``platform/device portability API'' is to
cover as many of the essential features of CUDA/HIP as possible to
provide functional correctness and to exploit the potential performance benefits.
This includes,
for example, parallel and asynchronous execution of tasks, overlapping of
data transfers with task execution, and by providing access to
shared memory communication, if available. Furthermore, the portability API
should provide services to enhance performance portability of
the implementation by allowing to query the capabilities of the
devices to tune the execution at runtime to match the target's features.

There are not a large number of choices for such a runtime API, especially
if limiting the list to alternatives that enjoy official driver support from
multiple accelerator vendors or to those that have a portable long-maintained open source implementation. In this regard, an open standard based API that has increased in popularity is SYCL~\cite{SYCL}. SYCL
resembles CUDA in being a C++-based single-source API. However, as a layer for supporting other languages on top, it lacks means to explicitly load and launch pre-built kernels defined in non-SYCL language at host runtime. It has an interoperability layer that allows one to build OpenCL C kernels, but then the additional value of using SYCL as a portability layer instead of using OpenCL directly is not obvious.

Recently, OpenMP~\cite{OpenMP} has been considered for portability layer usage and,
in fact, specifically for implementing CUDA in~\cite{10.1145/3559009.3569687}.
While OpenMP enjoys support from a wide range of vendors,
we believe OpenMP is not ideal for this use case since it doesn't
define a device-side program representation, making future-proof cross-vendor portable fat binary generation difficult. It also can be considered
a high-level ``application-programmer-facing'' API similarly to SYCL,
thus offers constructs and overheads for programmer-productivity which are unneccessary for a portability layer.

\gls{hsa} is an open heterogeneous platform
specification that also defines a runtime API~\cite{HSA,HSART}.
A key differentiating feature of \gls{hsa} is that it standardizes on shared virtual memory,
making system-wide virtual memory addressing a required feature from
implementations. In hindsight this requirement was too much
too early, as system-wide virtual memory support is only recently appearing
in hardware and still usually requires explicit allocation or mapping
calls from the programmer.
For this work, \gls{hsa} could have been a valid choice for
a lightweight portability layer, but activity on the specification has ceased,
with mainly AMD using only selected parts of it in their software stack (the ROCr runtime component).

Another option to consider would be Vulkan~\cite{Vulkan} since it also
provides a compute pipeline stage which allows specifying general purpose compute
kernels. However, the
feature set of the compute kernels lacks some of OpenCL's features such as \gls{svm} which
would require further standard extensions.
It might be that in the future
the feature gap gets narrower and it would become a viable option. Meanwhile, layering OpenCL on top of Vulkan is an interesting option pursued by multiple open source projects to cover the devices which previously only had Vulkan driver support.

Level Zero~\cite{l0} is an API at a similar level of abstraction as \gls{hsa} and OpenCL.
However, it's currently only supported on Intel's devices, thus is not suitable for future-proof cross-vendor fat binaries.

Interestingly, OpenCL was originally created to provide an open cross-vendor alternative to the proprietary CUDA GPGPU (General Purpose GPU) programming model. \myhl{After its introduction in 2009} it has received wide support from hardware vendors, but some application developers are known to dislike it because it can be considered too low level as an application-facing API and unproductive with the driver and feature support lagging behind the proprietary alternatives.
However, as demonstrated in this paper, OpenCL can provide an excellent portability layer for implementing other higher-level programming models and APIs on top. This portability is due to multiple vendor's official support for the minimum feature set of OpenCL version 3.0 and to multiple long-maintained open source implementations. For these reasons, we have chosen OpenCL as the runtime through which to access the GPU.

\subsection{Device Program Representations}
\label{subsec:deviceProgramRepresentations}

Heterogeneous platforms suffer from the problem of device-side program description portability. There is a wide range of instruction-set architectures the kernels can target, and when the program is distributed in a binary form, the targets are known only at run time.
Thus, the choice of the format in which the device programs (kernels) are stored is critical as it should cover as many of the potential targets as possible.
Furthermore, the representation should be ``future-proof'' in a sense that the produced fat binaries could be made to run on entirely new platforms by only referring to the API specification.
At the time of this writing, there still seems to be no clear winning program representation in this regard. Various portable implementations of application-facing APIs resort to very fat binaries which store copies of the device program in multiple (virtual) instruction-set architectures to cover the various targets and offloading runtimes it might encounter at execution time. This is the case with~\cite{10.1145/3559009.3569687} and originally in AdaptiveCpp~\cite{10.1145/3529538.3530005}.

Recently AdaptiveCpp started storing kernels in the LLVM~\cite{LLVM} compiler \gls{ir} instead of storing multiple different binaries depending on the target. In this scheme, LLVM \gls{ir} is lowered to various target-dependent formats at runtime at the point when the target is known~\cite{OpenSYCLfatbin}.
This approach has benefits in comparison to storing abundance of device binaries in the another alternative, and works in theory, but it is also known that LLVM \gls{ir} is not supposed to be a portable program representation as it can embed target-specific intrinsics, has target specific data layout and endianness among other challenges.
LLVM \gls{ir} is not guaranteed to be stable across LLVM versions, which means that the fat binaries should have access to an LLVM library of version the \gls{ir} was generated with, which at worst requires to embed the LLVM library along and the required backends to the fat binary, forming an unnecessary dependency.
The problem of LLVM \gls{ir} not being target-independent nor stable across LLVM versions was attempted to be addressed by earlier \gls{spir} versions 1.2 and 2.0~\cite{SPIR2}: These first \gls{spir} versions were designed to support OpenCL C language kernels and were based on defined versions of LLVM \gls{ir}, which proved to be difficult to maintain long term.
LLVM-based \gls{spir} versions were later obsoleted in favor of the \gls{spirv}~\cite{SPIRV} format.
The goal for \gls{spirv} is to provide a robust cross-vendor specified intermediate language which is not affected by LLVM upstream changes and that shares specification effort with the Vulkan community.

\gls{hsa} specification defines an intermediate language called \gls{hsail} and a binary representation called BRIG~\cite{HSAIL}. A key technical difference between \gls{hsail} and \gls{spirv} format, is that \gls{hsail} has 
a fixed number of registers and an address space for spills unlike \gls{spirv}, which has infinite virtual registers due to being based on the \gls{ssa}~\cite{SSA} representation.
\gls{hsa} made a choice to not define a higher-level programming language (like OpenCL C) for the device programs, but only standardized a low level \gls{ir}.
As with the \gls{hsa} runtime specification, however, the activity on the \gls{hsail} specification has stalled.
There was also a GCC-based frontend for consuming BRIGs in a target-portable fashion, but after activity on \gls{hsa} quieted, the ``BRIG frontend'' was removed from the upstream GCC source code repository in a May 2021 commit.

In conclusion, while official \gls{spirv} OpenCL environment support from processor vendors is extensive as of this writing, it seems to be still the best option for a cross-platform representation given that it is an open standard defined democratically by multiple hardware vendors and is relied upon by OpenCL, Vulkan, and SYCL implementations among other use cases such as Direct-X adopting it.
In addition, thanks to open source tooling support available and useful \gls{spirv} producers such as \chipstar and Intel oneAPI \gls{dpc} appearing, the list of supported targets is expected to grow in the future. Therefore, we considered \gls{spirv} and OpenCL to be future-proof open standards on which to base the implementation.

\section{Implementing HIP/CUDA on OpenCL Runtime API}
\label{section:implementation}

The primary goal for \chipstar is to support the subset of CUDA features
as defined by HIP and expand the feature set beyond it whenever feasible
while relying on the chosen open standard APIs as much as possible.
In this section, we discuss how the OpenCL/\gls{spirv} specifications can be matched with
the commonly used features of CUDA/HIP and identify the most impactful gaps
that we believe should be covered in the future.

\subsection{Memory Model}

Due to their common history in GPGPU programming, CUDA/HIP and OpenCL share various
platform and memory model abstractions. For example, ``device memory'' is the same as
``global memory'' in OpenCL terminology (``shared'' is ``local'').
To avoid confusion in terminology we use only the CUDA/HIP terms in the rest of
this article. Similarly, we refer to the original CUDA versions when talking about
functions that have their counterparts in the HIP API.

A key difference between OpenCL and CUDA that required addressing is the
fact that CUDA implicitly infers the address space of the data in the device
program side whereas in OpenCL (before v2.0) the address space must be declared explicitly.
The CUDA's implicit address space inference is similar to the 'generic'
address space concept introduced in OpenCL v2.0, which was utilized to bridge
this gap.

The simplest interface in CUDA's host-side device memory management is \func{cudaMalloc()}.
\myhl{It returns a raw pointer to the targeted device's global memory, instead of an opaque buffer handle.}
% as is the case with OpenCL's basic buffer management functionality. This presents a small
%but significant difference from the OpenCL v1.x specification for device memory management;
%OpenCL v1.x only provides a buffer management API (\func{clCreateBuffer()} and others) which returns opaque \type{cl\_mem} handles.
%
%The opaque buffer handles cannot be used to implement CUDA device memory allocation
%because they do not provide access to the underlying raw device address or passing addresses in other data
%structures, which is allowed with the CUDA device pointers. 
\myhl{In order to implement these
capabilities, we utilized the }\gls{svm} \myhl{API that first appeared in the OpenCL v2.0.}
%The raw pointers vs buffers difference along with the
%implicit address space inference required us to lift the minimum OpenCL
%version to v2.0 to support even the most basic CUDA programs.

The \gls{svm} allocation API returns a raw pointer to a shared
virtual address space region. The ``\gls{cg} \gls{svm}'' variant can be used to
implement the basic device memory allocation. Mapping device memory allocation to \gls{cg} \gls{svm}
has a drawback that the device driver must support some of the unneeded \gls{svm} features such as
mapping the allocated regions to the virtual address space although just returning physical
device memory pointers would suffice. This means that the \chipstar implementation is actually
implementing CUDA's Unified Memory model by default. To alleviate the potential performance
impact of this, \chipstar can also use the Intel \gls{usm}
extension (\textit{cl\_intel\_unified\_shared\_memory}~\cite{intel-usm}), if supported by the runtime. \gls{usm} enables allocating strictly
device-only allocations, but still returns virtual pointers, which can be problematic for some implementations. 

\myhl{In order to provide a minimal allocation API} matching the basic \func{cudaMalloc()}'s needs \myhl{without requiring a }\gls{svm}\myhl{-capable OpenCL driver}, we introduced a new extension (\textit{cl\_ext\_buffer\_device\_address}\cite{intel-buffer}) that enables querying the raw device pointer of a cl\_mem allocation without needing to map the buffer to the same address range in the host's virtual memory. \chipstar can use any of these alternative memory management APIs, if advertised by the targeted OpenCL device.

CUDA provides an API to \textit{pin} memory so it's kept resident in the host memory and
optionally made accessible by devices from kernel code and is not swapped out to disk. 
The primary APIs to this functionality are
\func{cudaHostAlloc()} and \func{cudaHostRegister()}. The former allocates pinned
memory directly and the latter pins a previous host allocation. \func{cudaHostAlloc()}
is simple to implement with coarse grained \gls{svm} since by the coincidence of using
a shared virtual memory allocation, the buffers are by default accessible in both the host and
the device using the same pointer. However, the allocation might not be resident for the
duration of the execution, for example, if a CPU device is allowed to swap out such
allocations. However, that aspect can only be observed by the programmer as a performance difference.

\func{cudaHostRegister()} is a bit more challenging to implement on top of \gls{cg} \gls{svm} since it
allows registering a host address range to be a pinned region accessible both from the host and
the device \textit{after}
the host memory has been allocated. Since the allocation might not have
been originally allocated with the OpenCL SVM allocation API, but with a system memory allocator or even
from the stack, to implement correct functionality in this case, \chipstar creates
a shadow buffer using \func{clSVMAlloc()} and synchronizes it with the host region at
kernel start and end points. OpenCL 2.1 added a new \func{clEnqueueSVMMigrateMem()} API that enables fine grained specification of where regions of \gls{svm} are migrated, but it is not useful for this case since the source of \func{cudaHostRegister()} can be any host memory area whereas the API handles only \gls{svm} allocations.

The NVIDIA architectures post-compute capability 6 support on-demand page migration which
relies on the hardware memory management unit (page fault-based buffer migrations) for coherence
of the Unified Memory allocations. This frees the programmer from the need to perform explicit memory allocation and synchronization calls. This functionality maps to the Fine-Grained System \gls{svm} of OpenCL, but since hardware and driver support for fine-grain \gls{svm} is very rare at the time of this writing, on-demand page migration is not yet implemented by \chipstar.

\subsection{Tasks and Events}

The semantics of CUDA \textit{streams} and the ability to execute tasks/commands
asynchronously maps well to the \textit{command queues} of OpenCL. Each stream is expected
to execute commands in-order, which matches the in-order command queue semantics of OpenCL.
Commands are allowed to execute concurrently even within in-order command queues in OpenCL,
as long as the results are not observable from the outside, enabling concurrent kernel
execution~\cite{OpenCL}.

\subsection{Textures}

\chipstar supports only a subset of texture objects due to a limitation in OpenCL images. The notable differences between HIP/CUDA and OpenCL are that the texture objects are pointers to opaque C/C++ structures whereas in OpenCL/\gls{spirv} there is a special type per image dimensionality and that the texture objects can be loaded indirectly whereas OpenCL images can be only passed to kernels as kernel arguments. Therefore, some constructs such as the following cannot be expressed in \gls{spirv}:

\begin{verbatim}
  hipTextureObject_t Tx = ...;
  Ty Tv = cond ? tex2D<Ty>(Tx, X, Y) 
               : tex1D<Ty>(Tx, X)
\end{verbatim}

An LLVM pass is responsible for lowering texture object API based texture functions to OpenCL image fetches. The pass analyses endpoints of the texture objects by following their use-def and def-use chains. If the pass sees that a texture object is coming from a kernel parameter and it is only used by texture fetch calls for the same dimensionality, it will replace the texture object parameter with image and sampler parameters and translates the texture fetch calls with OpenCL image fetch calls of matching dimensionality which consume the new kernel parameter. 

%%%%%%%%%%%%%%%%%%%%%%%%%%%%%%%%%%%%%%%%%%%%%%%%%%%%%%%%%%%%%%%%%%%%%%%%%

\section{Compilation Aspects}
\label{section:compilation}

This section discusses the compilation flow used by \chipstar. We introduce the overall compilation flow, the device library implementation and summarize our findings on the key needs to extend the OpenCL/\gls{spirv} standards to bridge key feature gaps between the specifications.

\subsection{The Compilation Flow}

The offline compilation flow of \chipstar is built on the LLVM Project's~\cite{LLVM} Clang~\cite{Clang} frontend which provides the frontend language handling and splitting of the single source input to the device and host parts. The overall compilation process is shown in Fig.~\ref{fig:compilation}. It relies on the CUDA/HIP frontend of Clang, which was extended to produce \gls{spirv} binaries as an option to PTX or AMDIL for the device program. The LLVM \func{opt} tool is used to invoke special LLVM passes provided by \chipstar for lowering HIP features to the OpenCL-\gls{spirv} environment. The \gls{spirv} translation is performed using Khronos' LLVM-SPIRV-Translator tool~\cite{llvm-spirv}.

\begin{figure*}
    \centering
    \includegraphics[scale=1]{chipstar-compilation-v2.pdf}
    \caption{The construction of a HIP fat binary. Device code is compiled into LLVM \gls{ir}, chipStar applies a series of LLVM passes at optimization time. The resulting \gls{ir} is embedded into the binary and linked against the device library. }
    \label{fig:compilation}
\end{figure*}

Most of the compilation-related changes have been upstreamed to the LLVM project and very little compilation-related functionality remains within the \chipstar code base. The notable exceptions are compiler passes that handle CUDA vs. OpenCL differences in \func{printf()}, implement a device side \func{abort()} feature, handling of CUDA's device-side global variables, and an indirect memory access analyzer. The indirect memory access analyzer marks kernels that are known to not indirectly access allocations, which removes  unnecessary synchronizations for the majority of
benchmarks seen so far. Otherwise, due to CUDA's memory model, each launcher kernel can potentially access any previously allocated buffer, inducing significant unnecessary
data synchronization overheads in the common case where the kernels only access buffers set through their arguments.

\subsection{Lazy Just-in-Time Compilation}

Fig.~\ref{fig:online-compilation} shows the online compilation flow from \gls{spirv} to device code in the \chipstar runtime.
When a kernel launch is requested, the kernel function stub pointer is used to look up the associated \gls{spirv} module which, in turn, is \gls{jit} compiled to machine code.
To enhance runtime portability, the built-in library of the on-line device provides variations in built-in HIP functions for different device capabilities that are linked to the user’s device programs at runtime. For example, for HIP floating-point atomics the runtime chooses between an implementation that maps them to corresponding native functions via a \gls{spirv} extension or emulates them via atomic exchange operations.

\begin{figure*}
    \centering
    \includegraphics[scale=0.9]{chipstar-rt-compile-n-link.pdf}
    \caption{The just-in-time compilation flow. Once the program is executed, prior to calling main, a series of internal HIP calls are executed. These calls parse the fat binary and register the embedded \gls{spirv} files as modules. In lazy \gls{jit} mode, these modules are not compiled into kernels until their launch is requested.}
    \label{fig:online-compilation}        
\end{figure*}

%Initially, the kernel compilation was implemented in an eager manner in \chipstar: all of the kernels were compiled prior to entering main().  If the SPIR-V has a lot of kernels bundled in, but only a small subset of them are invoked by each application run, the compilation time could grow very high. For example, a neural network framework included all neural network operators in a single SPIR-V library, requiring its compilation at each network launch although only a subset of the operators were actually utilized.

\chipstar implements a lazy \gls{jit} compilation strategy \myhl{to avoid compiling all kernels of large }\gls{spirv}\myhl{ libraries such as those in neural network libraries, unless called from the host program.} 
%With lazy \gls{jit} enabled
%modules are not compiled immediately upon creation, but rather compilation is deferred until the first kernel call that requires the module. 
When compilation occurs, the \gls{spirv} binary is converted to the target format, build flags and options are set up, and backend-specific compilation methods (such as \func{clCreateProgramWithIL} for OpenCL) are used. 
Once compiled, modules are cached for future use. This approach significantly reduces startup time for applications with many kernels, as only the necessary kernels are compiled. 

The implementation includes performance monitoring features, tracking compilation time and logging whether modules are loaded from cache or compiled fresh.
The lazy \gls{jit} system is designed to be backend-agnostic, allowing different backends (Level0, OpenCL, etc.) to implement their own specific compilation strategies while maintaining consistent lazy compilation behavior. This flexibility enhances \chipstar's portability across various platforms and backends.

\subsection{Device Library}
The \textit{chipstar} device-side library implements the HIP math API, by using a combination of OpenCL C math built-ins, LLVM built-ins, OCML (part of ROCm-Device-Libs), and custom implementations.
Many of the functions in the HIP math API have an equivalent OpenCL built-in with adequate accuracy guarantees. However there are a few exceptions that cannot be mapped directly, and thus require software-based emulation such as floating-point atomics on some devices. The main challenge in terms of a fast yet portable implementation of the functions is due to differences in math accuracy requirements between CUDA/HIP and OpenCL. While CUDA provides very specific ULP accuracy requirements \cite{cuda-ulp} (most of which are higher than those in the OpenCL specification \cite{opencl-ulp}) for their math library, no such requirements were specified for HIP 6.1 which chipStar implements. These ULP requirements are now specified as of HIP 6.3 \cite{hip-ulp} but still seem to diverge from those required by CUDA.  

Furthermore, CUDA/HIP defines a set of \textit{intrinsics}, which are faster yet less accurate versions of the standard functions. 
This exposes a further difficulty when aiming for a portable, yet fast implementation: the level of accuracy achievable depends heavily on the targeted platform. Since CUDA is inherently meant not to be cross-vendor portable, the intrinsics are defined only to match the CUDA microarchitecture in an optimal manner, which might not be the case for other devices. 

OpenCL covers the use case of accessing fast but less accurate hardware operations by means of 1) a relaxed mathematics flag that can be enabled at device program build time and 2) with so-called native built-in functions in the built-in kernel API. Unfortunately, neither of these are usable for implementing the CUDA intrinsics by default due to not guaranteeing enough ULP accuracy as required by CUDA API. The relaxed math in OpenCL defines maximum rounding errors, but they are usually slightly less than what the CUDA intrinsics require Fig. \ref{ulp-math}. The OpenCL native built-in functions are an even worse fit for this use since they guarantee nothing of the accuracy but leave it entirely up to the implementation Fig.\ref{ulp-fast-math}. There is not even a possibility to query for the maximum error via a runtime API, the accuracy must be discovered via trial-and-error or from documentation of the hardware vendor. 

\begin{table*}
\centering
\begin{tabular}{lcccc}
\hline
\textbf{Function} & \textbf{Intel Arc A770} & \textbf{Intel UHD 770} & \textbf{AMD gfx906} & \textbf{Intel i9-13900K} \\
\hline
cos               & 1                       & 1                      & 1                   & 2                         \\
exp               & 2                       & 2                      & 1                   & 1                         \\
log               & 1                       & 1                      & 2                   & 1                         \\
\hline
\end{tabular}
\caption{Maximum ULP Differences for Standard OpenCL Math Functions. OpenCL specifies ULP requirements of 4, 3, and 3 for cos, exp, and log, respectively.}
\label{ulp-math}

\end{table*}

\begin{table*}
\centering

\begin{tabular}{lcccc}
\hline
\textbf{Function} & \textbf{Intel Arc A770} & \textbf{Intel UHD 770} & \textbf{AMD gfx906} & \textbf{Intel i9-13900K} \\
\hline
native\_cos& 27.54                   & 27.54                  & 6.07                & 1084.87                   \\
native\_exp& 0.84                    & 0.84                   & 0.67                & 474.68                    \\
native\_log& 0.36                    & 0.36                   & 0.46                & 154.81                    \\
\hline
\end{tabular}
\caption{Average ULP Differences for Native OpenCL Math Functions. OpenCL does not have ULP requirements for native functions.}
\label{ulp-fast-math}

\end{table*}

Previous research \cite{ulp-comparisons-paper} shows significant disparities in the precision of GPU math functions, with CUDA consistently achieving the highest accuracy, followed by HIP, LLVM, and OpenCL (estimated based on relaxed math). CUDA’s functions maintain ULP errors close to 1, while HIP and LLVM exhibit progressively higher errors, sometimes exceeding 3-5 ULP. OpenCL’s relaxed math and native built-in functions lack strict accuracy guarantees, making it challenging to replicate CUDA’s precision without significant adjustments or corrections.

The ``correctness first'' principle mandates the use of arithmetics that have  guaranteed accuracy, which means to not receive any performance benefits of simplified implementations. This approach is not taken by \gls{dpc}, for example, as it defaults to a more relaxed floating point precision model (-fp-model=fast).
Achieving ULP requirements specified in the CUDA API would require emulating them in software . Performance impact would likely be too drastic to make \chipstar usable for high performance workloads.
Since we cannot make any guarantees about precision, we have chosen to map regular and intrinsic functions to the same OpenCL default accuracy function implementations.

We plan to optimize this aspect in the future via a new standard extension with a set of built-ins that guarantee the CUDA accuracy requirements to the application programmer while enabling the targeted platform to optimize and implement them as efficiently as possible.

\subsection{OpenCL Extensions}

The \chipstar compilation flow is built such that different advanced OpenCL features and extensions are not required from the target platform's driver or device unless the compiled input application specifically needs them. Although the minimal OpenCL 3.0 feature set plus coarse-grained \gls{svm} and \gls{spirv} consumption support covers a significant part of the most commonly used CUDA and HIP features, some functionalities require or can be improved with extensions to the OpenCL or \gls{spirv} specifications.

\begin{table*}[ht]
    \centering
    \begin{tabular}{|p{5 cm}|p{5cm}|}
    \hline
\textbf{Extension} & \textbf{CUDA/HIP feature(s)} \\
    \hline
cl\_intel\_unified\_shared\_memory & Used for optimized \func{cudaMalloc()} when available. \\
   \hline
cl\_ext\_buffer\_device\_address & Used for optimized \func{cudaMalloc()} when \gls{usm} nor \gls{svm} are available. \\
   \hline
cl\_intel\_required\_subgroup\_size & Used to fix the warp-size. \\
    \hline
cl\_khr\_fp64                       & If double precision floating point is used. \\
    \hline
cl\_khr\_subgroups                  & Warp-level synchronization with \func{\_\_syncwarp()}.\\
    \hline
cl\_khr\_subgroup\_ballot           & Warp-level ballot operations. \\
    \hline
cl\_khr\_subgroup\_shuffle          & Warp-level shuffle operations. \\
    \hline
    \end{tabular}
    \caption{OpenCL 3.0 standard extensions that \chipstar can use currently to implement CUDA/HIP features if the compiled application uses them.}
    \label{table:extensions}
\end{table*}

\begin{table*}[ht]
    \centering

    \begin{tabular}{|p{5 cm}|p{5cm}|p{5cm}|}
    \hline
\textbf{Extension (working title)} & \textbf{CUDA/HIP feature(s)} & \textbf{Status} \\
    \hline
cl\_ext\_alive\_only\_barrier       & A special work-group barrier for barrier calls which might not be reached by work-items that have exited the kernel as allowed by the CUDA's execution model. & Draft. \\
    \hline
cl\_ext\_cuda\_math     & Implement math functions and intrinsics with precision requirements that match CUDA's. To enable more optimized reduced precision intrinsics. & To be proposed.  \\
    \hline
cl\_ext\_device\_side\_abort        & Implement \func{\_\_trap()} on the low-level runtime side. The current implementation requires compiler transformations. & Public draft.  \\
    \hline
cl\_ext\_extended\_device\_properties & \func{hipGetDeviceProperties()} can be used to query more device properties than the basic OpenCL device or platform query APIs support, this fills the gap. & To be proposed. \\
    \hline
cl\_ext\_relaxed\_printf\_address\_space &  CUDA's \func{printf()}behavior with non-constant address spaces. Currently handled with compiler transformations. & Public draft. \\
   \hline   
cl\_khr\_command\_buffer            & For optimized implementation of CUDA graph re-execution. & Public. \\
    \hline
cl\_ext\_command\_buffer\_host\_data & For optimized implementation of CUDA graphs which transfer data between the host and the device. & Draft. \\
    \hline
cl\_ext\_command\_buffer\_host\_sync & For optimized implementation of CUDA graphs which synchronize with the host. & Public draft. \\
    \hline
cl\_ext\_subgroup\_id\_mapping & For forcing the desired thread id mapping when calling warp-level primitives that depend on the fixed warp size or the thread id ordering. but Luckily, in practical targets mapping is already the desired one by default. & Draft. \\
    \hline

    \end{tabular}
    \caption{Planned or drafted OpenCL 3.0 standard extensions that \chipstar might use in the future to implement CUDA/HIP features if the application uses them. The status column describes the state of the extension at the time of this article's publication. }
    \label{table:proposedExtensions}
\end{table*}




In Table~\ref{table:extensions} we summarize the standard extensions \chipstar can already utilize and which CUDA/HIP feature triggers their need. Table~\ref{table:proposedExtensions} describes further work-in-progress extension proposals we have identified to be useful for CUDA/HIP portability. These extensions are in different stages in the Khronos Group standardization process, which is noted in the table.\footnote{Note to reviewers: We will update the status for the final article version.}

Most of the extensions are relatively straightforward and the brief description in the table should suffice to grasp their purpose. However, the handling of warp-level primitives calls for a bit more thorough explanation:
One of the execution model differences between CUDA and OpenCL is that CUDA presents a finer grained fixed size grouping of the threads (OpenCL work-items) than the blocks (work-groups) called a \textit{warp}. In earlier CUDA versions, the threads in a warp could be assumed to execute in lock-step, implying that the enabled threads in the same warp would execute the same instruction. This implied that in some cases explicit synchronization could be omitted: In case of a usual read-modify-update case, the programmer could trust that the warp's threads all execute the read part before any of them proceeds to the update part, enabling in-place-updates without explicit synchronization. In later versions of the CUDA specification, the use of lock-step behavior in program logic was deprecated, but the feature has to be supported for legacy applications~\cite{cuda-lockstep}.

In addition to older CUDA programs potentially relying on the lock-step semantics to omit explicit synchronization, the fixed size warps (32 threads for NVIDIA and usually 64 threads in AMD devices) affect the execution semantics when executing warp-level functions that rely on the warp grouping and the mapping of the threads to the lanes of the warp.  Such primitives include the warp shuffles, which read data from a specific lane within the warp, and the explicit warp synchronization primitives.
The OpenCL specification, on the other hand, doesn't have a warp concept, but the work-items are free to make progress in any order and grouping. The specification, however, has a feature extension called ``subgroups'' that is used to implement the warp semantics in \chipstar when the kernel is detected to need it. However, in contrast to warps which have a specified form and content which allows the programmer to utilize them reliably, the basic subgroups of OpenCL are ``implementation-oriented''; they enable grouped execution in a manner that is simplest or most efficient for the driver and the hardware at hand. The sizes of the OpenCL subgroups are not fixed, but must be queried per kernel by the programmer in the basic extension. Also the way work-items are mapped to subgroup lanes so they can be referred to when using cross-lane intrinsics is also implementation-defined. To close the gap between subgroups and warps, an enhanced extension that \textit{forces} the subgroup size of the kernel to the desired size along with the linear id mapping is being proposed.

\subsection{Unsupported HIP/CUDA APIs}
\myhl{The following APIs have not yet been implemented as feature implementation timelines are driven by application requirements:Cooperative Groups,  Memory Pools, Inter-process Communication, Peer-to-Peer Access, and Occupancy APIs. Implementing some of these APIs might require additional OpenCL extensions. This is left for future work.}

%%%%%%%%%%%%%%%%%%%%%%%%%%%%%%%%%%%%%%%%%%%%%%%%%%%%%%%%%%%%%%%%%%%%%%%%%

\section{Evaluation}
\label{section:performance}

We evaluated  \chipstar  performance on various OpenCL-capable CPU and GPU platforms using a subset of the HeCbench benchmark collection~\cite{HeCbench}. All of
the results were produced using the \chipstar v1.2.1  release.

The HeCbench benchmark application selection criteria for each comparison was as follows:
\begin{enumerate}
    \item The application had the necessary API/language variations with a HIP version that could be built with \chipstar v1.2.1 and its ported libraries.
    \item For SYCL/CUDA comparisons, there must not have been significant identified performance-affecting structural or implementation differences between the SYCL/CUDA and HIP versions of the application. Some of the identified ``unfair differences'' were fixed and submitted to the HeCbench repository.  The exact branch used for benchmarking can be found here: https://github.com/CHIP-SPV/HeCBench/tree/chipStar-bench

    \item The application could verify its results and had to validate correctly on all platforms involved in the comparison.
    \item Applications that required hardware or OpenCL driver features that were missing or too limited on the platform were omitted. This mostly concerned the runs on embedded/integrated GPUs with limited memory or lack of double precision floating point support.
\end{enumerate}

\subsection{OneAPI \gls{dpc} on Intel oneAPI SDK}
\label{sec:SYCL-comparison}

For this evaluation we chose a subset of benchmarks included in the HeCbench suite and compared the performance of their HIP versions of the benchmarks against the SYCL versions. 
The SYCL versions were compiled with Intel's \gls{dpc} shipped with the oneAPI v2024.2.2 release. 

The following consumer-grade hardware from Intel, \myhl{Nvidia}, and AMD was used to perform the  benchmarking: Intel A770 discrete GPU, Intel UHD 770 integrated GPU, Nvidia RTX3060, AMD Vega VII and Intel i9-13900k CPU. 

Since both \chipstar and \gls{dpc} can use OpenCL as a backend, the evaluations are performed using the same OpenCL driver on the same GPUs. This  isolates the differences between the tested software stacks to the runtime and the LLVM \gls{ir} level device code compiler optimizations. 

\begin{figure*}
    \centering
    \includegraphics[width=1\linewidth]{chipStar-vs-sycl-speedup-truncated.pdf}
    \caption{\myhl{chipStar speedup over SYCL via} \gls{dpc} \myhl{with relaxed math, excluding 20 benchmarks where the performance difference was negligible }}
    \label{fig:best-vs-best}
\end{figure*}
In Fig. \ref{fig:best-vs-best}, 
we compare the expected fastest results between SYCL/\gls{dpc} and chipStar when floating point relaxations are enabled to the extent the implementations can utilize them. As expected, there are only minor differences in the performance since
 both \gls{dpc} and \chipstar implement their respective runtime by leveraging  OpenCL as the GPU API
and have a similar compilation flow: kernels are compiled into LLVM \gls{ir} which is then translated to \gls{spirv} which, in turn, is then \gls{jit} compiled to device code at runtime.  

For the vast majority of benchmarks, the performance difference between \gls{dpc} and \chipstar was negligible, but there are outliers to both directions. \gls{dpc} can currently benefit from FP relaxations more as it defaults to -fp-model=fast \cite{intel-fp-fast} which enables optimizations like floating-point reassociation, fused multiply-add operations, and the omission of certain intermediate rounding steps, as well as the use of less accurate device-side math functions. The HIP/CUDA equivalent is --use\_fast\_math \cite{cuda-fast-math} but \chipstar v.1.2.1 does not yet support this compiler flag.

We investigated the outliers and identified several reasons for the performance achieved:

\begin{itemize} {
    \item \textbf{Device library differences.} }There are significant differences in the device libraries: \gls{dpc} uses the Intel Math Functions (IMF) Device Library \cite{imf-website} whereas chipStar relies on a combination of LLVM built-ins, native OpenCL operations, a few custom implementations, and finally ROCm's OCML bitcode  implementations to provide complete coverage. Furthermore, \gls{dpc} defaulting to -fp-model=fast enables the use of \_native intrinsics which are significantly faster (and more accurate) than non-native implementations.
    \item \textbf{Effect of \gls{jit} Flags. } chipStar sees significantly higher performance increases when \gls{jit} flag  -cl-fast-relaxed-math is passed to the OpenCL runtime. 
    \item \textbf{Floating point relaxations}. \gls{dpc} defaults to -fp-model=fast which results in more aggressive optimizations by removing IEEE 754 guarantees and assuming no NaN/Inf values, generation of FMAs. In comparison, HIP allows only FMA contractions by default.

    \item \textbf{Value assumptions.} \gls{ir} produced by \gls{dpc} takes advantage of AssumeTrueKHR which is a \gls{spirv} instruction that tells the compiler to assume a given condition is always true, allowing it to optimize code more aggressively under that assumption. Using this instruction should give it a performance advantage \myhl{over} chipStar. 
    \item \textbf{Intel-specific decorators.} \gls{dpc} tends to produce \gls{ir} which often contains Intel-specific operations that are meant to help with optimization when targeting Intel platforms which support the extensions, such as OpAliasDomainDeclINTEL, OpAliasScopeDeclINTEL whereas chipStar \gls{ir} is more generic.

\end{itemize}

One thing to note is that the following results include only the benchmarks which passed correctness checks and in this regard chipStar ended up successfully running 6 \myhl{fewer benchmarks} compared to SYCL. A lot of these benchmarks have quite tight epsilon bounds so it's likely that with small adjustments the number could be reduced.

In order to analyze the benefits of IMF, we modified the chipStar device library to link against IMF for exp and sqrt calls and tested the performance again which resulted \myhl{in the} adam benchmark going from 0.16x to 0.50x.

After we applied the -cl-fast-relaxed math \gls{jit} flag, all of these benchmarks performed equally well in \chipstar and \gls{dpc}. In conclusion, all the cases where chipStar is slower than \gls{dpc} can be explained by the differences in math device library and the more aggressive default optimization of the \gls{ir} in terms of floating point calculations due to \gls{dpc} defaulting to -fp-model=fast.

\myhl{We selected} 9 benchmarks where chipStar-v1.2.1 outperforms \gls{dpc} \myhl{for closer examination}. Of these, most significant difference is seen in gaussian. Upon \myhl{further examination}, the time spent inside kernels is identical between these two runtimes so the most likely reason for the difference would be \gls{dpc} overhead. This overhead cost could be comprised of either \gls{dpc} startup costs, \gls{dpc} kernel invocation costs or a combination of both.
\textbf{}
To test for this, we isolated the outperforming benchmarks and performed a \myhl{scaling} experiment by comparing performance on the original problem size to the performance achieved with the number of iterations \myhl{or problem size increased by 5x} (Fig.~\ref{fig:chipstar-outperform-sycl-1x}). \myhl{This scaling leaves the runtime of a single kernel unchanged, which} isolates the \gls{dpc} startup overheads.

\begin{figure*}
    \centering
    \includegraphics[width=1\linewidth]{chipStar-vs-sycl-scaled.pdf}
    \caption{\myhl{chipStar speedup over SYCL via} \gls{dpc} \myhl{at 1x and 5x problem sizes}.}
    \label{fig:chipstar-outperform-sycl-1x}
\end{figure*}

Scaling the number of iterations reduced the geomean chipStar-v1.2.1 speedup from 1.31x to 1.17x indicating that \gls{dpc}
has significantly higher startup costs compared to chipStar.

We analyzed the cases with the most dramatic differences and identified various explanations:
Many of the benchmarks executed very short kernel commands, making the benchmark actually mostly measure the host API call execution speed.
For example, the ``overlay'' benchmark could be sped up significantly by switching off the profiling command queue feature. 
In some cases the device built-ins were more optimized in \chipstar than in \gls{dpc}, in some cases it was the opposite.
For example, when we compared the \chipstar and \gls{dpc} LLVM \gls{ir}s of the device code for the ``nlll'' benchmark, we found that only \chipstar performed the if-conversion optimization that converts some of the very small branches to conditional moves, which provided significant benefits.

In conclusion, \chipstar has lower startup costs than \gls{dpc} and either slightly outperforms or matches \gls{dpc} performance across a variety of benchmarks \myhl{though} there are some exceptions in either direction.

%%%%%%%%%%%%%%%%%%%%%%%%%%%%%%%%%%%%%%%%%%%%%%%%%%%%%%%%%%%%%%%%%%%%%%%%%


\subsection{CUDA on NVIDIA CUDA SDK}

It is interesting to compare the performance of CUDA programs compiled and ran using the NVIDIA's proprietary CUDA versus \chipstar over an OpenCL runtime. In this experiment, we first compiled applications using the CUDA SDK 12.4 to get a baseline. The same benchmark cases (the CUDA versions) were then compiled using \chipstar to the portable fat binary that uses OpenCL as the portability layer which was then run on rusticl/zink, an OpenCL implementation on top of NVidia's proprietary Vulkan driver. The execution time when compiling using strict maths was measured on an NVIDIA RTX 3060 GPU with the results shown in Fig.~\ref{fig:rtx3060-cudasdk-vs-rusticl}. The geometrical of 1.01 tells that the overhead of the chipStar, the OpenCL API and the rusticl OpenCL runtimes is negligible on average. A few outlier cases vary to one direction or another.

\begin{figure*}
       \centering
       \includegraphics[width=1\linewidth]{cuda-vs-rusticl-truncated.pdf}
      \caption{\myhl{chipStar via rusticl/zink stack, speedup over CUDA SDK excluding 36 benchmarks where the performance difference was negligible}}
      \label{fig:rtx3060-cudasdk-vs-rusticl}
\end{figure*}

\subsection{HIP on AMD ROCm}

Since \chipstar can be viewed as a more portable implementation of HIP, it is interesting to compare its speed against the original HIP implementation from AMD. For this comparison, we utilized the HIP compiler and runtime from the AMD ROCm package version 6.2.4 as a baseline to compile and execute a set of HeCbench HIP benchmarks on an AMD Radeon Pro VII GPU. To run the \chipstar fat binaries on the same GPU, we used the rusticl OpenCL implementation on the radeonsi driver. AMD's OpenCL implementation does not support \gls{spirv} input at the time of this writing, preventing its use in this comparison for running the \chipstar binaries.

We have already demonstrated that the \chipstar runtime does not introduce additional overheads in Fig. \ref{fig:best-vs-best} so the following performance differences are a product of rusticl, not \chipstar. Rusticl is still in the development phase and is not yet optimized for all targets, which explains the much larger performance differences in Fig. \ref{fig:radeonprovii_rocm_vs_rusticl} compared to Fig. \ref{fig:best-vs-best}

Given the current state of rusticl, there is not much value in doing an extensive performance anomaly studies. However, one such anomaly is `pnpoly' which is more than 2.5x faster on chipStar as can be seen in Fig. \ref{fig:radeonprovii_rocm_vs_rusticl}. The reason is that ROCm is slightly faster for tile sizes smaller than 32 and for larger ones notably slower. Unfortunately for ROCm, the benchmark tracks times for the largest tile which happens to be the slowest one on ROCm. 

\begin{figure*}
      \centering
      \includegraphics[width=1\linewidth]{rocm-vs-chipstar.pdf}
      \caption{\myhl{chipStar  via rusticl/radeonsi stack speedup over ROCm. Running on AMD Radeon Pro VII.}
      }
      \label{fig:radeonprovii_rocm_vs_rusticl}
\end{figure*}

\subsection{Portability Testing}

In order to test the extent of portability of the runtime API layer based on the OpenCL standard, and to verify that offloading to a GPU via \chipstar can bring speedups in comparison to running on the host CPU, we compiled and executed sets of HeCbench applications on various platforms which included both a CPU and a GPU with a capable enough OpenCL support to execute the same compiled \chipstar fat binary on both devices. Given the experimental state of some of these platforms,  the following performance numbers are not indicative of the true hardware potential and thus should be interpreted solely as indicators of application portability and not performance. The platforms and their results are presented below.

\paragraph{RISC-V CPU \& PowerVR GPU:} In this experiment we utilized the VisionFive2 single board computer for building and running the benchmarks. PoCL~\cite{PoCL} was used for running the benchmarks on the CPU and the proprietary OpenCL driver from Imagination Technologies was used for the GPU.  The results are visualized in Fig.~\ref{fig:intel-visionfive2-gpu-cpu}. The lower performance (0.74 geom mean) of the PowerVR GPU vs RISC-V CPU can be explained by the GPU having much less on-chip resources than most benchmarks could utilize. 
The GPU also has a native workgroup size of only 32 (subgroup size 16), while most HeCBench benchmarks use a workgroup size ranging from 128 to 1024, leading to additional thread context switches. Due to not being able to force the subgroup size to match the warp width in this platform also prevented some of the benchmarks from running. Furthermore, the GPU's limited local memory is used also to store images, samplers, the OpenCL constant data and pointers to global memory - in addition to the shared data of the application kernels~\cite{PowerVRPerfGuide}. The memory limitations and the lack of fp64 support were the main reason some of the test cases were not running at all on this platform.

\begin{figure*}
    \centering
    \includegraphics[width=1\linewidth]{riscv.pdf}
          \caption{\myhl{chipStar on PowerVR GPU, speedup over RISC-V CPU. }}
    \label{fig:intel-visionfive2-gpu-cpu}
\end{figure*}
% \paragraph{Intel(R) Core(TM) i9-13900K} This platform has a 24-core 13th Gen Intel(R) Core(TM) i9-13900K and an Intel UHD Graphics 770 integrated GPU. OpenCL on the CPU and the GPU were supported by Intel's OpenCL drivers  24.35.30872.22. The results are shown in Fig.~\ref{fig:intel-i9-cpu-vs-igpu}.  It's important to note that the i9 CPU, having 24 cores, is quite powerful compared to the iGPU so we don't necessarily expect a significant offloading speedup in this case, especially when the OpenCL CPU driver is able to vectorize OpenCL work-items across the SIMD lanes of the CPU. However, as expected, most of the test cases run on this platform and are significantly faster on the iGPU, validating the portability as well as the offloading benefits.

% \begin{figure*}[tb]
%       \includegraphics[width=1\linewidth]{Figures-2025/igpu-vs-cpu.pdf}
%       \caption{Speed on Intel UHD Graphics 770 iGPU speed normalized to i9-13900K CPU.} 
%       \label{fig:intel-i9-cpu-vs-igpu}
% \end{figure*}

\paragraph{ARM Cortex A53+A73 CPU \& Mali G52 GPU} For the CPU, PoCL~\cite{PoCL} was used as the OpenCL driver while the GPU was supported by the ARM's proprietary OpenCL driver OpenCL C 3.0 v1.r40p0-01eac0.06c59e7df4d178b1ae2ad8082e91ad02. The results are shown in Fig.~\ref{fig:mali-vs-cortex}. A lot of applications were not able to run because of limited memory in the GPU and lack of double precision floating point support. However, various benchmarks showed significant benefits from CPU to GPU offloading, as expected.

\begin{figure*}
      \centering
      \includegraphics[width=1\linewidth]{arm.pdf}
      \caption{\myhl{chipStar on ARM Mali GPU speedup over ARM Cortex CPU.}}
      \label{fig:mali-vs-cortex}
\end{figure*}

%%%%%%%%%%%%%%%%%%%%%%%%%%%%%%%%%%%%%%%%%%%%%%%%%%%%%%%%%%%%%%%%%%%%%%%%%

\section{HPC Application Case Study: GAMESS-GPU-HF}
\label{section:applications}

In order to further test \chipstar in practice, we ported a
complex HIP/CUDA-based HPC application and its dependency library using \chipstar. The test environment for the experiment was the Aurora supercomputer utilizing Intel Datacenter Intel® Data Center GPU Max Series (referred to from here on as PVCs, as in Ponte Vecchio) as the accelerator part~\cite{aurora}.

\subsection{GAMESS-GPU-HF}

General Atomic and Molecular Electronic Structure System (GAMESS~\cite{gamess,gamess2}) is a quantum chemistry software package which implements many electronic structure methods. 
The code base is primarily in Fortran 77/90 with some C/C++ and a CUDA library. Recently a new GPU version of the Hartree-Fock (HF) and RI-MP2 methods were implemented in CUDA which scales to 4096 nodes on Summit, an Nvidia V100-based supercomputer \cite{gamess_cuda1, gamess_cuda2, summit}.
In this porting case we focused on the Hartee-Fock (HF) algorithm used by a CUDA library in GAMESS described in \cite{gamess_cuda1}, which has been ported to HIP. The HF method is a common quantum chemistry method which is often the starting point for other higher-accuracy methods.

The HF method primarily involves the computation of $N^4$ two electron integrals (where $N$ is a measure of molecular system size) as well as matrix contractions of the two electron integrals once they are formed, and computation of eigenvectors.

The two electron integrals are implemented as HIP/CUDA kernels which were optimized for Nvidia GPUs. The kernels total over 20,000 lines of HIP/CUDA  code. The matrix contractions and eigensolves are done on the GPU via calls to the HIP math libraries hipBLAS and hipSOLVER.

Since the application uses ROCm software platform libraries hipBLAS and hipSOLVER, they needed to be ported as well. The required interfaces of these libraries were implemented for Intel hardware by using Intel oneMKL as a backend. This is done with two layers: first, a shim library, H4I-MKLShim \cite{mkl_shim}, which provides shims for the SYCL-based oneMKL functions. This is designed to be used by other libraries which wish to call the oneMKL functions from a different API, such as hipBLAS, to use Intel GPUs. The chipStar project currently implements hipBLAS, hipSOLVER, and hipFFT in the libraries H4I-HipBLAS \cite{chipBLAS}, H4I-HipSOLVER \cite{chipSOLVER}, H4I-HipFFT \cite{chipFFT}, respectively. These allow calls to hipBLAS, hipSOLVER, and hipFFT functions to run on Intel GPUs.

In terms of functionality, the HF code compiles and was verified to run correctly with \chipstar on PVCs. The porting effort was relatively low, with one exception due to a small but significant specification difference in CUDA vs. OpenCL related to kernel thread synchronization: In CUDA group barriers are not counting in exited threads, meaning that there can be early returns from the kernel by a subset of the threads after which it is still legal to perform barrier synchronization with the remaining subset -- the exited threads are just not counted in. In OpenCL this case is undefined behavior and in many implementations can lead to a deadlock. To tackle this gap, an OpenCL extension adding a group barrier with similar semantics would be needed (see \textit{cl\_ext\_alive\_only\_barrier} in Table~\ref{table:extensions}).

\subsubsection{Performance}

The performance of the HF code was measured by compiling and running the same HIP source code on a PVC through \chipstar (Release 1.2.1), an Nvidia A100 through CUDA 12.2.2 with ROCm 6.0.0, and an AMD MI250 through ROCm 6.3.0. To be clear, this is comparing \chipstar on the PVC GPU to native HIP and CUDA on the Nvidia and AMD GPUs. 
To investigate the performance, a HF energy computation of a cluster of 150 water molecules with a STO-3G basis set was run 10 times on PVC, A100, and MI250. The average and standard deviation of the runtimes are displayed in Table~\ref{table:gamess_perf}.

Table~\ref{table:gamess_perf} shows that the total HF energy calculation time (the SCF time) on the Nvidia A100 is shortest (1.66 s) and on one GCD of an AMD MI250 is the longest (4.71 s). The Intel PVC time, through \chipstar with the OpenCL backend, is 2.8s, about 1.7x the runtime on the A100 GPU. From comparing the memory bandwidth and peak double-precision floating point operations possible on an A100 \cite{a100_measured} and a PVC stack \cite{applencourt2024ponte}, we see that we expect memory-bound codes on a PVC stack to take about 1.3x (1.3 $\frac{TB}{s}$ / 1.0 $\frac{TB}{s}$) the time on an A100 and compute-bound codes on a PVC stack to take about 0.56x (9.4 $\frac{TFlop}{s}$ / 17.0 $\frac{TFlop}{s}$) the time on an A100. Note that the 1.3x and 0.56x are upper bounds for performance since not all the runtime is on the GPUs, and the GPU library is a complex code with multiple asynchronous kernels and memory copies. A full performance analysis is out of the scope of this paper. 

Although the Intel PVC time through \chipstar is 1.7x slower than the A100 time, this is not too far from the 1.3x and 0.56x upper bound expectations based on hardware comparisons. The runtime on a PVC stack with \chipstar is competitive with the runtime on other architectures and roughly within expectations based on the compute- and memory-peak comparisons of a PVC stack and an A100. Thus we expect HIP applications currently running on Nvidia and AMD GPUs to run and perform reasonably well with \chipstar on Intel GPUs.

\begin{table}[t]
\centering
\begin{tabular}{cccl}
\cline{1-3}
 &
  \begin{tabular}[c]{@{}c@{}}Average SCF Time (s) \\ (Ratio over\\ Nvidia A100)\end{tabular} &
  \begin{tabular}[c]{@{}c@{}}Standard \\ Deviation\end{tabular} &
   \\ \cline{1-3}
Nvidia A100          & \begin{tabular}[c]{@{}c@{}}1.66\\ (1x)\end{tabular} & 0.01                 &  \\ \cline{1-3}
\begin{tabular}[c]{@{}c@{}}AMD MI250\\ (one GCD)\end{tabular} &
  \begin{tabular}[c]{@{}c@{}}4.71\\ (2.8x)\end{tabular} &
  0.01 &
   \\ \cline{1-3}
\begin{tabular}[c]{@{}c@{}}Intel PVC\\ (one Stack,\\ OpenCL)\end{tabular} &
  \begin{tabular}[c]{@{}c@{}}2.84\\ (1.7x)\end{tabular} &
  0.03 &
   \\ \cline{1-3}
\multicolumn{1}{l}{} & \multicolumn{1}{l}{}                                & \multicolumn{1}{l}{} & 
\end{tabular}
    \caption{Timing comparison for SCF, Fock, and DIIS times (s) for GPU integral code across Intel, AMD, and Nvidia. Average over 10 runs.}
    \label{table:gamess_perf}
\end{table}

%%%%%%%%%%%%%%%%%%%%%%%%%%%%%%%%%%%%%%%%%%%%%%%%%%%%%%%%%%%%%%%%%%%%%%%%%

\section{Related Work}
\label{section:relatedWork}

The origin of \textit{chipstar} is on the HIPCL~\cite{HIPCL} prototype which first tested the concept of compiling HIP programs to fat binaries relying on OpenCL and \gls{spirv}. The \chipstar tool described in this article is a result of an almost a complete rewrite of the HIPCL code base and over approximately three years of continuous development work by multiple partners and HPC users. The HIPCL code base was initially forked to a separate code base to utilize the Level Zero~\cite{l0} low level API directly (HIPLZ~\cite{HIPLZ}) after which the OpenCL backend of HIPCL and the Level Zero backend of HIPLZ were merged to the same code base discussed in this article.
A large number of missing essential features have been implemented since the initial prototypes were published. This article thus significantly expands upon the original poster abstract that introduced the early-prototype-stage HIPCL and now presents a much more mature software stack usable for a wider range of real-world workloads.

When comparing \chipstar to other HIP implementations, obviously the original ROCm, the AMD's official GPU software platform~\cite{ROCm}, is the baseline. ROCm consists of the general purpose programming API compilation and runtime support for HIP, and a set of libraries that support different degrees of compatibility with the CUDA platform. \chipstar is not a new backend in addition to the AMD GPU and NVIDIA GPU backends provided by the AMD's offering, but has an important technical difference: \chipstar aims to offer runtime portability by its open standard based fat binary, removing the need to recompile the input software per target vendor platform, which is the case with ROCm.

SYCLomatic~\cite{SYCLomatic} is a tool contributed by Intel Corporation for converting CUDA sources to the cross-vendor open standard SYCL~\cite{SYCL}. Similar to AMD's HIPify, but in contrast to \chipstar which aims for source-level compatibility, SYCLomatic is a source-to-source conversion tool, which has its good and bad sides. The most apparent implication of relying on source-to-source conversion is more about maintenance aspects than technical ones; it neccessitates the further development of the converted application to proceed using the SYCL API instead or in addition to CUDA. The main drawback is that in reality many code bases are difficult or impossible to convert solely to SYCL without having the CUDA version as a backup due to legacy, risk-management or technical reasons. The main benefit is that SYCL is an open standard, in constrast to CUDA, enabling more fair competition ground between hardware vendors. Thus, being able to target many platforms from a fat binary compiled from the unmodified CUDA/HIP source code base using a \chipstar-style open platform approach can have its benefits. Furthermore, since \chipstar is not a linkage-time or binary translation solution, but requires recompilation, it coincidentally also encourages utilizing and further developing the cross-vendor ecosystem APIs it relies upon. Furthermore, as of this writing SYCLomatic supports only CUDA, not HIP, while HIP has an increasing number of new applications implemented directly using it.

ZLUDA~\cite{ZLUDA} is a tool for running unmodified CUDA binaries on AMD GPUs. It works by reimplementing the \cuda driver API, and converting NVIDIA PTX~\cite{ptx} to the vendor-specific \gls{ir}s. Since it is a ``drop-in solution'' that works at program loading/linkage time, it can execute unmodified CUDA fat binaries, which is very comfortable to the end users as it doesn't require access to the source code of the application. While we see ZLUDA as an excellent tool, it requires reverse engineering CUDA SDK's binary interfaces and keeping up-to-date with the NVIDIA PTX as it evolves. We believe, in the longer term, especially as more of the missing extensions we describe are adopted by OpenCL implementations, \chipstar can provide a more robust solution. 
In addition, ZLUDA also doesn't support HIP as an input and now only targets AMD GPUs, whereas a key goal of \chipstar is extensive cross-vendor portability.

MCUDA~\cite{MCUDA} is the oldest tool we found for porting CUDA programs to non-NVIDIA platforms. MCUDA does source-to-source translation of kernels in a fashion that the translated kernels can execute efficiently on CPUs on a single CPU thread while respecting the barrier synchronization. In the case of \chipstar, since it uses OpenCL as its portability layer, it can similarly target also vectorized CPU execution through CPU-targeting OpenCL implementations such as the Intel OpenCL CPU driver and PoCL's CPU drivers~\cite{PoCL}. Both of them are capable of vectorizing work-items (CUDA/HIP threads) inside work-groups, which translates to implicit autovectorization of CUDA/HIP kernels across CUDA threads and provide the benefits of CPU execution such as easier kernel debugging.

Swan~\cite{Swan} is another early source-to-source tool for CUDA porting. It generates OpenCL code from CUDA, providing similar level of portability as \chipstar does. Another similar tool, CU2CL~\cite{CU2CL} was published in the same year as \cite{Swan}. Neither Swan nor CU2CL are maintained any longer.  In comparison to \chipstar, the main technical differences to these tools are that \chipstar utilizes the latest version of the OpenCL standard to support the newer CUDA/HIP features, uses \gls{spirv} as the intermediate language (no need to generate textual OpenCL C with its limitations) and it doesn't suffer from problems related to source-to-source translations as \chipstar provides source-level compatibility.

The closest comparable CUDA porting tool we could find is CUDA-on-CL~\cite{CUDAonCL}. Like \chipstar, it similarly compiles CUDA programs using Clang/LLVM-based compiler chain to binaries which then execute on OpenCL platforms. However, similarly to Swan and CU2L, it compiles device kernels to OpenCL C whereas \chipstar uses \gls{spirv} as the portable binary format. Other technical differences in \chipstar are related to the use of modern OpenCL standard features to implement some of the features of CUDA. These include using \gls{svm} to implement raw pointers and implementing warp-level primitives such as shuffles using the subgroup features. 

%%%%%%%%%%%%%%%%%%%%%%%%%%%%%%%%%%%%%%%%%%%%%%%%%%%%%%%%%%%%%%%%%%%%%%%%%

\section{Conclusions}
\label{section:conclusions}

In this article, we presented \chipstar, a compilation flow and a runtime for CUDA/HIP applications using open cross-vendor supported standards. In comparison to previous tools, \chipstar's goal is on source-level compatiblity which we believe has longer-term robustness benefits in comparison to binary translation. 
It relies on open standards at the portability layer level which in our opinion has significant inherent far-reaching value.

The performance of HIP benchmarks using \chipstar was shown to be on par or surpass their SYCL/\gls{dpc} versions. We've also shown that \chipstar is very competitive against CUDA running on Nvidia GPU with comparisons to HIP and AMD hardware being less favorable due to OpenCL driver options on AMD hardware.

An example of the source-level compatibility was provided with GAMESS-GPU-HF, a code base with a significant number of kernel code lines. This demonstrates that \chipstar is a useful option for applications that are not feasible to port to more cross-vendor supported open standard input APIs such as SYCL or OpenMP.

\myhl{In the future, we will focus on expanding the HIP/CUDA feature coverage such as Graphs API and collaborative workgroups in} \chipstar.\myhl{ We will also expand the set of supported core libraries (hipFFT, hipSPARSE, etc) in the CUDA and HIP ecosystems to enable more real-world applications.}

%\myhl{In the future, }\chipstar\myhl{ will focus on expanding the HIP/CUDA feature coverage such as Graphs API and collaborative workgroups, while expanding the set of supported core libraries (hipFFT, hipSPARSE, etc) in the CUDA and HIP ecosystems to cover more real-world applications. }

\begin{acks}

This research was supported by the Exascale Computing Project (17-SC-20-SC), a collaborative effort of the U.S. Department of Energy Office of Science and the National Nuclear Security Administration.

This work was supported by the Argonne Leadership Computing Facility, which is a DOE Office of Science User Facility supported under Contract DE-AC02-06CH11357. We also gratefully acknowledge the computing resources provided and operated by the Joint Laboratory for System Evaluation (JLSE) at Argonne National Laboratory.

This manuscript has been coauthored by UT-Battelle, LLC under Contract No. DE-AC05-00OR22725 with the U.S. Department of Energy. The United States Government retains and the publisher, by accepting the article for publication, acknowledges that the United States Government retains a non-exclusive, paid-up, irrevocable, world-wide license to publish or reproduce the published form of this manuscript, or allow others to do so, for United States Government purposes.  The Department of Energy will provide public access to these results of federally sponsored research in accordance with the DOE Public Access Plan (\url{http://energy.gov/downloads/doe-public-access-plan}).

This research used resources of the Oak Ridge Leadership Computing Facility at the Oak Ridge National Laboratory, which is supported by the Office of Science of the U.S. Department of Energy under Contract No. DE-AC05-00OR22725.

This research used resources of the National Energy Research Scientific Computing Center (NERSC), a Department of Energy Office of Science User Facility.

Tampere University authors' contributions were in part funded via the DARE SGA1 Project, which is an European High-Performance Computing Joint Undertaking (JU) under Grant Agreement No. 101202459. The JU receives support from the European Union’s Horizon Europe research and innovation programme and Spain, Germany, Czechia, Italy, Netherlands, Belgium, Finland, Greece, Croatia, Portugal, Poland, Sweden, France and Austria.

\end{acks}
%\printglossary[type=\acronymtype]
\documentclass[lettersize,journal]{IEEEtran}
\usepackage{amsmath,amsfonts}
\usepackage{algorithmic}
\usepackage{algorithm}
\usepackage{array}
%\usepackage[caption=false,font=normalsize,labelfont=sf,textfont=sf]{subfig}
\usepackage{textcomp}
\usepackage{stfloats}
\usepackage{url}
\usepackage{verbatim}
\usepackage{graphicx}
\usepackage{cite}
\usepackage{color}
\usepackage{amssymb}
\usepackage{xspace}
\usepackage{textcomp}
\usepackage{svg}
%\usepackage{subfigure}
\usepackage{caption}
\usepackage{subcaption}
%\usepackage{draftwatermark}
%\SetWatermarkText{DRAFT}
%\SetWatermarkScale{1}

% https://tex.stackexchange.com/questions/326897/vertical-alignment-of-a-turned-cell
\usepackage{rotating}
\usepackage{array,makecell,multirow}

\usepackage{ifthen}
\newboolean{showcomments}
\setboolean{showcomments}{true}
\ifthenelse{\boolean{showcomments}}
{ \newcommand{\mynote}[3]{
     \fbox{\bfseries\sffamily\scriptsize#1}
        {\small$\blacktriangleright$\textsf{\emph{\color{#3}{#2}}}$\blacktriangleleft$}}
  \newcommand{\newtext}[1]{{\color{orange}{#1}}}}
{ \newcommand{\mynote}[3]{}
  \newcommand{\newtext}[1]{#1}}

% Please use a named note with this macro to comment the text:
\newcommand{\pj}[1]{ \mynote{PJ}{#1}{blue} }
\newcommand{\bv}[1]{ \mynote{BV}{#1}{green} }
\newcommand{\mb}[1]{ \mynote{MB}{#1}{cyan} }
\newcommand{\cb}[1]{ \mynote{CB}{#1}{magenta} }
\newcommand{\pv}[1]{ \mynote{PV}{#1}{yellow} }
\newcommand{\ba}[1]{ \mynote{BA}{#1}{brown} }
\newcommand{\kh}[1]{ \mynote{KH}{#1}{red} }
\newcommand{\hl}[1]{ \mynote{HL}{#1}{orange} }

\newcommand{\hiplz}{\texttt{HIPLZ}\xspace}
\newcommand{\hipcl}{\texttt{HIPCL}\xspace}
\newcommand{\hip}{\texttt{HIP}\xspace}
\newcommand{\opencl}{\texttt{OpenCL}\xspace}
\newcommand{\lz}{\texttt{L0}\xspace}
\newcommand{\sycl}{\texttt{SYCL}\xspace}
\newcommand{\cuda}{\texttt{CUDA}\xspace}
\newcommand{\chipstar}{\textit{chipStar}\xspace}
\newcommand{\func}[1]{$#1$\xspace}
\newcommand{\type}[1]{$#1$\xspace}

\hyphenation{op-tical net-works semi-conduc-tor IEEE-Xplore}

% IEEE policy on preprints seems to be reasonable:
% https://journals.ieeeauthorcenter.ieee.org/become-an-ieee-journal-author/publishing-ethics/guidelines-and-policies/submission-and-peer-review-policies/#electronic-reprints

% https://journals.ieeeauthorcenter.ieee.org/submit-your-article-for-peer-review/the-ieee-article-submission-process/
% TPDS manuscript types and submission length guidelines are described below. All page limits include references and author biographies. For regular papers, pages in excess of these limits after final layout of the accepted manuscript is complete are subject to Mandatory Overlength Page Charges (MOPC). Note: All supplemental material must be submitted as separate files and must not be included within the same PDF file as the main paper submission. There is no page limit on supplemental files. 

% Regular paper – 12 double column pages (Submissions may be up to 18 pages in length, subject to MOPC. All regular paper page limits include references and author biographies.)

\begin{document}

\title{\chipstar: Making HIP/CUDA Programs Cross-Vendor Portable by Relying on Open Standards}

%\author{pekka.jaaskelainen }
%\date{March 2023}

\author{Pekka Jääskeläinen, Henry Linjamäki, Michal Babej, Peng Tu, Sarkar Sarbojit, Ben Ashbaugh, Colleen Bertoni, Kevin Harms, Paulius Velesko, Philip C. Roth, Rahulkumar Gaytri, Jisheng Zhao, Karol Herbst, Brice Videau
        % <-this % stops a space
\thanks{Pekka Jääskeläinen, Henry Linjamäki, Michal Babej, Peng Tu, Sarbojit Sarkar and Ben Ashbaugh(?) are with Intel Corporation. \textit{Corresponding author: Pekka Jääskeläinen, email: pekka.jaaskelainen@intel.com}.}
\thanks{Pekka Jääskeläinen is also with Tampere University, Finland. }
\thanks{Paulius Velesko is with Pagan LC.}
\thanks{Philip C. Roth is with Oak Ridge National Laboratory, ... }
\thanks{Rahulkumar Gaytri is with National Energy Research Scientific Computing Center, ...}
\thanks{Jisheng Zhao is with Georgia Institute of Technology, Atlanta, Georgia.}
\thanks{Karol Herbst is with Red Hat, Inc.}
\thanks{Brice Videau, Colleen Bertoni and Kevin Harms are with Argonne National Laboratory, ...}
\thanks{\pj{The authors are not in any particular order. I put myself as the 1st author as I'm leading the writing, and then I ordered the co-authors according to their affil.}}
%\thanks{This paper was produced by the IEEE Publication Technology Group. They are in Piscataway, NJ.}% <-this % stops a space
%\thanks{Manuscript received April 19, 2021; revised August 16, 2021.}}

% The paper headers
\markboth{IEEE Transactions on Parallel and Distributed Systems,~Vol.~X, No.~Y, Month~YEAR}%
{Shell \MakeLowercase{\textit{et al.}}: A Sample Article Using IEEEtran.cls for IEEE Journals}}

%\IEEEpubid{0000--0000/00\$00.00~\copyright~2021 IEEE}
% Remember, if you use this you must call \IEEEpubidadjcol in the second
% column for its text to clear the IEEEpubid mark.

\maketitle

%%%%%%%%%%%%%%%%%%%%%%%%%%%%%%%%%%%%%%%%%%%%%%%%%%%%%%%%%%%%%%%%%%%%%%%%%

\begin{abstract}

%This document describes the most common article elements and how to use the IEEEtran class with \LaTeX \ to produce files that are suitable for submission to the IEEE.  IEEEtran can produce conference, journal, and technical note (correspondence) papers with a suitable choice of class options.

%\pj{This first paragraph is optional, we can remove it:}
%Due to NVIDIA dominating the GPU market and despite its lack of cross-vendor portability, the C/C++-based application programming interface of CUDA and its related key libraries are still used in a significant fraction of software utilizing GPU-based acceleration. AMD's ROCm and its Heterogeneous-compute Interface for Portability (HIP) aims to alleviate the CUDA's lack of portability by providing a route out from the NVIDIA CUDA platform to AMD's devices.

  We describe \chipstar, an open source software stack which enables building unmodified CUDA and HIP programs to binaries that rely solely on open cross-vendor standards OpenCL and SPIR-V. The relevant technical aspects of \chipstar and the feature mismatches between CUDA/HIP APIs and OpenCL are discussed along with a set of standard extension proposals to bridge the essential gaps in the future.
  The key benefit of the software stack is its portability, which is demonstrated by providing performance evaluations on a diversity of less common CPU/GPU platforms including RISC-V/PowerVR and ARM Mali. A comparison against the original AMD HIP platform provides a geometric mean of X.XX \pj{TODO (Henry)}, a reasonable price to pay for the enhanced portability.    
  Although being a relatively young open source code base, \chipstar is now considered mature enough for wider testing and even production use, which is supported by successful porting of GAMESS, a complex HPC application which was deployed in the Aurora supercomputer.

\end{abstract}

\begin{IEEEkeywords}
CUDA, HIP, OpenCL, SPIR-V, Portability, Shared Virtual Memory
\end{IEEEkeywords}

%%%%%%%%%%%%%%%%%%%%%%%%%%%%%%%%%%%%%%%%%%%%%%%%%%%%%%%%%%%%%%%%%%%%%%%%%

\section{Introduction}

\IEEEPARstart{W}{alled} garden strategy is popular among market dominating companies. Its idea is to lock-in customers to company's products by making escaping the gates of the garden as costly as possible. NVIDIA's CUDA software platform is considered to be one of such walled gardens. It in part helps NVIDIA to expand and keep a foothold of their GPU market advantage, and at the same time maintain high innovation pace on the software APIs since there is no need to work with standardization committees that always have to aim for a consensus among multiple participating vendors.

Naturally, for end-users and the competing hardware vendors, the situation of a single-vendor dictated API is not ideal. End-users would prefer open standard software interfaces that enable switching the targeted hardware without incurring significant non-recurring engineering costs required for porting the applications and libraries to a new software platform just to be able to utilize the newly purchased hardware optimally. Similarly, other hardware vendors, aiming to get their piece of the market pie, would prefer an API that is not controlled by a single vendor.

AMD's ROCm~\cite{ROCm} software platform and its Heterogeneous-compute Interface for Portability (HIP) language~\cite{hip} helps escaping the CUDA walled garden by providing a route out from the NVIDIA CUDA platform to AMD's devices. HIP defines a subset of CUDA that is more easily portable to various hardware, thanks mainly to omitting various advanced features available in the later CUDA versions.
%(some of these features are discussed in Section~\ref{subsection:compatgaps}).
In order to enable easy automated transition path from CUDA applications, HIP is largely a copy of a CUDA C/C++ API subset with a few minor differences and renamed functions. HIP alleviates the CUDA portability problem, but doesn't solve it satisfactorily due to AMD targeting their self-specified low level ROCm APIs which are not actively supported on non-AMD platforms.
%\kh{this is related to the other point I brought up. It can be read as AMD supporting other HW somehow, but I think the truth is, it's not supported at all, especially as of today ROCm only emits AMD GPU ISA}\pj{OK. Then "for Portability" is quite misleading today. Do you have a reference to a manual or somewhere that says the CUDA output is not supported anymore?} 
An open source HIP/CUDA software platform solely based on open standards with a sincere aim for cross-vendor portability is still lacking.

With the \chipstar software stack described in this article we aim to help the CUDA/HIP
portability challenge. In contrast to previous solutions that either require source-to-source
conversion from CUDA programs~\cite{SYCLomatic}, that can lead to costly multiple codebase maintainance, or target binary-level compatibility of existing CUDA/HIP programs~\cite{ZLUDA} that
rely on questionable reverse engineering of proprietary binary interfaces (a brittle longer-term strategy), \chipstar chooses a middle-ground approach which enables source-level compatibility of HIP/CUDA programs by compiling them to a runtime portable ``fat binary'' that utilizes solely open standards and can execute on any platform supporting the required standard features without recompilation.

With this article we make the following contributions:

\begin{enumerate}
  \item We publish internal design choices of a software platform \chipstar that enables porting applications from NVIDIA-driven CUDA and AMD-driven ROCm platforms to any current and future platform supporting the cross-vendor open standards OpenCL and SPIR-V,
  \item evaluate its performance in comparison to running the CUDA applications directly using NVIDIA SDK or converting the applications to a popular open-standard based CUDA alternative SYCL~\cite{SYCL} and
  \item demonstrate the usability of OpenCL as a portability layer to implement other languages/APIs on top. The portability is shown by providing performance numbers on a RISC-V CPU \& PowerVR GPU and ARM platforms as well as on discrete GPUs from all main vendors. 
\end{enumerate}

The rest of the article is structured as follows: Section~\ref{section:portabilityAPIs} discusses our rationale for choosing OpenCL~\cite{OpenCL} and its device-side program representation SPIR-V~\cite{SPIRV} as the core APIs to support runtime portability in \chipstar.
Section~\ref{section:implementation} details the key technical issues in implementing the HIP/CUDA runtime on these APIs, while Section~\ref{section:compilation} focuses on the compilation aspects. Performance evaluation results are shown in Section~\ref{section:performance}. 
Section~\ref{section:applications} presents the GAMESS porting case study using \chipstar, and finally Section~\ref{section:conclusions} concludes the article.

%%%%%%%%%%%%%%%%%%%%%%%%%%%%%%%%%%%%%%%%%%%%%%%%%%%%%%%%%%%%%%%%%%%%%%%%%

\section{Heterogeneous Platform Hardware Abstraction Layers}
\label{section:portabilityAPIs}

As a technical background, we provide our considerations for the ``hardware abstraction layer'' API options forming the platform portability layer for \chipstar. The discussion is split to runtime APIs used to control the execution from the host side and device program representations providing an abstraction for the kernel side programs in a portable manner. Due to the abundance of potential target devices available for acceleration, we consider it important to be able to embed the device programs in an open standard-based intermediate representation (IR) and using JIT compilation for lowering the program to the target ISA of the accelerator at deployment or launch time. This enables future-proof ``fat binaries'' which can be supported on new platforms by implementing the specification of the IR. Another alternative would be to provide only source-level
compatibility where the application needs to be recompiled for each host and a device pair of interest, hindering binary distribution.

\subsection{Runtime APIs}

In practice, both CUDA and HIP are single-vendor supported programming models.
This is reflected, for example, in their platform property APIs which define limited
queries for device properties, highlighting features in each vendor's
GPU offerings. The goal of \chipstar is to
expand the portability of applications implemented using the CUDA/HIP APIs.
Therefore, the key requirement to the underlying ``platform/device portability API'' is to
cover as many of the essential features of CUDA/HIP as possible to
provide functional correctness and to exploit the potential performance benefits.
This includes,
for example, parallel and asynchronous execution of tasks, overlapping of
data transfers with task execution, and by providing access to
shared memory communication, if available. Furthermore, the portability API
should provide services to enhance performance portability of
the implementation by allowing to query the capabilities of the
devices to tune the execution at runtime to match the target's features.

There are not a large number of choices for such a runtime API, especially
if limiting the list to alternatives that enjoy official driver support from
multiple accelerator vendors or to those that have a portable long-maintained open source
implementation. In this regard, an open standard based API that has increased
in popularity is SYCL~\cite{SYCL}. SYCL
resembles CUDA in being a C++-based single-source API and
is a potential option for a portable runtime layer.
%It diverged from its original goal of an improved C++ binding to OpenCL to a more independent ecosystem with multi-backend implementations.
However, SYCL doesn't provide a runtime API or a way to query the
features of the underlying platform to tune the execution at
runtime.
%However, we consider OpenCL being a better match for portability
%layer use since it's a lightweight C API and has proprietary driver
%support from all major vendors and defines device and platform
%queries that can be used to tune the execution at runtime.
% SYCL should also support task graphs well:
%OpenCL also has a powerful task graph abstraction that can be interfaced
%with multiple in-order and out-of-order command queues~\cite{OpenCLTLP},
%events and, more recently, the relaunch of the task graphs can
%be optimized with command buffering,

Recently, OpenMP~\cite{OpenMP} has been considered for portability layer usage and,
in fact, specifically for implementing CUDA in~\cite{10.1145/3559009.3569687}.
While OpenMP enjoys support from a wide range of vendors,
we believe OpenMP is not ideal for this use case since it doesn't
define a device-side program representation, making future-proof cross-vendor portable fat binary generation difficult. It also can be considered
a high-level ``application-programmer-facing'' API similarly to SYCL,
thus offers constructs and overheads for programmer-productivity which are unneccessary for a portability layer.

Heterogeneous System Architecture (HSA) is an open heterogeneous platform
specification that also defines a runtime API~\cite{HSA,HSART}.
A key differentiating feature of HSA is that it standardizes on shared virtual memory,
making system-wide virtual memory addressing a required feature from
implementations. In hindsight this requirement was likely too much
too early, as system-wide virtual memory support is only relatively recently appearing
in hardware and still usually requires explicit allocation or mapping
calls from the programmer.
For this work, HSA could have been a valid choice for
a lightweight portability layer, but activity on the specification has seized with mainly AMD using only
selected parts of it in their software stack (the ROCr runtime component).
%\kh{It might make sense to talk with Dave Airlie about HSA specifically, from what I've heard is HSA was an attempt by AMD to make something cross-vendor, but in the end they were the only one deciding/doing anything. But I was never involved in HSA myself or even cared at all. I just know that Dave has some experience there.}\pj{I was also personally somewhat involved: I attended some of the meetings and implemented the obsoleted GCCBRIG frontend for HSAIL consumption for a then-client. Indeed it was like what you say; it just didn't get enough traction from other vendors and AMD dominated and everyone saw the standarding being AMD dominated (still do) although ARM was there for instance.}

Another option to consider would be Vulkan~\cite{Vulkan} since it also
provides a compute pipeline stage which allows specifying general purpose compute
kernels. However, the
feature set of the compute kernels lacks some of the OpenCL's features such as SVM which
would require further standard extensions.
%\kh{I don't think this is a strong point, because most of the kernel features are implemented in software anyway. All of the OpenCL C builtins can be implemented in OpenCL C as done by the libclc sub-project in LLVM. Other features like Program scope variables, printf or pipes are just buffers at runtime you pass into the kernel.}\pj{Right. You can implement everything in software, but can one provide efficiency/perf. portability if the standard lacks some of the features? If we cannot pinpoint any such features (other than that people consider it even more boiler plate to write than OpenCL), I can just mention it as a valid alternative.}\kh{There might be some micro-optimization possible fine tuning the implementation to specific hardware, but the GPU code to implement those built-ins is generally quite complex anyway. I think there is value in having a OpenCL C builtin library which could output vulkan SPIR-V instead for ease of use as currently it's only supporting OpenCL SPIR-V. For rusticl + zink what happens in mesa is, that we convert OpenCL SPIR-V to nir (mesa internal IR) which then gets translated to Vulkan SPIR-V and that works perfectly fine. So I think if somebody really cared, they could make the libclc stuff output vulkan SPIR-V instead.}\pj{OK. What about "platform level features" such as SVM?} 
It might be that in the future
the feature gap gets narrower and it would become a viable option. Meanwhile, layering OpenCL on top of Vulkan is an interesting option pursued by multiple open source projects to cover the devices which previously only had Vulkan driver support.

Level Zero~\cite{l0} is an API at a similar level of abstraction as HSA and OpenCL.
However, it's currently only supported by Intel's device, thus is not suitable for future-proof cross-vendor fat binaries.
%For \chipstar, Level Zero has a benefit that it uses the open standard SPIR-V as the device program
%representation, which in fact made it relatively easy to add an option for using Level Zero directly to control Intel GPU devices as an alternative to OpenCL~\cite{HIPLZ}.

Interestingly, OpenCL was originally created to provide an open cross-vendor alternative to the proprietary CUDA GPGPU programming model. It hasn't received wide support from application developers likely because it has been considered too low level and unproductive with the driver and feature support lagging behind the proprietary alternatives. However, as demonstrated by \chipstar, thanks to the official support from multiple vendors for the minimum feature set of the version 3.0 of the standard and multiple long-maintained open source implementations available, OpenCL provides a good portability layer for implementing other higher-level programming models and APIs on top.

\subsection{Device Program Representations}
\label{subsec:deviceProgramRepresentations}

Heterogeneous platforms suffer from the problem of device-side program description portability. There is a wide range of instruction-set architectures the kernels can target, and when the program is distributed in a binary form, the targets are known only at run time.
Thus, the choice of the format in which the device programs (kernels) are stored is critical as it should cover as many of the potential targets as possible.
Furthermore, the representation should be ``future-proof'' in a sense that the produced fat binaries could be made run in entirely new platforms by only referring to the API specification.
At the time of this writing, there still seem to be no clear winning program representation in this regard and various portable implementations of application-facing APIs are resorting to very fat binaries which store copies of the device program in multiple (virtual) instruction-set architectures to cover the various targets and offloading runtimes it might encounter at execution time. This is the case with~\cite{10.1145/3559009.3569687} and originally in AdaptiveCpp~\cite{10.1145/3529538.3530005}.

Recently AdaptiveCpp started storing kernels in the LLVM~\cite{LLVM} compiler Intermediate Representation (IR) instead of storing multiple different binaries depending on the target. In this scheme, LLVM IR is lowered to various target-dependent formats at runtime at the point when the target is known~\cite{OpenSYCLfatbin}.
This approach has benefits in comparison to storing abundance of device binaries in the another alternative, and works in theory, but it is also known that LLVM IR is not supposed to be a portable program representation as it can embed target-specific intrinsics, has target specific data layout and endianness among other challenges.
LLVM IR is not guaranteed to be stable across LLVM versions, which means that the fat binaries should have access to an LLVM library of version the IR was generated with, which at worst requires to embed the LLVM library along and the required backends to the fat binary, forming an unnecessary dependency.
The problem of LLVM IR not being target-independent nor stable across LLVM versions was attempted to be addressed by earlier Standard Portable Intermediate Representation (SPIR) versions 1.2 and 2.0~\cite{SPIR2}: These first SPIR versions were designed to support OpenCL C language kernels and were based on defined versions of LLVM IR, which proved to be difficult to maintain long term.
%\pj{Ben: do you have insights on the historical reasons here?}.
The LLVM-based SPIR-versions were later obsoleted in favor of the SPIR-V~\cite{SPIRV} format.
The goal for SPIR-V is to provide a robust cross-vendor specified intermediate language which is not affected by LLVM upstream changes and that shares specification effort with the Vulkan community.

HSA specification defines an intermediate language called HSAIL and a binary representation called BRIG~\cite{HSAIL}. A key technical difference of HSAIL in comparison to the SPIR-V format \chipstar chose to use is that HSAIL had a fixed number of registers and an address space for spills unlike SPIR-V, which has infinite virtual registers due to being based on the Static Single Assignment (SSA)~\cite{SSA} representation.
Similarly to Level Zero after it, HSA made a choice to not define a higher-level programming language (like OpenCL C) for the device programs, but only standadized a low level IR.
Like is the case with the HSA runtime specification, however, the activity on the HSAIL spec has stalled.
There was also a GCC-based frontend for consuming BRIGs in a target-portable fashion, but after activity on HSA quieted, the ``BRIG frontend'' was removed from the upstream GCC source code repository in a May 2021 commit.

As a conclusion, while SPIR-V OpenCL environment support from processor vendors is not very extensive as of this writing, it seems to be still the best option for a cross-platform representation given it's an open standard defined democratically by multiple hardware vendors and is relied upon by OpenCL and Vulkan and SYCL implementations among others. 
%
Thanks to open source tooling support available and useful SPIR-V producers such as \chipstar and DPC++ appearing, the list of supported targets is expected to grow in the future.

%Since it relies on a well-specified specification, the fat binaries produced by \chipstar relying on OpenCL and SPIR-V are ``future-proof'', making it feasible to add binary-level support to new devices while only referring to the specifications.

%The two core OpenCL features required by all CUDA/HIP applications when built with \chipstar are the SPIR-V input and coarse-grain SVM, both of which are optional features of the OpenCL 3.0 standard.
%Although OpenCL has again risen in popularity in the recent years, thanks to its more easily minimally-implementable 3.0 version, as of this writing, only ARM GPUs, Intel GPUs and Intel CPUs have vendor-provided OpenCL drivers that support SPIR-V input.

%Fortunately there are various active open source OpenCL implementation projects that can be used and expanded to fill up the lack of features in the proprietary drivers at least until they catch up. The two most vibrant ones are Rusticl and Portable Computing Language (PoCL~\cite{PoCL}). These two projects were used utilized to extend the portability of \chipstar-produced fat binaries to CPU targets.

\section{Implementing HIP/CUDA on OpenCL Runtime API}
\label{section:implementation}

The primary goal for \chipstar is to support the subset of CUDA features
as defined by HIP and expanding the feature set beyond it whenever feasible
while relying on the chosen open standard APIs as much as possible.
In this section, we discuss how the OpenCL/SPIR-V specifications match with
the commonly used features of CUDA/HIP and identify the most impactful gaps
that we believe should be covered in the future.
%At the time of this writing, HIP refers to features at CUDA version 9.0
%or older.
%, thus excludes modern functionality available in later NVIDIA
%devices such as page-fault relying unified memory.

\subsection{Memory Model}

Due to their common history in GPGPU programming, CUDA/HIP and OpenCL, share various
platform and memory model abstractions. For example, ``device memory'' is the same as
``global memory'' in OpenCL terminology (``shared'' is ``local'').
To avoid confusion in terminology we use only the CUDA/HIP terms in the rest of
this article. Similarly, we refer to the original CUDA versions when talking about
functions that have their counterparts in the HIP API.

A key difference between OpenCL and CUDA that required addressing was the
fact that CUDA implicitly infers the address space of the data in the device
program side whereas in OpenCL (before v2.0) the address space must be declared explicitly.
The CUDA's implicit address space inference is similar to the 'generic'
address space concept introduced in OpenCL v2.0, which was utilized to bridge
this gap.

The simplest interface in CUDA's host-side device memory management is \func{cudaMalloc()}.
It returns a raw pointer to the targeted device's global memory, instead of an opaque handle
as is the case with OpenCL's basic buffer management functionality. This presents a small
but significant difference from the OpenCL v1.x specification for device memory management;
OpenCL v1.x only provides a buffer management API (\func{clCreateBuffer()} and others) which
returns opaque \type{cl\_mem} handles.
%
The opaque buffer handles cannot be used to implement CUDA device memory allocation
because they do not provide access to the underlying raw device address or passing addresses in other data
structures, which is allowed with the CUDA device pointers. In order to implement these
capabilities, we utilized the Shared Virtual Memory (SVM) API that first appeared in
the OpenCL v2.0. The raw pointers was the another key difference along with the
implicit address space inference which required us to lift the minimum OpenCL
version to v2.0 to support even the most basic CUDA programs.

The SVM allocation API returns a raw pointer to a shared
virtual address space region. The ``Coarse Grained buffer SVM'' (CG SVM) variant can be used to
implement the basic device memory allocation. Mapping device memory allocation to CG SVM
has a drawback that the device driver must support some of the unneeded SVM features such as
mapping the allocated regions to the virtual address space although just returning physical
device memory pointers would suffice. This means that the \chipstar implementation is actually
implementing CUDA's Unified Memory model by default. To alleviate the potential performance
impact of this, \chipstar can also use the Intel Unified Shared Memory (USM)
extension (cl\_intel\_unified\_shared\_memory), if supported by the runtime. USM enables allocating strictly
device-only allocations, but still returns virtual pointers, which can be problematic for some implementations. In order to provide an allocation API matching the basic \func{cudaMalloc()}'s needs, we introduced a new extension (cl\_ext\_buffer\_device\_address) that enables querying the raw device pointer of a cl\_mem allocation without needing to map the buffer to the same address range in the host's virtual memory. \chipstar can use any of these alternative APIs.

CUDA provides an API to \textit{pin} memory so it's kept resident in the host memory and
optionally made accessible by devices from kernel code and is not swapped out to disk. 
The primary APIs to this functionality are
\func{cudaHostAlloc()} and \func{cudaHostRegister()}. The former allocates pinned
memory directly and the latter pins a previous host allocation. \func{cudaHostAlloc()}
is simple to implement with coarse grained SVM since by the coincidence of using
a shared virtual memory allocation, the buffers are by default accessible in both the host and
the device using the same pointer. However, the allocation might not be resident for the
duration of the execution, for example, if a CPU device is allowed to swap out such
allocations. That aspect, however, is only potentially inspectable as a performance difference.
\func{cudaHostRegister()} is a bit more challenging to implement on top of CG SVM since it
allows registering a host address range to be a pinned region accessible both from the host and
the device \textit{after}
the host memory has been allocated. Since the allocation might not be originally
been allocated with the OpenCL SVM allocation API, but with a system memory allocator or even
from the stack, to implement correct functionality in this case, \chipstar creates
a shadow buffer using \func{clSVMAlloc()} and synchronizes it with the host region at
kernel start and end points. OpenCL 2.1 added a new \func{clEnqueueSVMMigrateMem()} API that enables fine grained specification of where regions of SVM are migrated, but is not useful for this case since the source of \func{cudaHostRegister()} can be any host memory area whereas the API handles only SVM allocations.
%\kh{OpenCL 2.1 added clEnqueueSVMMigrateMem to give hints to the runtime where to put the content of the buffer, might make sense to mention it here and say why it's not a good fit.}\pj{Done.}\kh{I think something at the end is missing here}
% https://developer.nvidia.com/blog/unified-memory-cuda-beginners/

The later NVIDIA architectures since compute capability 6 support on-demand page migration which
relies on hardware memory management unit (page fault based buffer migrations) for coherence
on the Unified Memory allocations. This frees the programmer from the need to perform explicit memory allocation and synchronization calls. The functionality maps to the Fine-Grained System SVM of OpenCL, but since its support by hardware and drivers is very rare at the time of this writing,
it is not yet implemented by \chipstar.

\subsection{Tasks and Events}

The semantics of CUDA \textit{streams} and the ability to execute tasks/commands
asynchronously maps well to the \textit{command queues} of OpenCL. Each stream is expected
to execute commands in-order, which matches the in-order command queue semantics of OpenCL.
Commands are allowed to execute concurrently even within in-order command queues in OpenCL,
as long as the results are not observable from the outside, enabling concurrent kernel
execution~\cite{OpenCL}.

To facilitate out-of-order execution from CUDA/HIP programs the programmer has to rely
on the explicit event synchronization and recording APIs which are implemented
using OpenCL out-of-order queues by \chipstar, if supported by the target. The CUDA/HIP event API differs from the OpenCL: In CUDA/HIP the user is responsible for explicitly creating and recording events, while in OpenCL the runtime implicitly creates events when enqueuing commands. Recording events in \chipstar is implemented by creating a new marker-type \type{cl\_event} (\func{clEnqueueMarkerWithWaitList()}) when \func{cudaEventRecord()} is called.

\subsection{Textures}

\chipstar supports only a simple subset of texture objects due to a limitation in OpenCL images. The notable differences between HIP/CUDA and OpenCL are that the texture objects are pointers to opaque C/C++ structures whereas in OpenCL/SPIR-V there is a special type per image dimensionality and that the texture objects can be loaded indirectly whereas OpenCL images can be only passed to kernels as kernel arguments. Therefore, some constructs such as the following cannot be expressed in SPIR-V:

\begin{verbatim}
  hipTextureObject_t Tx = ...;
  Ty Tv = cond ? tex2D<Ty>(Tx, X, Y) 
               : tex1D<Ty>(Tx, X)
\end{verbatim}


% Pekka> Hiding this for now. It's an interesting feature, but not strictly needed for CUDA/HIP.
%\subsection{Lower Layer API Interoperability}

%\pj{TODO Sarbojit: Describe the HIP-OpenCL and HIP-SYCL interoperability APIs and their use cases.}
%\pj{Can you add code examples of using the different interop APIs?}

%The native interoperability API can be used to initialize HIP context (with assigned device \& command queue) from a set of native (LevelZero/OpenCL) object handles, or in the opposite direction to retrieve a set of native handles from an existing HIP context. Thread-safe use of handles is currently left to the application (which should be non-issue with OpenCL since it is thread-safe). Additionally, there are two APIs that create a HIP event from native event handle, and vice-versa. These can be used for interoperability of HIP code with native code while maintaining asynchronous execution.\pj{Can we share buffers somehow between APIs? Or is that down to the "external memory extension"?}\pj{Is the SYCL interop via LZ/OpenCL, no direct API calls?}

%\subsection{OpenCL-CUDA/HIP Compatibility Gaps}

%\pj{Pekka TODO: This is a verbatim copy from HIPCL, to update:}
%Most of the HIP API maps trivially to the OpenCL API, with some notable exceptions which might call for new OpenCL extensions:\pj{TODO: We should just make them extensions (proposals) to clean up the story.}

%\mb{Pekka TODO: do we also list APIs which can be implemented but aren't yet (because nobody's done the work) ? looking quickly at CHIPBindings.cc, there are >50 hip API functions which have not been implemented, things like Peer2peer, hipIPC*, hipModuleOccupancy*, hipProfiler*, hipMemPool*, hip{Malloc,Free}Async etc; some might require OpenCL extensions }
%\pj{I think not worth listing here, as it's only a matter of time when these are implemented and if apps do not use them, they %are not high prio.}

%\begin{itemize}

%\item {hipGetDeviceProperties()}: for certain device properties, there is no portable way to get the information via the OpenCL device query API.\pj{this should be an easy extension}

% Pekka> I think we can do without this as it's visible only in terms
% of latency/performance to the user, and there should be also other
% similar features which can be observed only in terms of perf., not
% functional correctness (e.g. the typical concurrency to parallelism mapping).
%\item {hipSetDeviceFlags()}: the flags to this call control how the host thread interacts with the driver thread while waiting for the device (yield the host thread to OS, or spin wait).
%
%\item{hipEventCreateWithFlags()}: provides per-event control of the synchronize behaviour (yield thread/spin wait). However, these APIs affect only performance, not correctness, thus can be implemented as no-operations.

% Pekka> Cannot we really implement this without an extension even if we had kernel metadata to traverse? \mb{possibly, if we can always figure out the correct alignments & padding}
%\item {hipModuleLaunchKernel()}: passing args by ``extra'' parameter requires an API for setting all kernel arguments at once.\pj{a new clEnqueueNDRange variation with a HSA-style-specified exact layout argument buffer layout might be useful in any case.}

%\item {hipGraph API}: the API to create, update & launch graphs. The existing cl_khr_command_buffer extension is not sufficient, since we need to work with SVM. (discussed below in the "opencl and spirv extensions" section).

%\item {hipHostRegister}: we'll need an OpenCL extension to implement this (unless there is something already we could use, i haven't checked).

%\end{itemize}

%%%%%%%%%%%%%%%%%%%%%%%%%%%%%%%%%%%%%%%%%%%%%%%%%%%%%%%%%%%%%%%%%%%%%%%%%

\section{Compilation Aspects}
\label{section:compilation}

This section discusses the compilation flow used by \chipstar. We introduce the overall compilation flow, the kernel built-in library implementation and summarize our findings on the key needs to extend the OpenCL/SPIR-V standards to bridge key feature gaps between the specifications.

\subsection{The Compilation Flow}

The offline compilation flow of \chipstar is based on the LLVM Project's~\cite{LLVM} Clang~\cite{Clang} frontend. The overall compilation process is shown in Fig.~\ref{fig:compilation}. It relies on the CUDA/HIP frontend of Clang, which was extended to produce SPIR-V binaries as an option to PTX or AMDIL for the device program. The LLVM \func{opt} tool is used to invoke special LLVM passes provided by \chipstar for lowering HIP features to the OpenCL-SPIR-V environment. The SPIR-V translation is performed using Khronos' LLVM-SPIRV-Translator tool.

\begin{figure*}
    \centering
    \includegraphics[scale=1]{figs/chipstar-compilation-v2.pdf}
    \caption{The offline compilation flow.}
    \label{fig:compilation}
\end{figure*}

Most of the compilation related changes have been upstreamed to the LLVM project and very little compilation related functionality remains within the \chipstar code base. The notable exceptions are compiler passes that handle CUDA vs. OpenCL differences in \func{printf()}, implement a device side \func{abort()} feature, handling of the CUDA's device-side global variables, and an indirect memory access analyzer. The indirect memory access analyzer marks kernels that are known to not indirectly access allocations, which removes the unneccessary synchronizations for the majority of
benchmarks seen so far. Otherwise, due to the CUDA's memory model, each launcher kernel can potentially access any previously allocated buffer, inducing significant unneccessary
data synchronization overheads in the common case where the kernels only access buffers set through their arguments.

%One of the performance-impacting device program passes done in \chipstar is indirect memory
%access analysis: In CG SVM, memory consistency between device and host memories is guaranteed at
%the execution boundaries of kernel commands referring to the SVM allocation. To enforce
%data synchronization, kernels referring to the SVM allocations must either refer to the
%SVM allocations as kernel arguments, or be explicitly marked to indirectly use other
%allocations. The latter poses a challenge, since in principle any kernel can refer to any
%previous allocation as CUDA device pointers can
%be passed inside data structures or global variables.
%For CG SVM,
%OpenCL's \func{clSetKernelExecInfo()} must be used to list all potentially used SVM allocations
%that are not referred to by the argument list.
%This poses a signficant performance overhead risk since the \chipstar runtime must play it safe
%and register all possible previous allocations to any launched kernel, unless proven that the
%kernel doesn't refer to a particular allocation. For applications with a lot of allocations and
%kernels that only use subsets of the buffers, a lot of unnecessary data
%synchronizations between the host and the device memories can happen. To alleviate this
%problem for the seemingly common case of only referring to buffers in the argument list,
%\chipstar implements a kernel analyzer that 

%\pj{Henry (?) TODO: Discuss the eager compilation slowness problem and how it was solved in chipstar and PoCL-level0.}

Fig.~\ref{fig:online-compilation} shows the online compilation flow from SPIR-V to device code in \chipstar runtime. A SPIR-V module is compiled just-in-time when a kernel associated with it is launched. To enhance runtime portability, the online device builtin library provides HIP builtin function variations for different device capabilities which are linked to the user’s device programs at runtime. For example, for HIP floating-point atomics the runtime chooses between an implementation that maps them to corresponding native functions via a SPIR-V extension or emulates them via atomic exchange operations.

\begin{figure*}
    \centering
    \includegraphics[scale=0.9]{figs/chipstar-rt-compile-n-link.pdf}
    \caption{The just-in-time compilation flow.}
    \label{fig:online-compilation}        
\end{figure*}


\subsection{Device Built-in Library}

The \textit{chipstar} device-side library implements the HIP math API, by using a combination of OpenCL C math builtins, OCML (part of ROCm-Device-Libs), and custom implementations.
A lot of the functions in the math API have an equivalent OpenCL builtin with adequate accuracy guarantees with a few exceptions that cannot be mapped directly, and thus require software based emulation such as floating-point atomics on some devices. The main challenge in terms of a fast yet portable implementation of the functions are due to differences in math accuracy requirements between CUDA/HIP and OpenCL: most of the standard math functions of CUDA are defined in higher accuracy than what the OpenCL standard requires.

Furthermore, CUDA/HIP defines a set of \textit{intrinsics}, which are faster yet less accurate versions of the standard functions. This exposes a further difficulty when aiming for a portable yet fast implementation: It heavily depends on the targeted platform what level of accuracy is achievable while still enabling execution time benefits. Since CUDA is inherently meant not to be cross-vendor portable, the intrinsics are defined only to match the CUDA microarchitecture in an optimal manner, which might not be the case for other devices. 

OpenCL covers the use case of accessing fast but inaccurate hardware operations by means of a relaxed mathematics flag that can be enabled at device program build time and with so called native built-in functions in the built-in kernel API. Unfortunately, neither of these are usable for implementing the CUDA intrinsics by default due to not guaranteeing enough accuracy: The relaxed math in OpenCL defines maximum rounding errors, but they are usually slightly less than what the CUDA intrinsics require. The OpenCL native built-in functions are even worse fit for this use since they guarantee nothing of the accuracy but leave it entirely up to the implementation. There is not even a possibility to query for the maximum error via a runtime API, but the accuracy must be discovered via trial-and-error or from documentation of the hardware vendor.
%\pj{TODO: Check the HIP statements of guaranteed accuracy.}

The ``correctness first'' principle requires implementing the functionality by default with guaranteed accurate enough arithmetics, which means to not receive any performance benefits of simplified implementations.
%
Correctness for the basic math functions would require software emulating them with added accuracy, of which performance impact would likely be too drastic to make \chipstar unusable for high performance workloads for which it is typically used. We chose a middle-ground where the basic math functions are implemented at the OpenCL accuracy by default and the intrinsics also utilize the default functions instead of the native functions, thus do not get any performance benefits from intrinsics. For the workloads we tested, this seemed to be a good enough solution.
We plan to optimize this aspect in the future via a new standard extension with a set of builtins that guarantee the CUDA accuracy requirements to the application programmer while enabling the targeted platform to optimize and implement them as efficiently as possible.

% https://github.com/CHIP-SPV/chip-spv/issues/222
% https://intel.github.io/llvm-docs/cuda/cuda-vs-opencl-math-builtin-precisions.html

An LLVM pass is responsible to lowering texture object API based texture functions to OpenCL image fetches. The pass analyses endpoints of the texture objects by following their use-def and def-use chains. If the pass sees that a texture object is coming from a kernel parameter and it is only used by texture fetch calls for the same dimensionality, it will replace the texture object parameter with image and sampler parameters and translates the texture fetch calls with OpenCL image fetch calls of matching dimensionality which consume the new kernel parameter. 

\subsection{OpenCL Extensions}

\chipstar compilation flow is built in a way that different advanced OpenCL features and extensions are not required from the target platform's driver or device unless the compiled input application specifically needs them. Although the minimal OpenCL 3.0 feature set plus coarse-grained SVM and SPIR-V consumption support covers a significant part of most commonly used CUDA and HIP features, some functionalities require or can be improved with various extensions to the OpenCL or SPIR-V specifications.

\begin{table*}[ht]
    \centering

    \begin{tabular}{|p{5 cm}|p{5cm}|p{5cm}|}
    \hline
\textbf{Extension name} & \textbf{CUDA/HIP feature(s)} & \textbf{Status} \\
    \hline
cl\_ext\_alive\_only\_barrier       & A special work-group barrier for barrier calls which might not be reached by work-items that have exited the kernel. As allowed by the CUDA's execution model. & To be proposed. \\
    \hline
cl\_ext\_cuda\_math     & Implement math functions and intrinsics with precision requirements that match CUDA's. To enable more optimized intrinsics. & To be proposed.  \\
    \hline
cl\_ext\_device\_side\_abort        & Implement \func{\_\_trap()} on the low-level runtime side. The current implementation requires compiler transformations. & Public.  \\
    \hline
cl\_ext\_extended\_device\_properties & \func{hipGetDeviceProperties()} can be used to query more device properties than the basic OpenCL device or platform query APIs support, this fills the gap. & To be proposed. \\
    \hline
cl\_ext\_relaxed\_printf\_address\_space &  CUDA's \func{printf()}behavior with non-constant address spaces. Currently handled with compiler transformations. & Public. \\
    \hline
cl\_intel\_unified\_shared\_memory & Used for optimized \func{cudaMalloc()} when available.  & Public. To promote to a general 'khr' extension. \\
   \hline
cl\_khr\_command\_buffer            & For optimized implementation of CUDA graph re-execution. & Public. SVM commands added in v0.9.4.  \\
    \hline
cl\_ext\_command\_buffer\_host\_data & For optimized implementation of CUDA graphs which transfer data between the host and the device. & To be proposed. \\
    \hline
cl\_ext\_command\_buffer\_host\_sync & For optimized implementation of CUDA graphs which synchronize with the host. & To be proposed. \\
    \hline
cl\_khr\_subgroup\_requirements & Used to fix the warp-size and force the desired thread id mapping when calling warp-level primitives that depend on the fixed warp size or the thread id ordering. & Private. Promotion to a general 'khr' extension proposed. \\
    \hline
cl\_khr\_fp64                       & If double precision floating point is used. & Public. \\
    \hline
cl\_khr\_global\_int32\_base\_atomics \newline
cl\_khr\_global\_int32\_extended\_atomics \newline
cl\_khr\_local\_int32\_base\_atomics  \newline
cl\_khr\_local\_int32\_extended\_atomics \newline
cl\_khr\_int64\_base\_atomics \newline
cl\_khr\_int64\_extended\_atomics & Atomic operations. & Public. \\
    \hline
cl\_khr\_subgroups                  & Warp-level synchronization with \func{\_\_syncwarp()}. & Public. \\
    \hline
cl\_khr\_subgroup\_ballot           & Warp-level ballot operations. & Public. \\
    \hline
cl\_khr\_subgroup\_shuffle          & Warp-level shuffle operations. & Public. \\
    \hline
    \end{tabular}
    \caption{OpenCL 3.0 standard extensions that \chipstar can use currently or will use in the future to implement CUDA/HIP features if the application uses them. Status describes the state of the extension at the time of this article's publication. }
    \label{table:extensions}
\end{table*}

In Table~\ref{table:extensions} we summarize the standard extensions \textit{chipstar} can utilize and which CUDA/HIP feature triggers their need. The extensions are in different stages in the Khronos Group standardization process, which is also noted in the table.\footnote{Note to reviewers: We will update the status for the final article version.}
Most of the extensions are straightforward and the brief description in the table should suffice to grasp their purpose. However, the handling of warp-level primitives calls for a bit more thorough explanation:
One of the execution model differences between CUDA to OpenCL is that CUDA presents a finer grained fixed size grouping of the threads (OpenCL work-items) than the blocks (work-groups) called a \textit{warp}. In the earlier CUDA versions, the threads in a warp could be assumed to execute in lock-step, implying that the enabled threads in the same warp would execute the same instruction. This implied that in some cases explicit synchronization could be omitted: In case of a usual read-modify-update case, the programmer could trust that the warp's threads all execute the read part before any of them proceed to the update part, enabling in-place-updates without explicit synchronization. However, with the later specification versions of CUDA relying on lock-step behavior in the program logic was deprecated~\cite{NVIDIAProgrammersManual}. ...

In addition to older CUDA programs potentially relying on the lock-step semantics to omit explicit synchronization, the fixed size warps (32 threads for NVIDIA and usually 64 threads in AMD devices) affect the execution semantics when executing warp-level functions that rely on the warp grouping and the mapping of the threads to the lanes of the warp.  Such primitives include the warp shuffles, which read data from a specific lane within the warp, and the explicit warp synchronization primitives.

%\pj{There could be a figure here with possible subgroup id mappings and how warps always map the threads in linear order.}
The OpenCL specification, on the other hand, doesn't have a warp concept, but the work-items are free to make progress in any order and grouping. It has a feature extension called ``subgroups'' which is used to implement the warp semantics in \chipstar when the kernel is detected to need it. However, in contrast to warps which have a specified form and content which allows the programmer to utilize them reliably, the basic subgroups of OpenCL are ``implementation-oriented''; they enable grouped execution in a manner that is simplest or most efficient for the driver and the hardware at hand. The sizes of the OpenCL subgroups are not fixed, but must be queried per kernel by the programmer in the basic extension. Also the way work-items are mapped to subgroup lanes (so they can be referred to when using cross-lane intrinsics) is also implementation-defined. To close the gap between subgroups and warps, a standard extension \textit{cl\_khr\_subgroup\_requirements} that \textit{forces} the subgroup size of the kernel to the desired size along with the linear id mapping was proposed.
%\kh{As Nvidia hardware generally has a warp size of 32 threads, what are the plans when OpenCL runtimes/devices can't support the size required by the CUDA application?}\pj{No plans so far. We could software-emulate different warp sizes using work-item loops/replication, but I'd rather not go there if not strictly necessary. Does RustiCL support 32-wide subgroups with NV/AMD targets?}\kh{As of today rusticl doesn't support forcing a subgroup size, but drivers are written in a way that it would be possible to do so once the appropriate CL extension is implemented. The AMD mesa driver also has a debug env variable to force 32-wide subgroups. The Intel driver chooses between 8/16/32 as it sees fit. So yeah, just a bit of code missing for it.}

% HIP doesn't support the new _sync-ones, so let's focus on it.
% Maybe also in the title of the paper.
%\pj{Pekka TODO: Non-uniform primitives.}

%\subsection{Compiling CUDA Applications Directly}
%\label{section:directCUDA}

%While the primary goal of \chipstar is to cover the CUDA/HIP APIs to the extent defined by the HIP programmer's manual, \chipstar implementation also supports a set of CUDA APIs directly. The ability to call CUDA APIs directly drops the need for source-to-source translations when porting originally CUDA applications to the platform. This is done by simply delegating the CUDA API calls to the HIP versions, similar to what HIP does with their CUDA mapping, but in reverse.\pj{to check}

%There has been legal controversy related to APIs how they are covered by the copyright laws in the past that has made legality of direct implementations of proprietary APIs unclear. This changed with the Supreme Court of the United States ruling of April 5, 2021 in the Google LLC vs. Oracle America, Inc. case, which stated that copying the Java API for use in the Android OS was considered ``fair use'' since it was done for compatibility purposes:

%\begin{quote}
%``Google’s copying of the Java SE API, which included only those
%lines of code that were needed to allow programmers to put their accrued talents to work in a new and transformative program, was a fair
%use of that material as a matter of law.''~\cite{JavaSupreme}
%\end{quote}

%Although no code was copied directly from the NVIDIA implementation, even if it was the case we believe
%our limited implementation of the CUDA API falls well within such fair use outlined in the ruling. However, since we, the \chipstar developers are engineers, not lawyers, we wanted to be extra careful that copyrights were not disrespected in any jurisdiction when adding support for direct CUDA API calls by using an implementation approach where only the programmer's manual was consulted for the API reference when implementing the CUDA compatibility headers to the \chipstar code base. 

%%%%%%%%%%%%%%%%%%%%%%%%%%%%%%%%%%%%%%%%%%%%%%%%%%%%%%%%%%%%%%%%%%%%%%%%%

%\section{Libraries}
\label{section:libraries}

The CUDA and ROCm software platform include a large set of useful libraries in addition to the general purpose program input support. These libraries include common routines such as BLAS (Basic Linear Algebra Subprograms) and Deep Neural Network (DNN) acceleration libraries. Implementing all of the libraries are out of scope of this work which focuses on providing portable and robust CUDA/HIP language support. However, we've already identified a set of key libraries and created example ports of them, enhancing portability of CUDA/HIP programs to Intel devices. The plan is to expand the set of supported libraries as required and demanded by the ported applications of interest. We discuss the currently supported libraries in the following and highlight their essential technical aspects.

\subsection{cuBLAS/hipBLAS for Intel GPUs}

BLAS is one of the core libraries in HPC. To implement the CUDA/HIP interfaces to BLAS for Intel hardware, we implemented an oneMKL backend to the hipBLAS performance library to run on Intel GPUs. Applications calling hipBLAS functions can thus run on Intel hardware without any code changes.
The oneMKL backend for hipBLAS uses the \chipstar interoperability feature with SYCL to invoke oneMKL’s SYCL functions. During the hipBLAS handle creation, the oneMKL backend extracts the native queue handle of a HIP stream using interop APIs from CHIP-SPV.  It uses the native queue handle to create a corresponding SYCL queue to execute the  oneMKL functions for the calls initiated from the hipBLAS handle. Since SYCL and \chipstar both support Level Zero and OpenCL runtimes, the oneMKL backend for hipBLAS also supports both: Users can switch between them using environment variables and the rest will work transparently in the hipBLAS library.

To assess the overheads of the implementation, we measured single precision GEMM function performance with 2048 x 2048 matrix size comparing hipBLAS, oneMKL SYCL and oneMKL OpenMP runtime and APIs on a pre-production Intel PVC system and measured execution time, comparing to running oneMKL directly from SYCL. The difference was negligible.

%%% PEKKA'S EDITING GOING HERE %%%

%Fig.~\ref{fig:hipBlas-rel-perf} shows the relative performance with oneMKL SYCL as the base.  We found that the performances are closely matched across all three runtimes.\pj{Not sure if it's worth adding the different iteration counts. If you look at the other results later, we took the best execution times over 100 iterations to give the roofline for perf. achievable. The perf. should vary only with a cold cache or if there's some other load in the system competing for the resources, both cases of which we should filter out this way.}

%Currently, oneMKL does not provide a pointer-mode API as defined in HIP and CUDA. To this end, we added a wrapper in oneMKL backend for hipBLAS to emulate pointer mode under oneMKL. In HIP device pointer mode, the oneMKL backend needs to copy scalar parameters, such as ‘alpha’ and ‘beta’ in GEMM between host and device memories.  Fig.~\ref{fig:hipBlas-host-vs-dev} shows more than 50\% performance drop with the workaround hence it is advised to avoid using device pointer mode in the current release.

%\begin{figure}
%     \centering
%     \begin{subfigure}[b]{0.5\textwidth}
%         \centering             
%         \includegraphics[width=1\textwidth]{figs/comparision_between_sycl_hip_openmp.pdf}
%         \caption{Average time taken for 5k gemm run with different runtime. Data is normalized with Sycl as 1.}
%         \label{fig:hipBlas-rel-perf}
%     \end{subfigure}     
%     \begin{subfigure}[b]{0.5\textwidth}
%         \centering             
%         \includegraphics[width=1\textwidth]{figs/comparision between_host_and_dev_ptr.pdf}
%         \caption{Average time taken for 5k gemm run with host and device pointers. Data is normalized with host pointer as 1.}
%         \label{fig:hipBlas-host-vs-dev}
%     \end{subfigure}
%     \caption{...\pj{todo}}
%\end{figure}

\subsection{cuDNN / MIOpen}

\pj{TODO: SYCL-DNN. Involve Codeplay with this paper?}

\subsection{rocPRIM for CUB compatibility}

\pj{TODO: test with cub}

\subsection{CUDA Graphs / MIGraphX}

\pj{TODO Michal: Describe mapping the Graph API to the command buffer API}


%%%%%%%%%%%%%%%%%%%%%%%%%%%%%%%%%%%%%%%%%%%%%%%%%%%%%%%%%%%%%%%%%%%%%%%%%

%\section{Debugging and Profiling Support}
\label{section:debuggingAndProfiling}

Thanks to using the open standard OpenCL as the portability layer, various debugging and profiling tooling options are available to use with little to no additional effort. The following discusses a set of tools that were successfully adopted and used through $chipstar$.

\subsection{Profiling Tools}

% VTune is rather Intel device-specific, uses its monitoring
% interfaces. We cannot provide
% much on top of OpenCL directly for
% wider device support since it doesn't
% have a monitoring API. The only
% data is the profiling time stamps
% from profiling queues.
%\subsubsection{VTune}

%\pj{TODO Paulius?}

\subsubsection{Tracing Heterogeneous APIs (THAPI)}

% https://github.com/argonne-lcf/THAPI

\pj{TODO Brice?}

\subsection{Debugging}

\pj{TODO: PoCL-CPU, GDB, Valgrind, debug info...}

%%%%%%%%%%%%%%%%%%%%%%%%%%%%%%%%%%%%%%%%%%%%%%%%%%%%%%%%%%%%%%%%%%%%%%%%%



%%%%%%%%%%%%%%%%%%%%%%%%%%%%%%%%%%%%%%%%%%%%%%%%%%%%%%%%%%%%%%%%%%%%%%%%%



%\subsection{libCEED}
%\pj{TODO: Paulius?}

%\subsection{Pytorch-HIP}

%\pj{TODO: Henry?}

%%%%%%%%%%%%%%%%%%%%%%%%%%%%%%%%%%%%%%%%%%%%%%%%%%%%%%%%%%%%%%%%%%%%%%%%%

\section{Evaluation}
\label{section:performance}

We evaluated the \chipstar speed on various OpenCL-capable CPU and GPU platforms using selections of benchmarks from the HeCbench collection~\cite{HeCbench}. All of
the results were produced using the \chipstar v1.2  release~\pj{TODO: rerun the benchmarks after the release is out}.

The HeCbench benchmark application selection criteria for each comparison was as follows:
\begin{enumerate}
    \item The application had the necessary API/language variations with an HIP version that could be built with \chipstar v1.2 and its ported libraries.
    \item For SYCL/CUDA comparisons, there must not have been significant identified performance-affecting structural or implementation differences between the SYCL/CUDA and HIP versions of the application. Some of the identified ``unfair differences'' were fixed and submitted back to the HeCbench repository \pj{TODO (Henry/Michal): a pull request of the changes.}.
    \item The application could verify its results and had to validate correctly on all platforms involved in the comparison.
    \item Applications that required hardware or OpenCL driver features that were missing or too limited on the platform were omitted.
    This mostly concerned the runs on embedded/integrated GPUs with limited memory or lack of double precision floating point support.
\end{enumerate}

\subsection{SYCL/DPC++}
\label{sec:SYCL-comparison}

For this evaluation we chose a subset of benchmarks included in the HeCbench suite and compared the performance of their HIP versions of the benchmarks against the SYCL versions. 
The SYCL versions were compiled with Intel's DPC++ shipped with the oneAPI v2024.X.X release\pj{TODO}. Thanks to the  OpenCL backend of DPC++, both versions of the applications could be executed using the same OpenCL driver on the same GPUs, nicely isolating
the differences between the tested software stacks to the runtime and the
LLVM IR level device code compiler optimizations. The results of this comparison are
shown in Fig.~\ref{fig:intel-arc-sycl-hip} for Intel ARC A760 discrete GPU and Fig.~\ref{fig:intel-i9-igpu-chipstar-vs-sycl} for the integrated GPU of Intel's 12900 i9.

\begin{figure}
      \includegraphics[scale=0.5]{figs/hecbench_intel_arc750_hip_vs_sycl.pdf}
      \caption{HIP/\chipstar speed (inverse of execution time) normalized to SYCL/DPC++ on Intel ARC A750. \pj{TODO (Michal):} Total of N benchmarks with geom. mean of X.}
      % plot.py -r -l '#b7cce9' -m 0.8 -s seaborn-v0_8-pastel -t "HeCBench, Intel Arc750, HIP vs SYCL speedup" -c test_20_strict_hip_oclBE_arc_after.csv -b test_20_strict_sycl_oclBE_arc_after.csv
      \label{fig:intel-arc-sycl-hip}
\end{figure}

\begin{figure}
      \includegraphics[scale=0.5]{figs/hecbench_i9_12900_igpu_hip_vs_sycl.pdf}
      \caption{HIP/\chipstar speed (inverse of execution time) normalized to SYCL/DPC++ on Intel i9 12900 iGPU. Higher is better. \pj{TODO Michal:} Total of N benchmarks with geom. mean of X.}
      \label{fig:intel-i9-igpu-chipstar-vs-sycl}
\end{figure}

Interestingly, although both the DPC++ and \chipstar use very similar tools and components in their runtime (OpenCL) and compilation flow (Clang/LLVM and SPIR-V as the device intermediate format), we measured significant variance in the execution performance to both directions.
We analyzed the cases with the most dramatic differences and identified various explanations:
Many of the benchmarks executed very short kernel commands, making the benchmark actually mostly measure the host API call execution speed.
For example, the ``overlay'' benchmark could be sped up significantly by switching off the profiling command queue feature. 
% that was accidentally left on by default. \hl{not accidental. Profiling is used for hipEventElapsedTime.}
In some cases the device built-ins were more optimized in \chipstar than in DPC++, in some cases it was the opposite.
For example, when we compared the \chipstar and DPC++ LLVM IRs of the device code for the ``nlll'' benchmark, we found that only \chipstar performed the if-conversion optimization that converts some of the very small branches to conditional moves, which provided significant benefits.

%%%%%%%%%%%%%%%%%%%%%%%%%%%%%%%%%%%%%%%%%%%%%%%%%%%%%%%%%%%%%%%%%%%%%%%%%

\subsection{CUDA on NVIDIA CUDA SDK}

Execution speed of CUDA programs compiled and ran using the NVIDIA's proprietary CUDA SDK is interesting to compare against the portable and open alternative provided by \chipstar. In this experiment we first compiled CUDA versions of the applications using CUDA SDK \pj{TODO (Henry): version} to get a baseline. The same benchmark cases (the HIP versions) were then compiled using \chipstar to the portable fat binary that uses OpenCL as the portability layer which was run with multiple alternative OpenCL options. The OpenCL options included ``PoCL-CUDA'', which is a work-in-progress OpenCL implemented using the upstream LLVM's PTX backend and libCUDA. In this case two memory allocation API options were measured for curiosity: the Coarse Grain SVM, and the buffer device address (BDA) extension. We also compared to rusticl/zink, an OpenCL implementation on top of Vulkan which implements (as of this writing) only the BDA alternative. The execution time was measured on an NVIDIA RTX 3060 GPU with the results shown in Fig.~\ref{fig:rtx3060-cudasdk-vs-pocl-vs-rusticl}.

The difference between SVN and BDA in ``lfib4'' is caused by the SVM implementation of PoCL-CUDA which uses managed allocation with virtual memory pages that are not created until accessed the first time. The slow down is due to the page faults in the kernel when it writes results to such an allocation. BDA, on the contrary, doesn't allocate virtual memory pages, thus doesn't suffer from this. Other bechmarks encounter this situation as well, but only for the first kernel execution.

\pj{TODO (Karol): Let's rerun with the latest rusticl to see if ``vanGenuchten'' is now faster.}

%Henry>The ``vanGenuchten'' explanation: Not sure what happens truly. The SPIR-V binary that gets passed to the Vulkan driver has pow() function inlined and the kernel outputs a result with more error in it (but still satisfies verification). Perhaps, rusticl/zink uses their own pow() implementation that is a bit relaxed in precision and faster? -Henry

\begin{figure*}[tb]
      \includegraphics[width=0.9\linewidth]{figs/hecbench_rtx3060_cudasdk_vs_pocl_n_rusticl.pdf}
      \caption{Speed of HIP/chipStar on multiple OpenCL runtime options normalized to running the CUDA versions compiled with the proprietary CUDA SDK. SVM = Coarse Grain SVM and BDA = the buffer device address extension. \pj{TODO Henry:} Total of N benchmarks with geom. mean of X.}
      \label{fig:rtx3060-cudasdk-vs-pocl-vs-rusticl}
\end{figure*}

\subsection{HIP on AMD ROCm}

Since \chipstar can be viewed as a more portable implementation of HIP, it is interesting to compare its speed against the original HIP implementation from AMD. For this comparison, we utilized the HIP compiler and runtime from the AMD ROCm package version \pj{Henry TODO} as a baseline to compile and execute a set of HeCbench HIP benchmarks on an AMD Radeon Pro VII GPU. To run the \chipstar fat binaries on the same GPU, we used the rusticl OpenCL implementation on the radeonsi driver. AMD's OpenCL implementation doesn't support SPIR-V input at the time of this writing, preventing its use in this comparison for running the \chipstar binaries.

In the numbers shown in Fig.~\ref{fig:radeonprovii_rocm_vs_rusticl}, an interesting anomaly to look closer at is `pnpoly' which is more than 2.5x faster on chipStar. The reason is that ROCm is slightly faster for tile sizes smaller than 32 and for larger ones notably slower. Unfortunately for ROCm, the benchmark tracks times for the largest tile which is happends to be the slowest one on ROCm. 

\begin{figure}[tb]
      \includegraphics[scale=0.5]{figs/hecbench_radeonprovii_rocm_vs_rusticl.pdf}
      \caption{Speed on AMD Radeon Pro VII through ROCm and chipStar/rusticl/radeonsi, normalized to ROCm. Higher is better for rusticl. \pj{TODO (Henry):} Total of N benchmarks with geom. mean of X.}
      \label{fig:radeonprovii_rocm_vs_rusticl}
\end{figure}


\subsection{Offloading to Integrated GPUs}

To test the extent of portability of the runtime API layer based on the OpenCL standard, we compiled and executed sets of HeCbench applications on various platforms which included both a CPU and a GPU with capable enough OpenCL support to execute the same compiled chipStar fat binary on both devices, enabling interesting CPU-to-GPU offload speedup comparisons. The platforms and their results are presented in the following.
%The HeCbench applications were chosen using the same selection criteria as described in Section~\ref{sec:SYCL-comparison}). 

\paragraph{RISC-V CPU \& PowerVR GPU} In this experiment we utilized the VisionFive2 single board computer for building and running the benchmarks. PoCL~\cite{PoCL} was used for running the benchmarks on the CPU and the proprietary OpenCL driver from Imagination Technologies was used for the GPU.  The results are visualized in Fig.~\ref{fig:intel-visionfive2-gpu-cpu}. The lower performance (0.74 geom mean) of the PowerVR GPU vs RISC-V CPU can be explained by the GPU having much less on-chip resources than most benchmarks could utilize. The CPU (JH7110) has a 32KB L1 i/dcache and a 2MB L2 cache, 4 scalar cores running at 1.5 GHz (no vector support), while the GPU has only 1 compute unit (CU) running at 600 MHz and 4KB of local memory per CU. The GPU also has a native workgroup size of only 32 (subgroup size 16), while most HeCBench benchmarks use a workgroup size ranging from 128 to 1024, leading to additional thread context switches. Furthermore, the GPU's limited local memory is used also to store images, samplers, the OpenCL constant data and pointers to global memory - in addition to the shared data of the application kernels~\cite{PowerVRPerfGuide}. 
      %{there are some benchmark outliers that are >10x slower on GPU vs CPU.
%        asmooth: allocates a local memory array of 1024 floats. This is equal to the on-chip local memory size (4KB) while the local memory is used also in other ways (see prev paragraph), so
%        this most likely results in spilling into global memory.
%        all-pairs-distance: uses atomicAdd and memory access with stride
%        }

\begin{figure}[tb]
      %\includesvg[width=\textwidth]{figs/hecbench_visionfive2_gpu_vs_cpu}
      \includegraphics[scale=0.5]{figs/hecbench_visionfive2_gpu_vs_cpu.pdf}
      \caption{Speed on PowerVR GPU normalized to the RISC-V CPU. \pj{TODO Michal:} Total of N benchmarks with geom. mean of X.}
      \label{fig:intel-visionfive2-gpu-cpu}
\end{figure}

\paragraph{Intel i9 12900} This platform has a 16-core x86-64 CPU and an Intel UHD Graphics 770 integrated GPU. OpenCL on the CPU and the GPU were supported by Intel's OpenCL drivers  \pj{TODO (Michal): driver versions}. The results are shown in Fig.~\ref{fig:intel-i9-cpu-vs-igpu}. In this platform, the offload benefits are more visible thanks to the more powerful GPU.

\begin{figure}[tb]
      \includegraphics[scale=0.5]{figs/hecbench_i9_12900_igpu_vs_cpu.pdf}
      \caption{Speed on Intel i9 12900 iGPU speed normalized to CPU. \pj{TODO Michal:} Total of N benchmarks with geom. mean of X.}
      \label{fig:intel-i9-cpu-vs-igpu}
\end{figure}

\paragraph{ARM Cortex A53+A73 CPU \& Mali G52 GPU} \pj{TODO (Michal): driver info.}  For the CPU, PoCL~\cite{PoCL} was used as the OpenCL driver while the GPU was supported by the ARM's proprietary OpenCL driver. The results are shown in Fig.~\ref{fig:mali-vs-cortex}. A lot of applications were dropped because of limited memory in the GPU and lack of double precision floating point support \pj{TODO (Michal): pls confirm}. However, various benchmarks showed significant benefits from CPU to GPU offloading, as expected.

\begin{figure}[tb]
      \includegraphics[scale=0.5]{figs/hecbench_malig52_vs_cortexa53a73.pdf}
      \caption{Speed on ARM Mali GPU normalized to the ARM Cortex CPU. \pj{TODO Michal:} Total of N benchmarks with geom. mean of X.}
      \label{fig:mali-vs-cortex}
\end{figure}

\section{Real-World Application Case Study: GAMESS}
\label{section:applications}

In order to further test \chipstar in practice, we ported a %set of 
complex HIP/CUDA-based HPC application and its dependency library using \chipstar. The test environment for the experiment was the Aurora supercomputer utilizing Intel Datacenter Intel® Data Center GPU Max Series (referred to from here on as PVCs, as in Ponte Vecchio) as the accelerator part~\cite{aurora}.
%The porting examples are described in the following subsections with the performance evaluations
%presented in the next section.
%The applications and which features or libraries of \chipstar they use are shown in Table~\ref{tab:applications}.

\subsection{GPU Integral Library (GAMESS-EXESS)}

General Atomic and Molecular Electronic Structure System (GAMESS~\cite{gamess,gamess2}) is a quantum chemistry software package which implements many electronic structure methods. 
The code base is primarily in Fortran 77/90 with some C/C++ and a CUDA library. Recently a new GPU version of the Hartree-Fock (HF) and RI-MP2 methods were implemented in CUDA which scales to 4096 nodes on Summit, an Nvidia V100-based supercomputer \cite{gamess_cuda1, gamess_cuda2, summit}.
In this porting case we focused on the Hartee-Fock (HF) algorithm used by a CUDA library in GAMESS described in \cite{gamess_cuda1}, which has been ported to HIP. The HF method is a common quantum chemistry method which is often the starting point for other higher-accuracy methods.  The HF method determines the molecular energy of a system by solving a set of non-linear eigenvalue equations iteratively.  It primarily involves the computation of $N^4$ two electron integrals (where $N$ is a measure of molecular system size) as well as matrix contractions of the two electron integrals once they are formed.
% Colleen> I'm ok to remove the discussion of the basis functions if the code works for all of them :) The two electron integrals are grouped into different classes, depending on the angular momentum of the basis functions used. The basis functions here are $s-$ ,$p-$, and $d-$, where $s$ is least complex and $d$ is the most complex.

The two electron integrals are implemented as HIP/CUDA kernels which were optimized for Nvidia GPUs and total over 20,000 lines of HIP/CUDA kernel code.
%From the non-basic features of HIP supported by \chipstar, the kernels use shared memory with \func{\_\_syncthreads()} calls to ensure copying values from global memory to shared memory completed for the threadblock before using it.\pj{Colleen: does it use any other "special" CUDA/HIP features on the host side?}
%

\subsection{hipBLAS and hipSOLVER}

Since the application uses ROCm software platform libraries hipBLAS and hipSOLVER, they needed to be ported as well. The required interfaces of these libraries were implemented for Intel hardware by using oneMKL as a backend. %Applications calling hipBLAS functions can thus run on Intel hardware without any code changes. 
For this porting case, a SYCL interoperatibility feature was added to \chipstar which was used to invoke oneMKL’s SYCL functions efficiently.

\pj{TODO (Colleen/?): A brief description how hipSOLVER was implemented on MKL?}

%hipBLAS and hipSOLVER calls are used to form intermediates. The main hipBLAS calls are hipblasDscal, hipblasDgemm, hipblasDcopy, hipblasDaxpy, hipblasDdot, hipblasDgemv, hipblasDgeam, and the main hipSOLVER call is hipsolverDsyevd.

%\pj{From these it would be good to summarize for example, that they cover the key blas functionality.} These are used at each iteration of the HF algorithm to (among other things) diagonalize the Fock matrix and construct the density matrix, which are key blas functionalities in the HF algorithm.

\subsection{Porting Notes}

In terms of functionality, the HF code compiles and was verified to run correctly with \chipstar on PVCs. The porting effort was relatively low, with one exception due to a small but significant specification difference in CUDA vs. OpenCL related to kernel thread synchronization: In CUDA group barriers are not counting in exited threads, meaning that there can be early returns from the kernel by a subset of the threads after which it is still legal to perform barrier synchronization with the remaining subset -- the exited threads are just not counted in. In OpenCL this case is undefined behavior and in many implementations can lead to a deadlock. To tackle this gap, an OpenCL extension adding a group barrier with similar semantics would be needed (see \textit{cl\_ext\_alive\_only\_barrier} in Table~\ref{table:extensions}).

\subsection{Performance}

%The GAMESS port was executed with the PVC component of the Aurora installation. 
The performance of GAMESS was measured by compiling and running the same HIP source code on a PVC through \chipstar and on an Nvidia A100 as well as an AMD MI250 using ROCm 6.0.0. 
%\pj{Is it running the HIP code via AMD HIP wrapper or the CUDA version?} \cb{via the AMD HIP wrapper -- everything is using the same HIP code (I put all the runscripts and output on the github https://github.com/colleeneb/gamess\_libcchem\_hip for the record. so it was https://github.com/colleeneb/gamess\_libcchem\_hip/blob/hip\_dev\_for\_intel/hip\_nvidiaa100.sh)} 
% Not sure if this context is needed here: --PJ
%The AMD MI250 is part of the JLSE cluster at ANL and is a Supermicro AS-4124GQ-TNMI composed of 2 AMD EPYC 7713 64c (Milan) CPUs and 4 AMD Instinct MI250~\cite{JLSE}. The Nvidia A100 is also part of the JLSE cluster and is composed of a AMD 7532 and 1 Nvidia A100 with 40GB and PCIe 4.0. 
The test run computed the HF energy of a cluster of 150 water molecules with a STO-3G basis set. The results are displayed in Table~\ref{table:gamess_perf}.

\begin{table*}[ht]
\centering
\begin{tabular}{l|l|l|l|l}
                   & Nvidia A100 & AMD MI250 & Intel PVC      & Intel PVC \\
                   &               &             & (OpenCL backend) & (Level Zero backend) \\ \hline
Total SCF time (s) &    1.998      &    26.09    & 4.9  & 3.6  
\end{tabular}
    \caption{Comparison of GPU integral code performance across Intel, AMD, and Nvidia}
    \label{table:gamess_perf}
\end{table*}

Table~\ref{table:gamess_perf} shows that the execution time on the Nvidia A100 is shortest and on AMD MI250 the longest. The energy calculation of the GAMESS simulation can be split into two main parts: Fock build time (time for computation of electron repulsion integrals and Fock matrix), and DIIS time (time for solving a set of linear equations). The Fock build time is primarily hand-written HIP kernels. The DIIS time is primarily BLAS and LAPACK calls, including calls to the hipSOLVER function hipsolverDsyevd. Table~\ref{table:gamess_breakdown} shows the timing breakdown for each architecture. Compared to the A100 times, the execution times on Intel PVC with the Level Zero backend are 2-5x slower. Although the Fock build time for the Intel PVC with the OpenCL backend is only 3x slower than the A100 time, the times are 25-53x slower for the DIIS and hipsolverDsyevd times. Similarly, although the Fock build time for the MI250 time is only 1.4x slower than the A100 time, and the DIIS time without hipsolverDsyevd is 0.7x the A100 time, the hipsolverDsyevd time is 109x the A100 time.\pj{TODO: Still need to identify why CL is so much slower. But is the LZ number good or not - how is A100 vs. PVC in terms of peak perf?}

% Colleen needs to update this section

\begin{table*}[ht]
\centering
\begin{tabular}{l|l|l|l|l}
 &
  Fock time &
  DIIS time without hipsolverDsyevd &
  hipsolverDsyevd time &
  Remainder \\
 &
  (ratio over A100 time) &
  (ratio over A100 time) &
  (ratio over A100 time) &
  (ratio over A100 time) \\ \hline
Nvidia   A100       & 1.46 (1x) & 0.183 (1x) & 0.230 (1x) & 0.128 (1x) \\
AMD MI250           & 2.03 (1.4x)  & 0.12 (0.7x)  &  25.10 (109x) & 0.09 (0.7x) \\
Intel PVC   (OpenCL) & 4.29 (2.9x)  & 4.73 (25.8x) & 12.34 (53.7x) & 0.54 (4.2x) \\
Intel PVC   (LevelZero)    & 3.11 (2.1x) & 0.954 (5.2x) & 0.984 (4.3x) & 0.262 (2x)
\end{tabular}
    \caption{Timing breakdown of the GPU integral HIP code across Intel, AMD, and Nvidia}
    \label{table:gamess_breakdown}
\end{table*}


\section{Related Work}
\label{section:relatedWork}

The origin of \textit{chipstar} is on the HIPCL~\cite{HIPCL} prototype which first tested the concept of compiling HIP programs to fat binaries relying on OpenCL and SPIR-V. The \chipstar tool described in this article is a result of an almost a complete rewrite of the HIPCL code base and over approximately three years of continuous development work by multiple partners and HPC users. The HIPCL code base was initially forked to a separate code base to utilize the Level Zero~\cite{l0} low level API directly (HIPLZ~\cite{HIPLZ}) after which the OpenCL backend of HIPCL and the Level Zero backend of HIPLZ were merged to the same code base discussed in this article.
A large number of missing essential features have been implemented since the initial prototypes were published. This article thus significantly expands upon the original poster abstract that introduced the early-prototype-stage HIPCL and now presents a much more mature software stack usable for a wider range of real-world workloads.
%The direct Level Zero access is used as an additional backend for comparison purposes in this article, with the primary focus being on the OpenCL backend.

%However, at the time of this writing, the recommended path from CUDA/HIP to Level Zero goes through the OpenCL backend and PoCL's~\cite{poclIJPP} Level Zero backend since the OpenCL code path has matured longer and is somewhat more robust.

When comparing \chipstar to other HIP implementations, obviously the original ROCm, the AMD's official GPU software platform~\cite{ROCm}, is the baseline. ROCm consists of the general purpose programming API compilation and runtime support for HIP, and a set of libraries that support different degrees of compatibility with the CUDA platform. \chipstar is not a new backend in addition to the AMD GPU and NVIDIA GPU backends provided by the AMD's offering, but has an important technical difference: \chipstar aims to offer runtime portability by its open standard based fat binary, removing the need to recompile the input software per target vendor platform, which is the case with ROCm.
%\pj{Brice/Paulius: Can you check that this is (still) true?}

%HIP is very close to CUDA, and in fact AMD provides a source-to-source translation tool called HIPify that can automate the porting process. Interestingly, although heavily based on the NVIDIA-driven CUDA, AMD now promotes HIP as the primary C++ programming API for their GPU platforms. Since AMD GPUs have increased their market share and received major design wins in large HPC installations, HIP as such has risen in importance as an application-facing interface.

SYCLomatic~\cite{SYCLomatic} is a tool contributed by Intel Corporation for converting CUDA sources to the cross-vendor open standard SYCL~\cite{SYCL}. Similar to AMD's HIPify, but in contrast to \chipstar which aims for source-level compatibility, SYCLomatic is a source-to-source conversion tool, which has its good and bad sides. The most apparent implication of relying on source-to-source conversion is more about maintenance aspects than technical ones; it neccessitates the further development of the converted application to proceed using the SYCL API instead or in addition to CUDA. The main drawback is that in reality many code bases are difficult or impossible to convert solely to SYCL without having the CUDA version as a backup due to legacy, risk-management or technical reasons. The main benefit is that SYCL is an open standard, in constrast to CUDA, enabling more fair competition ground between hardware vendors. Thus, being able to target many platforms from a fat binary compiled from the unmodified CUDA/HIP source code base using a \chipstar-style open platform approach can have its benefits. Furthermore, since \chipstar is not a linkage-time or binary translation solution, but requires recompilation, it coincidentally also encourages utilizing and further developing the cross-vendor ecosystem APIs it relies upon. Furthermore, as of this writing SYCLomatic supports only CUDA, not HIP, while HIP has an increasing number of new applications implemented directly using it.

% https://github.com/vosen/ZLUDA
ZLUDA~\cite{ZLUDA} is a tool for running unmodified CUDA binaries on AMD GPUs. It works by reimplementing the \cuda driver API, and converting NVIDIA PTX~\cite{ptx} to the vendor-specific IRs. Since it is a ``drop-in solution'' that works at program loading/linkage time, it can execute unmodified CUDA fat binaries, which is very comfortable to the end users as it doesn't require access to the source code of the application. While we see ZLUDA as an excellent tool, it requires reverse engineering CUDA SDK's binary interfaces and keeping up-to-date with the NVIDIA PTX as it evolves. We believe, in the longer term, especially as more of the missing extensions we describe are adopted by OpenCL implementations, \chipstar can provide a more robust solution. 
%Of course only time will tell how well this turns out to be the case. 
In addition, ZLUDA also doesn't support HIP as an input and now only targets AMD GPUs, whereas a key goal of \chipstar is extensive cross-vendor portability.

%However, its developed has stalled and it only supports a limited subset of applications and only on the Intel devices supported by the Level Zero API. ZLUDA author claims in their web page that they can achieve performance benefits when running straight on top of the lower level \lz instead of the somewhat higher level OpenCL. Since \chipstar supports both, we were able to measure this difference accurately, and found it to be negligible\pj{to do actually}.
%A key benefit of skipping a cross-vendor standardized layer is that PTX has instructions which map directly to the Intel GPU instructions which are not exposed in OpenCL C.
%Although \chipstar uses the OpenCL runtime for portability, it targets SPIR-V instead of OpenCL C as the device-side programming language, thus this drawback does not appear with it. The potential overhead is first passing through LLVM IR, which might lose beneficial information, but that also is found not to be an issue according to the measurements presented in Section~\ref{TODO} \pj{to do actually}.

MCUDA~\cite{MCUDA} is the oldest tool we found for porting CUDA programs to non-NVIDIA platforms. MCUDA does source-to-source translation of kernels in a fashion that the translated kernels can execute efficiently on CPUs on a single CPU thread while respecting the barrier synchronization. In the case of \chipstar, since it uses OpenCL as its portability layer, it can similarly target also vectorized CPU execution through CPU-targeting OpenCL implementations such as the Intel OpenCL CPU driver and PoCL's CPU drivers~\cite{PoCL}. Both of them are capable of vectorizing work-items (CUDA/HIP threads) inside work-groups, which translates to implicit autovectorization of CUDA/HIP kernels across CUDA threads and provide the benefits of CPU execution such as easier kernel debugging.

Swan~\cite{Swan} is another early source-to-source tool for CUDA porting. It generates OpenCL code from CUDA, providing similar level of portability as \chipstar does. Another similar tool, CU2CL~\cite{CU2CL} was published in the same year as \cite{Swan}. Neither Swan nor CU2CL are maintained any longer.  In comparison to \chipstar, the main technical differences to these tools are that \chipstar utilizes the latest version of the OpenCL standard to support the newer CUDA/HIP features, uses SPIR-V as the intermediate language (no need to generate textual OpenCL C with its limitations) and it doesn't suffer from problems related to source-to-source translations as \chipstar provides source-level compatibility.

The closest comparable CUDA porting tool we could find is CUDA-on-CL~\cite{CUDAonCL}. Like \chipstar, it similarly compiles CUDA programs using Clang/LLVM-based compiler chain to binaries which then execute on OpenCL platforms. However, similarly to Swan and CU2L, it compiles device kernels to OpenCL C whereas \chipstar uses SPIR-V as the portable binary format. Other technical differences in \chipstar are related to the use of modern OpenCL standard features to implement some of the features of CUDA. These include using SVM to implement raw pointers and implementing warp-level primitives such as shuffles using the subgroup features. 




%%%%%%%%%%%%%%%%%%%%%%%%%%%%%%%%%%%%%%%%%%%%%%%%%%%%%%%%%%%%%%%%%%%%%%%%%

% This doesn't fit as the page limit is only 12pp. If we resubmit to another journal, it's interesting info to add and can be easily copy-pasted from the report:

%\section{Supporting Newer CUDA Features}
%\label{section:directCUDA}

%HIP is a subset of CUDA features, roughly at version 8~\pj{check this}. Thus, it doesn't include support some of the newer features which can utilize some of the more advanced capabilities of the NVIDIA GPU platforms. Some of these features are difficult to implement efficiently on other vendors' GPU features, and since GPU offloading is primarily done with performance improvements in mind, a functional, but inefficient implementation is less interesting.

%However, for the purpose of completeness, it is interesting to highlight some of the more useful newer features in later CUDA versions, and consider implementation strategies for future work.

%\pj{Discuss features specific to CUDA, from the doc I wrote in Parmance.}

% TODO: the OpenCL extensions identified and proposed.

%%%%%%%%%%%%%%%%%%%%%%%%%%%%%%%%%%%%%%%%%%%%%%%%%%%%%%%%%%%%%%%%%%%%%%%%%

\section{Conclusions}
\label{section:conclusions}

In this article, we presented \chipstar, a compilation flow and a runtime for CUDA/HIP applications using open cross-vendor supported standards. In comparison to previous tools, \chipstar's goal is on source-level compatiblity which we believe has longer-term robustness benefits in comparison to binary translation. Whereas relying on standardized APIs has its drawbacks in shorter term due to the cross-vendor ``democratic concensus'' requirement, in terms of building open and fair heterogeneous computing ecosystem of the future, we consider using and expanding open standards aiming at the portability layer level has also significant far-reaching value.

The performance of HIP benchmarks using \chipstar was shown to be on par or surpass their SYCL versions when using a compiler/runtime with similar components (OpenCL and SPIR-V). An example of the source-level compatibilty was provided with GAMESS, a code base with a significant number of kernel code lines. This demonstrates that \chipstar is a useful option for applications that are not feasible to port to more cross-vendor supported open standard input APIs such as SYCL or OpenMP.

In the future, as \chipstar focuses on the HIP/CUDA API/language support, the main aspect that requires more engineering work is to expand the directly supported set of core libraries in the CUDA and HIP ecosystems to cover more real-world applications. On the HIP/CUDA core front, we aim to focus on the more advanced recent features such as collaborative groups as well as the standard extensions to bridge the remaining gaps between CUDA/HIP and OpenCL/SPIR-V.

\bibliography{IEEEabrv,chipstar}
\bibliographystyle{IEEEtran}
\newpage

\pj{Notes:
\begin{itemize}
    \item I merged the hipBLAS description to GAMESS for now.
 %   \item I removed the debugging/profiling section. Not enough sensible content. We should state somewhere that thanks to using OpenCL we can use any tools that can profile OpenCL. Using the CPU target helps in debugging thanks to running gdb.
 %   \item Can we add more porting case studies which demonstrate something different than GAMESS? CP2K? Is the pruned-down Pytorch-HIP sensible or too pruned down? Exabiome?
    \item I removed the Libraries section as there was too little technical content. cuBLAS is too brief and highlights the negative fact that we support only a small subset of the NVIDIA or AMD ecosystem libraries.
\end{itemize}}

\pj{todo:
  \begin{itemize}
    \item We should add more pictures. It's too much of text without pics. Anyone any idea of what and where?
    \item The article still reads a bit as a ``technical report'' instead of a ``scientific paper''. For more "scientific content" we should add a bit more complex technical content of some more trickier aspects, e.g., about the Graph implementation using command buffers. Ideas?
\end{itemize}}

\end{document}


\bibliographystyle{SageV}
%\bibliography{chipstar}
\documentclass[lettersize,journal]{IEEEtran}
\usepackage{amsmath,amsfonts}
\usepackage{algorithmic}
\usepackage{algorithm}
\usepackage{array}
%\usepackage[caption=false,font=normalsize,labelfont=sf,textfont=sf]{subfig}
\usepackage{textcomp}
\usepackage{stfloats}
\usepackage{url}
\usepackage{verbatim}
\usepackage{graphicx}
\usepackage{cite}
\usepackage{color}
\usepackage{amssymb}
\usepackage{xspace}
\usepackage{textcomp}
\usepackage{svg}
%\usepackage{subfigure}
\usepackage{caption}
\usepackage{subcaption}
%\usepackage{draftwatermark}
%\SetWatermarkText{DRAFT}
%\SetWatermarkScale{1}

% https://tex.stackexchange.com/questions/326897/vertical-alignment-of-a-turned-cell
\usepackage{rotating}
\usepackage{array,makecell,multirow}

\usepackage{ifthen}
\newboolean{showcomments}
\setboolean{showcomments}{true}
\ifthenelse{\boolean{showcomments}}
{ \newcommand{\mynote}[3]{
     \fbox{\bfseries\sffamily\scriptsize#1}
        {\small$\blacktriangleright$\textsf{\emph{\color{#3}{#2}}}$\blacktriangleleft$}}
  \newcommand{\newtext}[1]{{\color{orange}{#1}}}}
{ \newcommand{\mynote}[3]{}
  \newcommand{\newtext}[1]{#1}}

% Please use a named note with this macro to comment the text:
\newcommand{\pj}[1]{ \mynote{PJ}{#1}{blue} }
\newcommand{\bv}[1]{ \mynote{BV}{#1}{green} }
\newcommand{\mb}[1]{ \mynote{MB}{#1}{cyan} }
\newcommand{\cb}[1]{ \mynote{CB}{#1}{magenta} }
\newcommand{\pv}[1]{ \mynote{PV}{#1}{yellow} }
\newcommand{\ba}[1]{ \mynote{BA}{#1}{brown} }
\newcommand{\kh}[1]{ \mynote{KH}{#1}{red} }
\newcommand{\hl}[1]{ \mynote{HL}{#1}{orange} }

\newcommand{\hiplz}{\texttt{HIPLZ}\xspace}
\newcommand{\hipcl}{\texttt{HIPCL}\xspace}
\newcommand{\hip}{\texttt{HIP}\xspace}
\newcommand{\opencl}{\texttt{OpenCL}\xspace}
\newcommand{\lz}{\texttt{L0}\xspace}
\newcommand{\sycl}{\texttt{SYCL}\xspace}
\newcommand{\cuda}{\texttt{CUDA}\xspace}
\newcommand{\chipstar}{\textit{chipStar}\xspace}
\newcommand{\func}[1]{$#1$\xspace}
\newcommand{\type}[1]{$#1$\xspace}

\hyphenation{op-tical net-works semi-conduc-tor IEEE-Xplore}

% IEEE policy on preprints seems to be reasonable:
% https://journals.ieeeauthorcenter.ieee.org/become-an-ieee-journal-author/publishing-ethics/guidelines-and-policies/submission-and-peer-review-policies/#electronic-reprints

% https://journals.ieeeauthorcenter.ieee.org/submit-your-article-for-peer-review/the-ieee-article-submission-process/
% TPDS manuscript types and submission length guidelines are described below. All page limits include references and author biographies. For regular papers, pages in excess of these limits after final layout of the accepted manuscript is complete are subject to Mandatory Overlength Page Charges (MOPC). Note: All supplemental material must be submitted as separate files and must not be included within the same PDF file as the main paper submission. There is no page limit on supplemental files. 

% Regular paper – 12 double column pages (Submissions may be up to 18 pages in length, subject to MOPC. All regular paper page limits include references and author biographies.)

\begin{document}

\title{\chipstar: Making HIP/CUDA Programs Cross-Vendor Portable by Relying on Open Standards}

%\author{pekka.jaaskelainen }
%\date{March 2023}

\author{Pekka Jääskeläinen, Henry Linjamäki, Michal Babej, Peng Tu, Sarkar Sarbojit, Ben Ashbaugh, Colleen Bertoni, Kevin Harms, Paulius Velesko, Philip C. Roth, Rahulkumar Gaytri, Jisheng Zhao, Karol Herbst, Brice Videau
        % <-this % stops a space
\thanks{Pekka Jääskeläinen, Henry Linjamäki, Michal Babej, Peng Tu, Sarbojit Sarkar and Ben Ashbaugh(?) are with Intel Corporation. \textit{Corresponding author: Pekka Jääskeläinen, email: pekka.jaaskelainen@intel.com}.}
\thanks{Pekka Jääskeläinen is also with Tampere University, Finland. }
\thanks{Paulius Velesko is with Pagan LC.}
\thanks{Philip C. Roth is with Oak Ridge National Laboratory, ... }
\thanks{Rahulkumar Gaytri is with National Energy Research Scientific Computing Center, ...}
\thanks{Jisheng Zhao is with Georgia Institute of Technology, Atlanta, Georgia.}
\thanks{Karol Herbst is with Red Hat, Inc.}
\thanks{Brice Videau, Colleen Bertoni and Kevin Harms are with Argonne National Laboratory, ...}
\thanks{\pj{The authors are not in any particular order. I put myself as the 1st author as I'm leading the writing, and then I ordered the co-authors according to their affil.}}
%\thanks{This paper was produced by the IEEE Publication Technology Group. They are in Piscataway, NJ.}% <-this % stops a space
%\thanks{Manuscript received April 19, 2021; revised August 16, 2021.}}

% The paper headers
\markboth{IEEE Transactions on Parallel and Distributed Systems,~Vol.~X, No.~Y, Month~YEAR}%
{Shell \MakeLowercase{\textit{et al.}}: A Sample Article Using IEEEtran.cls for IEEE Journals}}

%\IEEEpubid{0000--0000/00\$00.00~\copyright~2021 IEEE}
% Remember, if you use this you must call \IEEEpubidadjcol in the second
% column for its text to clear the IEEEpubid mark.

\maketitle

%%%%%%%%%%%%%%%%%%%%%%%%%%%%%%%%%%%%%%%%%%%%%%%%%%%%%%%%%%%%%%%%%%%%%%%%%

\begin{abstract}

%This document describes the most common article elements and how to use the IEEEtran class with \LaTeX \ to produce files that are suitable for submission to the IEEE.  IEEEtran can produce conference, journal, and technical note (correspondence) papers with a suitable choice of class options.

%\pj{This first paragraph is optional, we can remove it:}
%Due to NVIDIA dominating the GPU market and despite its lack of cross-vendor portability, the C/C++-based application programming interface of CUDA and its related key libraries are still used in a significant fraction of software utilizing GPU-based acceleration. AMD's ROCm and its Heterogeneous-compute Interface for Portability (HIP) aims to alleviate the CUDA's lack of portability by providing a route out from the NVIDIA CUDA platform to AMD's devices.

  We describe \chipstar, an open source software stack which enables building unmodified CUDA and HIP programs to binaries that rely solely on open cross-vendor standards OpenCL and SPIR-V. The relevant technical aspects of \chipstar and the feature mismatches between CUDA/HIP APIs and OpenCL are discussed along with a set of standard extension proposals to bridge the essential gaps in the future.
  The key benefit of the software stack is its portability, which is demonstrated by providing performance evaluations on a diversity of less common CPU/GPU platforms including RISC-V/PowerVR and ARM Mali. A comparison against the original AMD HIP platform provides a geometric mean of X.XX \pj{TODO (Henry)}, a reasonable price to pay for the enhanced portability.    
  Although being a relatively young open source code base, \chipstar is now considered mature enough for wider testing and even production use, which is supported by successful porting of GAMESS, a complex HPC application which was deployed in the Aurora supercomputer.

\end{abstract}

\begin{IEEEkeywords}
CUDA, HIP, OpenCL, SPIR-V, Portability, Shared Virtual Memory
\end{IEEEkeywords}

%%%%%%%%%%%%%%%%%%%%%%%%%%%%%%%%%%%%%%%%%%%%%%%%%%%%%%%%%%%%%%%%%%%%%%%%%

\section{Introduction}

\IEEEPARstart{W}{alled} garden strategy is popular among market dominating companies. Its idea is to lock-in customers to company's products by making escaping the gates of the garden as costly as possible. NVIDIA's CUDA software platform is considered to be one of such walled gardens. It in part helps NVIDIA to expand and keep a foothold of their GPU market advantage, and at the same time maintain high innovation pace on the software APIs since there is no need to work with standardization committees that always have to aim for a consensus among multiple participating vendors.

Naturally, for end-users and the competing hardware vendors, the situation of a single-vendor dictated API is not ideal. End-users would prefer open standard software interfaces that enable switching the targeted hardware without incurring significant non-recurring engineering costs required for porting the applications and libraries to a new software platform just to be able to utilize the newly purchased hardware optimally. Similarly, other hardware vendors, aiming to get their piece of the market pie, would prefer an API that is not controlled by a single vendor.

AMD's ROCm~\cite{ROCm} software platform and its Heterogeneous-compute Interface for Portability (HIP) language~\cite{hip} helps escaping the CUDA walled garden by providing a route out from the NVIDIA CUDA platform to AMD's devices. HIP defines a subset of CUDA that is more easily portable to various hardware, thanks mainly to omitting various advanced features available in the later CUDA versions.
%(some of these features are discussed in Section~\ref{subsection:compatgaps}).
In order to enable easy automated transition path from CUDA applications, HIP is largely a copy of a CUDA C/C++ API subset with a few minor differences and renamed functions. HIP alleviates the CUDA portability problem, but doesn't solve it satisfactorily due to AMD targeting their self-specified low level ROCm APIs which are not actively supported on non-AMD platforms.
%\kh{this is related to the other point I brought up. It can be read as AMD supporting other HW somehow, but I think the truth is, it's not supported at all, especially as of today ROCm only emits AMD GPU ISA}\pj{OK. Then "for Portability" is quite misleading today. Do you have a reference to a manual or somewhere that says the CUDA output is not supported anymore?} 
An open source HIP/CUDA software platform solely based on open standards with a sincere aim for cross-vendor portability is still lacking.

With the \chipstar software stack described in this article we aim to help the CUDA/HIP
portability challenge. In contrast to previous solutions that either require source-to-source
conversion from CUDA programs~\cite{SYCLomatic}, that can lead to costly multiple codebase maintainance, or target binary-level compatibility of existing CUDA/HIP programs~\cite{ZLUDA} that
rely on questionable reverse engineering of proprietary binary interfaces (a brittle longer-term strategy), \chipstar chooses a middle-ground approach which enables source-level compatibility of HIP/CUDA programs by compiling them to a runtime portable ``fat binary'' that utilizes solely open standards and can execute on any platform supporting the required standard features without recompilation.

With this article we make the following contributions:

\begin{enumerate}
  \item We publish internal design choices of a software platform \chipstar that enables porting applications from NVIDIA-driven CUDA and AMD-driven ROCm platforms to any current and future platform supporting the cross-vendor open standards OpenCL and SPIR-V,
  \item evaluate its performance in comparison to running the CUDA applications directly using NVIDIA SDK or converting the applications to a popular open-standard based CUDA alternative SYCL~\cite{SYCL} and
  \item demonstrate the usability of OpenCL as a portability layer to implement other languages/APIs on top. The portability is shown by providing performance numbers on a RISC-V CPU \& PowerVR GPU and ARM platforms as well as on discrete GPUs from all main vendors. 
\end{enumerate}

The rest of the article is structured as follows: Section~\ref{section:portabilityAPIs} discusses our rationale for choosing OpenCL~\cite{OpenCL} and its device-side program representation SPIR-V~\cite{SPIRV} as the core APIs to support runtime portability in \chipstar.
Section~\ref{section:implementation} details the key technical issues in implementing the HIP/CUDA runtime on these APIs, while Section~\ref{section:compilation} focuses on the compilation aspects. Performance evaluation results are shown in Section~\ref{section:performance}. 
Section~\ref{section:applications} presents the GAMESS porting case study using \chipstar, and finally Section~\ref{section:conclusions} concludes the article.

%%%%%%%%%%%%%%%%%%%%%%%%%%%%%%%%%%%%%%%%%%%%%%%%%%%%%%%%%%%%%%%%%%%%%%%%%

\section{Heterogeneous Platform Hardware Abstraction Layers}
\label{section:portabilityAPIs}

As a technical background, we provide our considerations for the ``hardware abstraction layer'' API options forming the platform portability layer for \chipstar. The discussion is split to runtime APIs used to control the execution from the host side and device program representations providing an abstraction for the kernel side programs in a portable manner. Due to the abundance of potential target devices available for acceleration, we consider it important to be able to embed the device programs in an open standard-based intermediate representation (IR) and using JIT compilation for lowering the program to the target ISA of the accelerator at deployment or launch time. This enables future-proof ``fat binaries'' which can be supported on new platforms by implementing the specification of the IR. Another alternative would be to provide only source-level
compatibility where the application needs to be recompiled for each host and a device pair of interest, hindering binary distribution.

\subsection{Runtime APIs}

In practice, both CUDA and HIP are single-vendor supported programming models.
This is reflected, for example, in their platform property APIs which define limited
queries for device properties, highlighting features in each vendor's
GPU offerings. The goal of \chipstar is to
expand the portability of applications implemented using the CUDA/HIP APIs.
Therefore, the key requirement to the underlying ``platform/device portability API'' is to
cover as many of the essential features of CUDA/HIP as possible to
provide functional correctness and to exploit the potential performance benefits.
This includes,
for example, parallel and asynchronous execution of tasks, overlapping of
data transfers with task execution, and by providing access to
shared memory communication, if available. Furthermore, the portability API
should provide services to enhance performance portability of
the implementation by allowing to query the capabilities of the
devices to tune the execution at runtime to match the target's features.

There are not a large number of choices for such a runtime API, especially
if limiting the list to alternatives that enjoy official driver support from
multiple accelerator vendors or to those that have a portable long-maintained open source
implementation. In this regard, an open standard based API that has increased
in popularity is SYCL~\cite{SYCL}. SYCL
resembles CUDA in being a C++-based single-source API and
is a potential option for a portable runtime layer.
%It diverged from its original goal of an improved C++ binding to OpenCL to a more independent ecosystem with multi-backend implementations.
However, SYCL doesn't provide a runtime API or a way to query the
features of the underlying platform to tune the execution at
runtime.
%However, we consider OpenCL being a better match for portability
%layer use since it's a lightweight C API and has proprietary driver
%support from all major vendors and defines device and platform
%queries that can be used to tune the execution at runtime.
% SYCL should also support task graphs well:
%OpenCL also has a powerful task graph abstraction that can be interfaced
%with multiple in-order and out-of-order command queues~\cite{OpenCLTLP},
%events and, more recently, the relaunch of the task graphs can
%be optimized with command buffering,

Recently, OpenMP~\cite{OpenMP} has been considered for portability layer usage and,
in fact, specifically for implementing CUDA in~\cite{10.1145/3559009.3569687}.
While OpenMP enjoys support from a wide range of vendors,
we believe OpenMP is not ideal for this use case since it doesn't
define a device-side program representation, making future-proof cross-vendor portable fat binary generation difficult. It also can be considered
a high-level ``application-programmer-facing'' API similarly to SYCL,
thus offers constructs and overheads for programmer-productivity which are unneccessary for a portability layer.

Heterogeneous System Architecture (HSA) is an open heterogeneous platform
specification that also defines a runtime API~\cite{HSA,HSART}.
A key differentiating feature of HSA is that it standardizes on shared virtual memory,
making system-wide virtual memory addressing a required feature from
implementations. In hindsight this requirement was likely too much
too early, as system-wide virtual memory support is only relatively recently appearing
in hardware and still usually requires explicit allocation or mapping
calls from the programmer.
For this work, HSA could have been a valid choice for
a lightweight portability layer, but activity on the specification has seized with mainly AMD using only
selected parts of it in their software stack (the ROCr runtime component).
%\kh{It might make sense to talk with Dave Airlie about HSA specifically, from what I've heard is HSA was an attempt by AMD to make something cross-vendor, but in the end they were the only one deciding/doing anything. But I was never involved in HSA myself or even cared at all. I just know that Dave has some experience there.}\pj{I was also personally somewhat involved: I attended some of the meetings and implemented the obsoleted GCCBRIG frontend for HSAIL consumption for a then-client. Indeed it was like what you say; it just didn't get enough traction from other vendors and AMD dominated and everyone saw the standarding being AMD dominated (still do) although ARM was there for instance.}

Another option to consider would be Vulkan~\cite{Vulkan} since it also
provides a compute pipeline stage which allows specifying general purpose compute
kernels. However, the
feature set of the compute kernels lacks some of the OpenCL's features such as SVM which
would require further standard extensions.
%\kh{I don't think this is a strong point, because most of the kernel features are implemented in software anyway. All of the OpenCL C builtins can be implemented in OpenCL C as done by the libclc sub-project in LLVM. Other features like Program scope variables, printf or pipes are just buffers at runtime you pass into the kernel.}\pj{Right. You can implement everything in software, but can one provide efficiency/perf. portability if the standard lacks some of the features? If we cannot pinpoint any such features (other than that people consider it even more boiler plate to write than OpenCL), I can just mention it as a valid alternative.}\kh{There might be some micro-optimization possible fine tuning the implementation to specific hardware, but the GPU code to implement those built-ins is generally quite complex anyway. I think there is value in having a OpenCL C builtin library which could output vulkan SPIR-V instead for ease of use as currently it's only supporting OpenCL SPIR-V. For rusticl + zink what happens in mesa is, that we convert OpenCL SPIR-V to nir (mesa internal IR) which then gets translated to Vulkan SPIR-V and that works perfectly fine. So I think if somebody really cared, they could make the libclc stuff output vulkan SPIR-V instead.}\pj{OK. What about "platform level features" such as SVM?} 
It might be that in the future
the feature gap gets narrower and it would become a viable option. Meanwhile, layering OpenCL on top of Vulkan is an interesting option pursued by multiple open source projects to cover the devices which previously only had Vulkan driver support.

Level Zero~\cite{l0} is an API at a similar level of abstraction as HSA and OpenCL.
However, it's currently only supported by Intel's device, thus is not suitable for future-proof cross-vendor fat binaries.
%For \chipstar, Level Zero has a benefit that it uses the open standard SPIR-V as the device program
%representation, which in fact made it relatively easy to add an option for using Level Zero directly to control Intel GPU devices as an alternative to OpenCL~\cite{HIPLZ}.

Interestingly, OpenCL was originally created to provide an open cross-vendor alternative to the proprietary CUDA GPGPU programming model. It hasn't received wide support from application developers likely because it has been considered too low level and unproductive with the driver and feature support lagging behind the proprietary alternatives. However, as demonstrated by \chipstar, thanks to the official support from multiple vendors for the minimum feature set of the version 3.0 of the standard and multiple long-maintained open source implementations available, OpenCL provides a good portability layer for implementing other higher-level programming models and APIs on top.

\subsection{Device Program Representations}
\label{subsec:deviceProgramRepresentations}

Heterogeneous platforms suffer from the problem of device-side program description portability. There is a wide range of instruction-set architectures the kernels can target, and when the program is distributed in a binary form, the targets are known only at run time.
Thus, the choice of the format in which the device programs (kernels) are stored is critical as it should cover as many of the potential targets as possible.
Furthermore, the representation should be ``future-proof'' in a sense that the produced fat binaries could be made run in entirely new platforms by only referring to the API specification.
At the time of this writing, there still seem to be no clear winning program representation in this regard and various portable implementations of application-facing APIs are resorting to very fat binaries which store copies of the device program in multiple (virtual) instruction-set architectures to cover the various targets and offloading runtimes it might encounter at execution time. This is the case with~\cite{10.1145/3559009.3569687} and originally in AdaptiveCpp~\cite{10.1145/3529538.3530005}.

Recently AdaptiveCpp started storing kernels in the LLVM~\cite{LLVM} compiler Intermediate Representation (IR) instead of storing multiple different binaries depending on the target. In this scheme, LLVM IR is lowered to various target-dependent formats at runtime at the point when the target is known~\cite{OpenSYCLfatbin}.
This approach has benefits in comparison to storing abundance of device binaries in the another alternative, and works in theory, but it is also known that LLVM IR is not supposed to be a portable program representation as it can embed target-specific intrinsics, has target specific data layout and endianness among other challenges.
LLVM IR is not guaranteed to be stable across LLVM versions, which means that the fat binaries should have access to an LLVM library of version the IR was generated with, which at worst requires to embed the LLVM library along and the required backends to the fat binary, forming an unnecessary dependency.
The problem of LLVM IR not being target-independent nor stable across LLVM versions was attempted to be addressed by earlier Standard Portable Intermediate Representation (SPIR) versions 1.2 and 2.0~\cite{SPIR2}: These first SPIR versions were designed to support OpenCL C language kernels and were based on defined versions of LLVM IR, which proved to be difficult to maintain long term.
%\pj{Ben: do you have insights on the historical reasons here?}.
The LLVM-based SPIR-versions were later obsoleted in favor of the SPIR-V~\cite{SPIRV} format.
The goal for SPIR-V is to provide a robust cross-vendor specified intermediate language which is not affected by LLVM upstream changes and that shares specification effort with the Vulkan community.

HSA specification defines an intermediate language called HSAIL and a binary representation called BRIG~\cite{HSAIL}. A key technical difference of HSAIL in comparison to the SPIR-V format \chipstar chose to use is that HSAIL had a fixed number of registers and an address space for spills unlike SPIR-V, which has infinite virtual registers due to being based on the Static Single Assignment (SSA)~\cite{SSA} representation.
Similarly to Level Zero after it, HSA made a choice to not define a higher-level programming language (like OpenCL C) for the device programs, but only standadized a low level IR.
Like is the case with the HSA runtime specification, however, the activity on the HSAIL spec has stalled.
There was also a GCC-based frontend for consuming BRIGs in a target-portable fashion, but after activity on HSA quieted, the ``BRIG frontend'' was removed from the upstream GCC source code repository in a May 2021 commit.

As a conclusion, while SPIR-V OpenCL environment support from processor vendors is not very extensive as of this writing, it seems to be still the best option for a cross-platform representation given it's an open standard defined democratically by multiple hardware vendors and is relied upon by OpenCL and Vulkan and SYCL implementations among others. 
%
Thanks to open source tooling support available and useful SPIR-V producers such as \chipstar and DPC++ appearing, the list of supported targets is expected to grow in the future.

%Since it relies on a well-specified specification, the fat binaries produced by \chipstar relying on OpenCL and SPIR-V are ``future-proof'', making it feasible to add binary-level support to new devices while only referring to the specifications.

%The two core OpenCL features required by all CUDA/HIP applications when built with \chipstar are the SPIR-V input and coarse-grain SVM, both of which are optional features of the OpenCL 3.0 standard.
%Although OpenCL has again risen in popularity in the recent years, thanks to its more easily minimally-implementable 3.0 version, as of this writing, only ARM GPUs, Intel GPUs and Intel CPUs have vendor-provided OpenCL drivers that support SPIR-V input.

%Fortunately there are various active open source OpenCL implementation projects that can be used and expanded to fill up the lack of features in the proprietary drivers at least until they catch up. The two most vibrant ones are Rusticl and Portable Computing Language (PoCL~\cite{PoCL}). These two projects were used utilized to extend the portability of \chipstar-produced fat binaries to CPU targets.

\section{Implementing HIP/CUDA on OpenCL Runtime API}
\label{section:implementation}

The primary goal for \chipstar is to support the subset of CUDA features
as defined by HIP and expanding the feature set beyond it whenever feasible
while relying on the chosen open standard APIs as much as possible.
In this section, we discuss how the OpenCL/SPIR-V specifications match with
the commonly used features of CUDA/HIP and identify the most impactful gaps
that we believe should be covered in the future.
%At the time of this writing, HIP refers to features at CUDA version 9.0
%or older.
%, thus excludes modern functionality available in later NVIDIA
%devices such as page-fault relying unified memory.

\subsection{Memory Model}

Due to their common history in GPGPU programming, CUDA/HIP and OpenCL, share various
platform and memory model abstractions. For example, ``device memory'' is the same as
``global memory'' in OpenCL terminology (``shared'' is ``local'').
To avoid confusion in terminology we use only the CUDA/HIP terms in the rest of
this article. Similarly, we refer to the original CUDA versions when talking about
functions that have their counterparts in the HIP API.

A key difference between OpenCL and CUDA that required addressing was the
fact that CUDA implicitly infers the address space of the data in the device
program side whereas in OpenCL (before v2.0) the address space must be declared explicitly.
The CUDA's implicit address space inference is similar to the 'generic'
address space concept introduced in OpenCL v2.0, which was utilized to bridge
this gap.

The simplest interface in CUDA's host-side device memory management is \func{cudaMalloc()}.
It returns a raw pointer to the targeted device's global memory, instead of an opaque handle
as is the case with OpenCL's basic buffer management functionality. This presents a small
but significant difference from the OpenCL v1.x specification for device memory management;
OpenCL v1.x only provides a buffer management API (\func{clCreateBuffer()} and others) which
returns opaque \type{cl\_mem} handles.
%
The opaque buffer handles cannot be used to implement CUDA device memory allocation
because they do not provide access to the underlying raw device address or passing addresses in other data
structures, which is allowed with the CUDA device pointers. In order to implement these
capabilities, we utilized the Shared Virtual Memory (SVM) API that first appeared in
the OpenCL v2.0. The raw pointers was the another key difference along with the
implicit address space inference which required us to lift the minimum OpenCL
version to v2.0 to support even the most basic CUDA programs.

The SVM allocation API returns a raw pointer to a shared
virtual address space region. The ``Coarse Grained buffer SVM'' (CG SVM) variant can be used to
implement the basic device memory allocation. Mapping device memory allocation to CG SVM
has a drawback that the device driver must support some of the unneeded SVM features such as
mapping the allocated regions to the virtual address space although just returning physical
device memory pointers would suffice. This means that the \chipstar implementation is actually
implementing CUDA's Unified Memory model by default. To alleviate the potential performance
impact of this, \chipstar can also use the Intel Unified Shared Memory (USM)
extension (cl\_intel\_unified\_shared\_memory), if supported by the runtime. USM enables allocating strictly
device-only allocations, but still returns virtual pointers, which can be problematic for some implementations. In order to provide an allocation API matching the basic \func{cudaMalloc()}'s needs, we introduced a new extension (cl\_ext\_buffer\_device\_address) that enables querying the raw device pointer of a cl\_mem allocation without needing to map the buffer to the same address range in the host's virtual memory. \chipstar can use any of these alternative APIs.

CUDA provides an API to \textit{pin} memory so it's kept resident in the host memory and
optionally made accessible by devices from kernel code and is not swapped out to disk. 
The primary APIs to this functionality are
\func{cudaHostAlloc()} and \func{cudaHostRegister()}. The former allocates pinned
memory directly and the latter pins a previous host allocation. \func{cudaHostAlloc()}
is simple to implement with coarse grained SVM since by the coincidence of using
a shared virtual memory allocation, the buffers are by default accessible in both the host and
the device using the same pointer. However, the allocation might not be resident for the
duration of the execution, for example, if a CPU device is allowed to swap out such
allocations. That aspect, however, is only potentially inspectable as a performance difference.
\func{cudaHostRegister()} is a bit more challenging to implement on top of CG SVM since it
allows registering a host address range to be a pinned region accessible both from the host and
the device \textit{after}
the host memory has been allocated. Since the allocation might not be originally
been allocated with the OpenCL SVM allocation API, but with a system memory allocator or even
from the stack, to implement correct functionality in this case, \chipstar creates
a shadow buffer using \func{clSVMAlloc()} and synchronizes it with the host region at
kernel start and end points. OpenCL 2.1 added a new \func{clEnqueueSVMMigrateMem()} API that enables fine grained specification of where regions of SVM are migrated, but is not useful for this case since the source of \func{cudaHostRegister()} can be any host memory area whereas the API handles only SVM allocations.
%\kh{OpenCL 2.1 added clEnqueueSVMMigrateMem to give hints to the runtime where to put the content of the buffer, might make sense to mention it here and say why it's not a good fit.}\pj{Done.}\kh{I think something at the end is missing here}
% https://developer.nvidia.com/blog/unified-memory-cuda-beginners/

The later NVIDIA architectures since compute capability 6 support on-demand page migration which
relies on hardware memory management unit (page fault based buffer migrations) for coherence
on the Unified Memory allocations. This frees the programmer from the need to perform explicit memory allocation and synchronization calls. The functionality maps to the Fine-Grained System SVM of OpenCL, but since its support by hardware and drivers is very rare at the time of this writing,
it is not yet implemented by \chipstar.

\subsection{Tasks and Events}

The semantics of CUDA \textit{streams} and the ability to execute tasks/commands
asynchronously maps well to the \textit{command queues} of OpenCL. Each stream is expected
to execute commands in-order, which matches the in-order command queue semantics of OpenCL.
Commands are allowed to execute concurrently even within in-order command queues in OpenCL,
as long as the results are not observable from the outside, enabling concurrent kernel
execution~\cite{OpenCL}.

To facilitate out-of-order execution from CUDA/HIP programs the programmer has to rely
on the explicit event synchronization and recording APIs which are implemented
using OpenCL out-of-order queues by \chipstar, if supported by the target. The CUDA/HIP event API differs from the OpenCL: In CUDA/HIP the user is responsible for explicitly creating and recording events, while in OpenCL the runtime implicitly creates events when enqueuing commands. Recording events in \chipstar is implemented by creating a new marker-type \type{cl\_event} (\func{clEnqueueMarkerWithWaitList()}) when \func{cudaEventRecord()} is called.

\subsection{Textures}

\chipstar supports only a simple subset of texture objects due to a limitation in OpenCL images. The notable differences between HIP/CUDA and OpenCL are that the texture objects are pointers to opaque C/C++ structures whereas in OpenCL/SPIR-V there is a special type per image dimensionality and that the texture objects can be loaded indirectly whereas OpenCL images can be only passed to kernels as kernel arguments. Therefore, some constructs such as the following cannot be expressed in SPIR-V:

\begin{verbatim}
  hipTextureObject_t Tx = ...;
  Ty Tv = cond ? tex2D<Ty>(Tx, X, Y) 
               : tex1D<Ty>(Tx, X)
\end{verbatim}


% Pekka> Hiding this for now. It's an interesting feature, but not strictly needed for CUDA/HIP.
%\subsection{Lower Layer API Interoperability}

%\pj{TODO Sarbojit: Describe the HIP-OpenCL and HIP-SYCL interoperability APIs and their use cases.}
%\pj{Can you add code examples of using the different interop APIs?}

%The native interoperability API can be used to initialize HIP context (with assigned device \& command queue) from a set of native (LevelZero/OpenCL) object handles, or in the opposite direction to retrieve a set of native handles from an existing HIP context. Thread-safe use of handles is currently left to the application (which should be non-issue with OpenCL since it is thread-safe). Additionally, there are two APIs that create a HIP event from native event handle, and vice-versa. These can be used for interoperability of HIP code with native code while maintaining asynchronous execution.\pj{Can we share buffers somehow between APIs? Or is that down to the "external memory extension"?}\pj{Is the SYCL interop via LZ/OpenCL, no direct API calls?}

%\subsection{OpenCL-CUDA/HIP Compatibility Gaps}

%\pj{Pekka TODO: This is a verbatim copy from HIPCL, to update:}
%Most of the HIP API maps trivially to the OpenCL API, with some notable exceptions which might call for new OpenCL extensions:\pj{TODO: We should just make them extensions (proposals) to clean up the story.}

%\mb{Pekka TODO: do we also list APIs which can be implemented but aren't yet (because nobody's done the work) ? looking quickly at CHIPBindings.cc, there are >50 hip API functions which have not been implemented, things like Peer2peer, hipIPC*, hipModuleOccupancy*, hipProfiler*, hipMemPool*, hip{Malloc,Free}Async etc; some might require OpenCL extensions }
%\pj{I think not worth listing here, as it's only a matter of time when these are implemented and if apps do not use them, they %are not high prio.}

%\begin{itemize}

%\item {hipGetDeviceProperties()}: for certain device properties, there is no portable way to get the information via the OpenCL device query API.\pj{this should be an easy extension}

% Pekka> I think we can do without this as it's visible only in terms
% of latency/performance to the user, and there should be also other
% similar features which can be observed only in terms of perf., not
% functional correctness (e.g. the typical concurrency to parallelism mapping).
%\item {hipSetDeviceFlags()}: the flags to this call control how the host thread interacts with the driver thread while waiting for the device (yield the host thread to OS, or spin wait).
%
%\item{hipEventCreateWithFlags()}: provides per-event control of the synchronize behaviour (yield thread/spin wait). However, these APIs affect only performance, not correctness, thus can be implemented as no-operations.

% Pekka> Cannot we really implement this without an extension even if we had kernel metadata to traverse? \mb{possibly, if we can always figure out the correct alignments & padding}
%\item {hipModuleLaunchKernel()}: passing args by ``extra'' parameter requires an API for setting all kernel arguments at once.\pj{a new clEnqueueNDRange variation with a HSA-style-specified exact layout argument buffer layout might be useful in any case.}

%\item {hipGraph API}: the API to create, update & launch graphs. The existing cl_khr_command_buffer extension is not sufficient, since we need to work with SVM. (discussed below in the "opencl and spirv extensions" section).

%\item {hipHostRegister}: we'll need an OpenCL extension to implement this (unless there is something already we could use, i haven't checked).

%\end{itemize}

%%%%%%%%%%%%%%%%%%%%%%%%%%%%%%%%%%%%%%%%%%%%%%%%%%%%%%%%%%%%%%%%%%%%%%%%%

\section{Compilation Aspects}
\label{section:compilation}

This section discusses the compilation flow used by \chipstar. We introduce the overall compilation flow, the kernel built-in library implementation and summarize our findings on the key needs to extend the OpenCL/SPIR-V standards to bridge key feature gaps between the specifications.

\subsection{The Compilation Flow}

The offline compilation flow of \chipstar is based on the LLVM Project's~\cite{LLVM} Clang~\cite{Clang} frontend. The overall compilation process is shown in Fig.~\ref{fig:compilation}. It relies on the CUDA/HIP frontend of Clang, which was extended to produce SPIR-V binaries as an option to PTX or AMDIL for the device program. The LLVM \func{opt} tool is used to invoke special LLVM passes provided by \chipstar for lowering HIP features to the OpenCL-SPIR-V environment. The SPIR-V translation is performed using Khronos' LLVM-SPIRV-Translator tool.

\begin{figure*}
    \centering
    \includegraphics[scale=1]{figs/chipstar-compilation-v2.pdf}
    \caption{The offline compilation flow.}
    \label{fig:compilation}
\end{figure*}

Most of the compilation related changes have been upstreamed to the LLVM project and very little compilation related functionality remains within the \chipstar code base. The notable exceptions are compiler passes that handle CUDA vs. OpenCL differences in \func{printf()}, implement a device side \func{abort()} feature, handling of the CUDA's device-side global variables, and an indirect memory access analyzer. The indirect memory access analyzer marks kernels that are known to not indirectly access allocations, which removes the unneccessary synchronizations for the majority of
benchmarks seen so far. Otherwise, due to the CUDA's memory model, each launcher kernel can potentially access any previously allocated buffer, inducing significant unneccessary
data synchronization overheads in the common case where the kernels only access buffers set through their arguments.

%One of the performance-impacting device program passes done in \chipstar is indirect memory
%access analysis: In CG SVM, memory consistency between device and host memories is guaranteed at
%the execution boundaries of kernel commands referring to the SVM allocation. To enforce
%data synchronization, kernels referring to the SVM allocations must either refer to the
%SVM allocations as kernel arguments, or be explicitly marked to indirectly use other
%allocations. The latter poses a challenge, since in principle any kernel can refer to any
%previous allocation as CUDA device pointers can
%be passed inside data structures or global variables.
%For CG SVM,
%OpenCL's \func{clSetKernelExecInfo()} must be used to list all potentially used SVM allocations
%that are not referred to by the argument list.
%This poses a signficant performance overhead risk since the \chipstar runtime must play it safe
%and register all possible previous allocations to any launched kernel, unless proven that the
%kernel doesn't refer to a particular allocation. For applications with a lot of allocations and
%kernels that only use subsets of the buffers, a lot of unnecessary data
%synchronizations between the host and the device memories can happen. To alleviate this
%problem for the seemingly common case of only referring to buffers in the argument list,
%\chipstar implements a kernel analyzer that 

%\pj{Henry (?) TODO: Discuss the eager compilation slowness problem and how it was solved in chipstar and PoCL-level0.}

Fig.~\ref{fig:online-compilation} shows the online compilation flow from SPIR-V to device code in \chipstar runtime. A SPIR-V module is compiled just-in-time when a kernel associated with it is launched. To enhance runtime portability, the online device builtin library provides HIP builtin function variations for different device capabilities which are linked to the user’s device programs at runtime. For example, for HIP floating-point atomics the runtime chooses between an implementation that maps them to corresponding native functions via a SPIR-V extension or emulates them via atomic exchange operations.

\begin{figure*}
    \centering
    \includegraphics[scale=0.9]{figs/chipstar-rt-compile-n-link.pdf}
    \caption{The just-in-time compilation flow.}
    \label{fig:online-compilation}        
\end{figure*}


\subsection{Device Built-in Library}

The \textit{chipstar} device-side library implements the HIP math API, by using a combination of OpenCL C math builtins, OCML (part of ROCm-Device-Libs), and custom implementations.
A lot of the functions in the math API have an equivalent OpenCL builtin with adequate accuracy guarantees with a few exceptions that cannot be mapped directly, and thus require software based emulation such as floating-point atomics on some devices. The main challenge in terms of a fast yet portable implementation of the functions are due to differences in math accuracy requirements between CUDA/HIP and OpenCL: most of the standard math functions of CUDA are defined in higher accuracy than what the OpenCL standard requires.

Furthermore, CUDA/HIP defines a set of \textit{intrinsics}, which are faster yet less accurate versions of the standard functions. This exposes a further difficulty when aiming for a portable yet fast implementation: It heavily depends on the targeted platform what level of accuracy is achievable while still enabling execution time benefits. Since CUDA is inherently meant not to be cross-vendor portable, the intrinsics are defined only to match the CUDA microarchitecture in an optimal manner, which might not be the case for other devices. 

OpenCL covers the use case of accessing fast but inaccurate hardware operations by means of a relaxed mathematics flag that can be enabled at device program build time and with so called native built-in functions in the built-in kernel API. Unfortunately, neither of these are usable for implementing the CUDA intrinsics by default due to not guaranteeing enough accuracy: The relaxed math in OpenCL defines maximum rounding errors, but they are usually slightly less than what the CUDA intrinsics require. The OpenCL native built-in functions are even worse fit for this use since they guarantee nothing of the accuracy but leave it entirely up to the implementation. There is not even a possibility to query for the maximum error via a runtime API, but the accuracy must be discovered via trial-and-error or from documentation of the hardware vendor.
%\pj{TODO: Check the HIP statements of guaranteed accuracy.}

The ``correctness first'' principle requires implementing the functionality by default with guaranteed accurate enough arithmetics, which means to not receive any performance benefits of simplified implementations.
%
Correctness for the basic math functions would require software emulating them with added accuracy, of which performance impact would likely be too drastic to make \chipstar unusable for high performance workloads for which it is typically used. We chose a middle-ground where the basic math functions are implemented at the OpenCL accuracy by default and the intrinsics also utilize the default functions instead of the native functions, thus do not get any performance benefits from intrinsics. For the workloads we tested, this seemed to be a good enough solution.
We plan to optimize this aspect in the future via a new standard extension with a set of builtins that guarantee the CUDA accuracy requirements to the application programmer while enabling the targeted platform to optimize and implement them as efficiently as possible.

% https://github.com/CHIP-SPV/chip-spv/issues/222
% https://intel.github.io/llvm-docs/cuda/cuda-vs-opencl-math-builtin-precisions.html

An LLVM pass is responsible to lowering texture object API based texture functions to OpenCL image fetches. The pass analyses endpoints of the texture objects by following their use-def and def-use chains. If the pass sees that a texture object is coming from a kernel parameter and it is only used by texture fetch calls for the same dimensionality, it will replace the texture object parameter with image and sampler parameters and translates the texture fetch calls with OpenCL image fetch calls of matching dimensionality which consume the new kernel parameter. 

\subsection{OpenCL Extensions}

\chipstar compilation flow is built in a way that different advanced OpenCL features and extensions are not required from the target platform's driver or device unless the compiled input application specifically needs them. Although the minimal OpenCL 3.0 feature set plus coarse-grained SVM and SPIR-V consumption support covers a significant part of most commonly used CUDA and HIP features, some functionalities require or can be improved with various extensions to the OpenCL or SPIR-V specifications.

\begin{table*}[ht]
    \centering

    \begin{tabular}{|p{5 cm}|p{5cm}|p{5cm}|}
    \hline
\textbf{Extension name} & \textbf{CUDA/HIP feature(s)} & \textbf{Status} \\
    \hline
cl\_ext\_alive\_only\_barrier       & A special work-group barrier for barrier calls which might not be reached by work-items that have exited the kernel. As allowed by the CUDA's execution model. & To be proposed. \\
    \hline
cl\_ext\_cuda\_math     & Implement math functions and intrinsics with precision requirements that match CUDA's. To enable more optimized intrinsics. & To be proposed.  \\
    \hline
cl\_ext\_device\_side\_abort        & Implement \func{\_\_trap()} on the low-level runtime side. The current implementation requires compiler transformations. & Public.  \\
    \hline
cl\_ext\_extended\_device\_properties & \func{hipGetDeviceProperties()} can be used to query more device properties than the basic OpenCL device or platform query APIs support, this fills the gap. & To be proposed. \\
    \hline
cl\_ext\_relaxed\_printf\_address\_space &  CUDA's \func{printf()}behavior with non-constant address spaces. Currently handled with compiler transformations. & Public. \\
    \hline
cl\_intel\_unified\_shared\_memory & Used for optimized \func{cudaMalloc()} when available.  & Public. To promote to a general 'khr' extension. \\
   \hline
cl\_khr\_command\_buffer            & For optimized implementation of CUDA graph re-execution. & Public. SVM commands added in v0.9.4.  \\
    \hline
cl\_ext\_command\_buffer\_host\_data & For optimized implementation of CUDA graphs which transfer data between the host and the device. & To be proposed. \\
    \hline
cl\_ext\_command\_buffer\_host\_sync & For optimized implementation of CUDA graphs which synchronize with the host. & To be proposed. \\
    \hline
cl\_khr\_subgroup\_requirements & Used to fix the warp-size and force the desired thread id mapping when calling warp-level primitives that depend on the fixed warp size or the thread id ordering. & Private. Promotion to a general 'khr' extension proposed. \\
    \hline
cl\_khr\_fp64                       & If double precision floating point is used. & Public. \\
    \hline
cl\_khr\_global\_int32\_base\_atomics \newline
cl\_khr\_global\_int32\_extended\_atomics \newline
cl\_khr\_local\_int32\_base\_atomics  \newline
cl\_khr\_local\_int32\_extended\_atomics \newline
cl\_khr\_int64\_base\_atomics \newline
cl\_khr\_int64\_extended\_atomics & Atomic operations. & Public. \\
    \hline
cl\_khr\_subgroups                  & Warp-level synchronization with \func{\_\_syncwarp()}. & Public. \\
    \hline
cl\_khr\_subgroup\_ballot           & Warp-level ballot operations. & Public. \\
    \hline
cl\_khr\_subgroup\_shuffle          & Warp-level shuffle operations. & Public. \\
    \hline
    \end{tabular}
    \caption{OpenCL 3.0 standard extensions that \chipstar can use currently or will use in the future to implement CUDA/HIP features if the application uses them. Status describes the state of the extension at the time of this article's publication. }
    \label{table:extensions}
\end{table*}

In Table~\ref{table:extensions} we summarize the standard extensions \textit{chipstar} can utilize and which CUDA/HIP feature triggers their need. The extensions are in different stages in the Khronos Group standardization process, which is also noted in the table.\footnote{Note to reviewers: We will update the status for the final article version.}
Most of the extensions are straightforward and the brief description in the table should suffice to grasp their purpose. However, the handling of warp-level primitives calls for a bit more thorough explanation:
One of the execution model differences between CUDA to OpenCL is that CUDA presents a finer grained fixed size grouping of the threads (OpenCL work-items) than the blocks (work-groups) called a \textit{warp}. In the earlier CUDA versions, the threads in a warp could be assumed to execute in lock-step, implying that the enabled threads in the same warp would execute the same instruction. This implied that in some cases explicit synchronization could be omitted: In case of a usual read-modify-update case, the programmer could trust that the warp's threads all execute the read part before any of them proceed to the update part, enabling in-place-updates without explicit synchronization. However, with the later specification versions of CUDA relying on lock-step behavior in the program logic was deprecated~\cite{NVIDIAProgrammersManual}. ...

In addition to older CUDA programs potentially relying on the lock-step semantics to omit explicit synchronization, the fixed size warps (32 threads for NVIDIA and usually 64 threads in AMD devices) affect the execution semantics when executing warp-level functions that rely on the warp grouping and the mapping of the threads to the lanes of the warp.  Such primitives include the warp shuffles, which read data from a specific lane within the warp, and the explicit warp synchronization primitives.

%\pj{There could be a figure here with possible subgroup id mappings and how warps always map the threads in linear order.}
The OpenCL specification, on the other hand, doesn't have a warp concept, but the work-items are free to make progress in any order and grouping. It has a feature extension called ``subgroups'' which is used to implement the warp semantics in \chipstar when the kernel is detected to need it. However, in contrast to warps which have a specified form and content which allows the programmer to utilize them reliably, the basic subgroups of OpenCL are ``implementation-oriented''; they enable grouped execution in a manner that is simplest or most efficient for the driver and the hardware at hand. The sizes of the OpenCL subgroups are not fixed, but must be queried per kernel by the programmer in the basic extension. Also the way work-items are mapped to subgroup lanes (so they can be referred to when using cross-lane intrinsics) is also implementation-defined. To close the gap between subgroups and warps, a standard extension \textit{cl\_khr\_subgroup\_requirements} that \textit{forces} the subgroup size of the kernel to the desired size along with the linear id mapping was proposed.
%\kh{As Nvidia hardware generally has a warp size of 32 threads, what are the plans when OpenCL runtimes/devices can't support the size required by the CUDA application?}\pj{No plans so far. We could software-emulate different warp sizes using work-item loops/replication, but I'd rather not go there if not strictly necessary. Does RustiCL support 32-wide subgroups with NV/AMD targets?}\kh{As of today rusticl doesn't support forcing a subgroup size, but drivers are written in a way that it would be possible to do so once the appropriate CL extension is implemented. The AMD mesa driver also has a debug env variable to force 32-wide subgroups. The Intel driver chooses between 8/16/32 as it sees fit. So yeah, just a bit of code missing for it.}

% HIP doesn't support the new _sync-ones, so let's focus on it.
% Maybe also in the title of the paper.
%\pj{Pekka TODO: Non-uniform primitives.}

%\subsection{Compiling CUDA Applications Directly}
%\label{section:directCUDA}

%While the primary goal of \chipstar is to cover the CUDA/HIP APIs to the extent defined by the HIP programmer's manual, \chipstar implementation also supports a set of CUDA APIs directly. The ability to call CUDA APIs directly drops the need for source-to-source translations when porting originally CUDA applications to the platform. This is done by simply delegating the CUDA API calls to the HIP versions, similar to what HIP does with their CUDA mapping, but in reverse.\pj{to check}

%There has been legal controversy related to APIs how they are covered by the copyright laws in the past that has made legality of direct implementations of proprietary APIs unclear. This changed with the Supreme Court of the United States ruling of April 5, 2021 in the Google LLC vs. Oracle America, Inc. case, which stated that copying the Java API for use in the Android OS was considered ``fair use'' since it was done for compatibility purposes:

%\begin{quote}
%``Google’s copying of the Java SE API, which included only those
%lines of code that were needed to allow programmers to put their accrued talents to work in a new and transformative program, was a fair
%use of that material as a matter of law.''~\cite{JavaSupreme}
%\end{quote}

%Although no code was copied directly from the NVIDIA implementation, even if it was the case we believe
%our limited implementation of the CUDA API falls well within such fair use outlined in the ruling. However, since we, the \chipstar developers are engineers, not lawyers, we wanted to be extra careful that copyrights were not disrespected in any jurisdiction when adding support for direct CUDA API calls by using an implementation approach where only the programmer's manual was consulted for the API reference when implementing the CUDA compatibility headers to the \chipstar code base. 

%%%%%%%%%%%%%%%%%%%%%%%%%%%%%%%%%%%%%%%%%%%%%%%%%%%%%%%%%%%%%%%%%%%%%%%%%

%\section{Libraries}
\label{section:libraries}

The CUDA and ROCm software platform include a large set of useful libraries in addition to the general purpose program input support. These libraries include common routines such as BLAS (Basic Linear Algebra Subprograms) and Deep Neural Network (DNN) acceleration libraries. Implementing all of the libraries are out of scope of this work which focuses on providing portable and robust CUDA/HIP language support. However, we've already identified a set of key libraries and created example ports of them, enhancing portability of CUDA/HIP programs to Intel devices. The plan is to expand the set of supported libraries as required and demanded by the ported applications of interest. We discuss the currently supported libraries in the following and highlight their essential technical aspects.

\subsection{cuBLAS/hipBLAS for Intel GPUs}

BLAS is one of the core libraries in HPC. To implement the CUDA/HIP interfaces to BLAS for Intel hardware, we implemented an oneMKL backend to the hipBLAS performance library to run on Intel GPUs. Applications calling hipBLAS functions can thus run on Intel hardware without any code changes.
The oneMKL backend for hipBLAS uses the \chipstar interoperability feature with SYCL to invoke oneMKL’s SYCL functions. During the hipBLAS handle creation, the oneMKL backend extracts the native queue handle of a HIP stream using interop APIs from CHIP-SPV.  It uses the native queue handle to create a corresponding SYCL queue to execute the  oneMKL functions for the calls initiated from the hipBLAS handle. Since SYCL and \chipstar both support Level Zero and OpenCL runtimes, the oneMKL backend for hipBLAS also supports both: Users can switch between them using environment variables and the rest will work transparently in the hipBLAS library.

To assess the overheads of the implementation, we measured single precision GEMM function performance with 2048 x 2048 matrix size comparing hipBLAS, oneMKL SYCL and oneMKL OpenMP runtime and APIs on a pre-production Intel PVC system and measured execution time, comparing to running oneMKL directly from SYCL. The difference was negligible.

%%% PEKKA'S EDITING GOING HERE %%%

%Fig.~\ref{fig:hipBlas-rel-perf} shows the relative performance with oneMKL SYCL as the base.  We found that the performances are closely matched across all three runtimes.\pj{Not sure if it's worth adding the different iteration counts. If you look at the other results later, we took the best execution times over 100 iterations to give the roofline for perf. achievable. The perf. should vary only with a cold cache or if there's some other load in the system competing for the resources, both cases of which we should filter out this way.}

%Currently, oneMKL does not provide a pointer-mode API as defined in HIP and CUDA. To this end, we added a wrapper in oneMKL backend for hipBLAS to emulate pointer mode under oneMKL. In HIP device pointer mode, the oneMKL backend needs to copy scalar parameters, such as ‘alpha’ and ‘beta’ in GEMM between host and device memories.  Fig.~\ref{fig:hipBlas-host-vs-dev} shows more than 50\% performance drop with the workaround hence it is advised to avoid using device pointer mode in the current release.

%\begin{figure}
%     \centering
%     \begin{subfigure}[b]{0.5\textwidth}
%         \centering             
%         \includegraphics[width=1\textwidth]{figs/comparision_between_sycl_hip_openmp.pdf}
%         \caption{Average time taken for 5k gemm run with different runtime. Data is normalized with Sycl as 1.}
%         \label{fig:hipBlas-rel-perf}
%     \end{subfigure}     
%     \begin{subfigure}[b]{0.5\textwidth}
%         \centering             
%         \includegraphics[width=1\textwidth]{figs/comparision between_host_and_dev_ptr.pdf}
%         \caption{Average time taken for 5k gemm run with host and device pointers. Data is normalized with host pointer as 1.}
%         \label{fig:hipBlas-host-vs-dev}
%     \end{subfigure}
%     \caption{...\pj{todo}}
%\end{figure}

\subsection{cuDNN / MIOpen}

\pj{TODO: SYCL-DNN. Involve Codeplay with this paper?}

\subsection{rocPRIM for CUB compatibility}

\pj{TODO: test with cub}

\subsection{CUDA Graphs / MIGraphX}

\pj{TODO Michal: Describe mapping the Graph API to the command buffer API}


%%%%%%%%%%%%%%%%%%%%%%%%%%%%%%%%%%%%%%%%%%%%%%%%%%%%%%%%%%%%%%%%%%%%%%%%%

%\section{Debugging and Profiling Support}
\label{section:debuggingAndProfiling}

Thanks to using the open standard OpenCL as the portability layer, various debugging and profiling tooling options are available to use with little to no additional effort. The following discusses a set of tools that were successfully adopted and used through $chipstar$.

\subsection{Profiling Tools}

% VTune is rather Intel device-specific, uses its monitoring
% interfaces. We cannot provide
% much on top of OpenCL directly for
% wider device support since it doesn't
% have a monitoring API. The only
% data is the profiling time stamps
% from profiling queues.
%\subsubsection{VTune}

%\pj{TODO Paulius?}

\subsubsection{Tracing Heterogeneous APIs (THAPI)}

% https://github.com/argonne-lcf/THAPI

\pj{TODO Brice?}

\subsection{Debugging}

\pj{TODO: PoCL-CPU, GDB, Valgrind, debug info...}

%%%%%%%%%%%%%%%%%%%%%%%%%%%%%%%%%%%%%%%%%%%%%%%%%%%%%%%%%%%%%%%%%%%%%%%%%



%%%%%%%%%%%%%%%%%%%%%%%%%%%%%%%%%%%%%%%%%%%%%%%%%%%%%%%%%%%%%%%%%%%%%%%%%



%\subsection{libCEED}
%\pj{TODO: Paulius?}

%\subsection{Pytorch-HIP}

%\pj{TODO: Henry?}

%%%%%%%%%%%%%%%%%%%%%%%%%%%%%%%%%%%%%%%%%%%%%%%%%%%%%%%%%%%%%%%%%%%%%%%%%

\section{Evaluation}
\label{section:performance}

We evaluated the \chipstar speed on various OpenCL-capable CPU and GPU platforms using selections of benchmarks from the HeCbench collection~\cite{HeCbench}. All of
the results were produced using the \chipstar v1.2  release~\pj{TODO: rerun the benchmarks after the release is out}.

The HeCbench benchmark application selection criteria for each comparison was as follows:
\begin{enumerate}
    \item The application had the necessary API/language variations with an HIP version that could be built with \chipstar v1.2 and its ported libraries.
    \item For SYCL/CUDA comparisons, there must not have been significant identified performance-affecting structural or implementation differences between the SYCL/CUDA and HIP versions of the application. Some of the identified ``unfair differences'' were fixed and submitted back to the HeCbench repository \pj{TODO (Henry/Michal): a pull request of the changes.}.
    \item The application could verify its results and had to validate correctly on all platforms involved in the comparison.
    \item Applications that required hardware or OpenCL driver features that were missing or too limited on the platform were omitted.
    This mostly concerned the runs on embedded/integrated GPUs with limited memory or lack of double precision floating point support.
\end{enumerate}

\subsection{SYCL/DPC++}
\label{sec:SYCL-comparison}

For this evaluation we chose a subset of benchmarks included in the HeCbench suite and compared the performance of their HIP versions of the benchmarks against the SYCL versions. 
The SYCL versions were compiled with Intel's DPC++ shipped with the oneAPI v2024.X.X release\pj{TODO}. Thanks to the  OpenCL backend of DPC++, both versions of the applications could be executed using the same OpenCL driver on the same GPUs, nicely isolating
the differences between the tested software stacks to the runtime and the
LLVM IR level device code compiler optimizations. The results of this comparison are
shown in Fig.~\ref{fig:intel-arc-sycl-hip} for Intel ARC A760 discrete GPU and Fig.~\ref{fig:intel-i9-igpu-chipstar-vs-sycl} for the integrated GPU of Intel's 12900 i9.

\begin{figure}
      \includegraphics[scale=0.5]{figs/hecbench_intel_arc750_hip_vs_sycl.pdf}
      \caption{HIP/\chipstar speed (inverse of execution time) normalized to SYCL/DPC++ on Intel ARC A750. \pj{TODO (Michal):} Total of N benchmarks with geom. mean of X.}
      % plot.py -r -l '#b7cce9' -m 0.8 -s seaborn-v0_8-pastel -t "HeCBench, Intel Arc750, HIP vs SYCL speedup" -c test_20_strict_hip_oclBE_arc_after.csv -b test_20_strict_sycl_oclBE_arc_after.csv
      \label{fig:intel-arc-sycl-hip}
\end{figure}

\begin{figure}
      \includegraphics[scale=0.5]{figs/hecbench_i9_12900_igpu_hip_vs_sycl.pdf}
      \caption{HIP/\chipstar speed (inverse of execution time) normalized to SYCL/DPC++ on Intel i9 12900 iGPU. Higher is better. \pj{TODO Michal:} Total of N benchmarks with geom. mean of X.}
      \label{fig:intel-i9-igpu-chipstar-vs-sycl}
\end{figure}

Interestingly, although both the DPC++ and \chipstar use very similar tools and components in their runtime (OpenCL) and compilation flow (Clang/LLVM and SPIR-V as the device intermediate format), we measured significant variance in the execution performance to both directions.
We analyzed the cases with the most dramatic differences and identified various explanations:
Many of the benchmarks executed very short kernel commands, making the benchmark actually mostly measure the host API call execution speed.
For example, the ``overlay'' benchmark could be sped up significantly by switching off the profiling command queue feature. 
% that was accidentally left on by default. \hl{not accidental. Profiling is used for hipEventElapsedTime.}
In some cases the device built-ins were more optimized in \chipstar than in DPC++, in some cases it was the opposite.
For example, when we compared the \chipstar and DPC++ LLVM IRs of the device code for the ``nlll'' benchmark, we found that only \chipstar performed the if-conversion optimization that converts some of the very small branches to conditional moves, which provided significant benefits.

%%%%%%%%%%%%%%%%%%%%%%%%%%%%%%%%%%%%%%%%%%%%%%%%%%%%%%%%%%%%%%%%%%%%%%%%%

\subsection{CUDA on NVIDIA CUDA SDK}

Execution speed of CUDA programs compiled and ran using the NVIDIA's proprietary CUDA SDK is interesting to compare against the portable and open alternative provided by \chipstar. In this experiment we first compiled CUDA versions of the applications using CUDA SDK \pj{TODO (Henry): version} to get a baseline. The same benchmark cases (the HIP versions) were then compiled using \chipstar to the portable fat binary that uses OpenCL as the portability layer which was run with multiple alternative OpenCL options. The OpenCL options included ``PoCL-CUDA'', which is a work-in-progress OpenCL implemented using the upstream LLVM's PTX backend and libCUDA. In this case two memory allocation API options were measured for curiosity: the Coarse Grain SVM, and the buffer device address (BDA) extension. We also compared to rusticl/zink, an OpenCL implementation on top of Vulkan which implements (as of this writing) only the BDA alternative. The execution time was measured on an NVIDIA RTX 3060 GPU with the results shown in Fig.~\ref{fig:rtx3060-cudasdk-vs-pocl-vs-rusticl}.

The difference between SVN and BDA in ``lfib4'' is caused by the SVM implementation of PoCL-CUDA which uses managed allocation with virtual memory pages that are not created until accessed the first time. The slow down is due to the page faults in the kernel when it writes results to such an allocation. BDA, on the contrary, doesn't allocate virtual memory pages, thus doesn't suffer from this. Other bechmarks encounter this situation as well, but only for the first kernel execution.

\pj{TODO (Karol): Let's rerun with the latest rusticl to see if ``vanGenuchten'' is now faster.}

%Henry>The ``vanGenuchten'' explanation: Not sure what happens truly. The SPIR-V binary that gets passed to the Vulkan driver has pow() function inlined and the kernel outputs a result with more error in it (but still satisfies verification). Perhaps, rusticl/zink uses their own pow() implementation that is a bit relaxed in precision and faster? -Henry

\begin{figure*}[tb]
      \includegraphics[width=0.9\linewidth]{figs/hecbench_rtx3060_cudasdk_vs_pocl_n_rusticl.pdf}
      \caption{Speed of HIP/chipStar on multiple OpenCL runtime options normalized to running the CUDA versions compiled with the proprietary CUDA SDK. SVM = Coarse Grain SVM and BDA = the buffer device address extension. \pj{TODO Henry:} Total of N benchmarks with geom. mean of X.}
      \label{fig:rtx3060-cudasdk-vs-pocl-vs-rusticl}
\end{figure*}

\subsection{HIP on AMD ROCm}

Since \chipstar can be viewed as a more portable implementation of HIP, it is interesting to compare its speed against the original HIP implementation from AMD. For this comparison, we utilized the HIP compiler and runtime from the AMD ROCm package version \pj{Henry TODO} as a baseline to compile and execute a set of HeCbench HIP benchmarks on an AMD Radeon Pro VII GPU. To run the \chipstar fat binaries on the same GPU, we used the rusticl OpenCL implementation on the radeonsi driver. AMD's OpenCL implementation doesn't support SPIR-V input at the time of this writing, preventing its use in this comparison for running the \chipstar binaries.

In the numbers shown in Fig.~\ref{fig:radeonprovii_rocm_vs_rusticl}, an interesting anomaly to look closer at is `pnpoly' which is more than 2.5x faster on chipStar. The reason is that ROCm is slightly faster for tile sizes smaller than 32 and for larger ones notably slower. Unfortunately for ROCm, the benchmark tracks times for the largest tile which is happends to be the slowest one on ROCm. 

\begin{figure}[tb]
      \includegraphics[scale=0.5]{figs/hecbench_radeonprovii_rocm_vs_rusticl.pdf}
      \caption{Speed on AMD Radeon Pro VII through ROCm and chipStar/rusticl/radeonsi, normalized to ROCm. Higher is better for rusticl. \pj{TODO (Henry):} Total of N benchmarks with geom. mean of X.}
      \label{fig:radeonprovii_rocm_vs_rusticl}
\end{figure}


\subsection{Offloading to Integrated GPUs}

To test the extent of portability of the runtime API layer based on the OpenCL standard, we compiled and executed sets of HeCbench applications on various platforms which included both a CPU and a GPU with capable enough OpenCL support to execute the same compiled chipStar fat binary on both devices, enabling interesting CPU-to-GPU offload speedup comparisons. The platforms and their results are presented in the following.
%The HeCbench applications were chosen using the same selection criteria as described in Section~\ref{sec:SYCL-comparison}). 

\paragraph{RISC-V CPU \& PowerVR GPU} In this experiment we utilized the VisionFive2 single board computer for building and running the benchmarks. PoCL~\cite{PoCL} was used for running the benchmarks on the CPU and the proprietary OpenCL driver from Imagination Technologies was used for the GPU.  The results are visualized in Fig.~\ref{fig:intel-visionfive2-gpu-cpu}. The lower performance (0.74 geom mean) of the PowerVR GPU vs RISC-V CPU can be explained by the GPU having much less on-chip resources than most benchmarks could utilize. The CPU (JH7110) has a 32KB L1 i/dcache and a 2MB L2 cache, 4 scalar cores running at 1.5 GHz (no vector support), while the GPU has only 1 compute unit (CU) running at 600 MHz and 4KB of local memory per CU. The GPU also has a native workgroup size of only 32 (subgroup size 16), while most HeCBench benchmarks use a workgroup size ranging from 128 to 1024, leading to additional thread context switches. Furthermore, the GPU's limited local memory is used also to store images, samplers, the OpenCL constant data and pointers to global memory - in addition to the shared data of the application kernels~\cite{PowerVRPerfGuide}. 
      %{there are some benchmark outliers that are >10x slower on GPU vs CPU.
%        asmooth: allocates a local memory array of 1024 floats. This is equal to the on-chip local memory size (4KB) while the local memory is used also in other ways (see prev paragraph), so
%        this most likely results in spilling into global memory.
%        all-pairs-distance: uses atomicAdd and memory access with stride
%        }

\begin{figure}[tb]
      %\includesvg[width=\textwidth]{figs/hecbench_visionfive2_gpu_vs_cpu}
      \includegraphics[scale=0.5]{figs/hecbench_visionfive2_gpu_vs_cpu.pdf}
      \caption{Speed on PowerVR GPU normalized to the RISC-V CPU. \pj{TODO Michal:} Total of N benchmarks with geom. mean of X.}
      \label{fig:intel-visionfive2-gpu-cpu}
\end{figure}

\paragraph{Intel i9 12900} This platform has a 16-core x86-64 CPU and an Intel UHD Graphics 770 integrated GPU. OpenCL on the CPU and the GPU were supported by Intel's OpenCL drivers  \pj{TODO (Michal): driver versions}. The results are shown in Fig.~\ref{fig:intel-i9-cpu-vs-igpu}. In this platform, the offload benefits are more visible thanks to the more powerful GPU.

\begin{figure}[tb]
      \includegraphics[scale=0.5]{figs/hecbench_i9_12900_igpu_vs_cpu.pdf}
      \caption{Speed on Intel i9 12900 iGPU speed normalized to CPU. \pj{TODO Michal:} Total of N benchmarks with geom. mean of X.}
      \label{fig:intel-i9-cpu-vs-igpu}
\end{figure}

\paragraph{ARM Cortex A53+A73 CPU \& Mali G52 GPU} \pj{TODO (Michal): driver info.}  For the CPU, PoCL~\cite{PoCL} was used as the OpenCL driver while the GPU was supported by the ARM's proprietary OpenCL driver. The results are shown in Fig.~\ref{fig:mali-vs-cortex}. A lot of applications were dropped because of limited memory in the GPU and lack of double precision floating point support \pj{TODO (Michal): pls confirm}. However, various benchmarks showed significant benefits from CPU to GPU offloading, as expected.

\begin{figure}[tb]
      \includegraphics[scale=0.5]{figs/hecbench_malig52_vs_cortexa53a73.pdf}
      \caption{Speed on ARM Mali GPU normalized to the ARM Cortex CPU. \pj{TODO Michal:} Total of N benchmarks with geom. mean of X.}
      \label{fig:mali-vs-cortex}
\end{figure}

\section{Real-World Application Case Study: GAMESS}
\label{section:applications}

In order to further test \chipstar in practice, we ported a %set of 
complex HIP/CUDA-based HPC application and its dependency library using \chipstar. The test environment for the experiment was the Aurora supercomputer utilizing Intel Datacenter Intel® Data Center GPU Max Series (referred to from here on as PVCs, as in Ponte Vecchio) as the accelerator part~\cite{aurora}.
%The porting examples are described in the following subsections with the performance evaluations
%presented in the next section.
%The applications and which features or libraries of \chipstar they use are shown in Table~\ref{tab:applications}.

\subsection{GPU Integral Library (GAMESS-EXESS)}

General Atomic and Molecular Electronic Structure System (GAMESS~\cite{gamess,gamess2}) is a quantum chemistry software package which implements many electronic structure methods. 
The code base is primarily in Fortran 77/90 with some C/C++ and a CUDA library. Recently a new GPU version of the Hartree-Fock (HF) and RI-MP2 methods were implemented in CUDA which scales to 4096 nodes on Summit, an Nvidia V100-based supercomputer \cite{gamess_cuda1, gamess_cuda2, summit}.
In this porting case we focused on the Hartee-Fock (HF) algorithm used by a CUDA library in GAMESS described in \cite{gamess_cuda1}, which has been ported to HIP. The HF method is a common quantum chemistry method which is often the starting point for other higher-accuracy methods.  The HF method determines the molecular energy of a system by solving a set of non-linear eigenvalue equations iteratively.  It primarily involves the computation of $N^4$ two electron integrals (where $N$ is a measure of molecular system size) as well as matrix contractions of the two electron integrals once they are formed.
% Colleen> I'm ok to remove the discussion of the basis functions if the code works for all of them :) The two electron integrals are grouped into different classes, depending on the angular momentum of the basis functions used. The basis functions here are $s-$ ,$p-$, and $d-$, where $s$ is least complex and $d$ is the most complex.

The two electron integrals are implemented as HIP/CUDA kernels which were optimized for Nvidia GPUs and total over 20,000 lines of HIP/CUDA kernel code.
%From the non-basic features of HIP supported by \chipstar, the kernels use shared memory with \func{\_\_syncthreads()} calls to ensure copying values from global memory to shared memory completed for the threadblock before using it.\pj{Colleen: does it use any other "special" CUDA/HIP features on the host side?}
%

\subsection{hipBLAS and hipSOLVER}

Since the application uses ROCm software platform libraries hipBLAS and hipSOLVER, they needed to be ported as well. The required interfaces of these libraries were implemented for Intel hardware by using oneMKL as a backend. %Applications calling hipBLAS functions can thus run on Intel hardware without any code changes. 
For this porting case, a SYCL interoperatibility feature was added to \chipstar which was used to invoke oneMKL’s SYCL functions efficiently.

\pj{TODO (Colleen/?): A brief description how hipSOLVER was implemented on MKL?}

%hipBLAS and hipSOLVER calls are used to form intermediates. The main hipBLAS calls are hipblasDscal, hipblasDgemm, hipblasDcopy, hipblasDaxpy, hipblasDdot, hipblasDgemv, hipblasDgeam, and the main hipSOLVER call is hipsolverDsyevd.

%\pj{From these it would be good to summarize for example, that they cover the key blas functionality.} These are used at each iteration of the HF algorithm to (among other things) diagonalize the Fock matrix and construct the density matrix, which are key blas functionalities in the HF algorithm.

\subsection{Porting Notes}

In terms of functionality, the HF code compiles and was verified to run correctly with \chipstar on PVCs. The porting effort was relatively low, with one exception due to a small but significant specification difference in CUDA vs. OpenCL related to kernel thread synchronization: In CUDA group barriers are not counting in exited threads, meaning that there can be early returns from the kernel by a subset of the threads after which it is still legal to perform barrier synchronization with the remaining subset -- the exited threads are just not counted in. In OpenCL this case is undefined behavior and in many implementations can lead to a deadlock. To tackle this gap, an OpenCL extension adding a group barrier with similar semantics would be needed (see \textit{cl\_ext\_alive\_only\_barrier} in Table~\ref{table:extensions}).

\subsection{Performance}

%The GAMESS port was executed with the PVC component of the Aurora installation. 
The performance of GAMESS was measured by compiling and running the same HIP source code on a PVC through \chipstar and on an Nvidia A100 as well as an AMD MI250 using ROCm 6.0.0. 
%\pj{Is it running the HIP code via AMD HIP wrapper or the CUDA version?} \cb{via the AMD HIP wrapper -- everything is using the same HIP code (I put all the runscripts and output on the github https://github.com/colleeneb/gamess\_libcchem\_hip for the record. so it was https://github.com/colleeneb/gamess\_libcchem\_hip/blob/hip\_dev\_for\_intel/hip\_nvidiaa100.sh)} 
% Not sure if this context is needed here: --PJ
%The AMD MI250 is part of the JLSE cluster at ANL and is a Supermicro AS-4124GQ-TNMI composed of 2 AMD EPYC 7713 64c (Milan) CPUs and 4 AMD Instinct MI250~\cite{JLSE}. The Nvidia A100 is also part of the JLSE cluster and is composed of a AMD 7532 and 1 Nvidia A100 with 40GB and PCIe 4.0. 
The test run computed the HF energy of a cluster of 150 water molecules with a STO-3G basis set. The results are displayed in Table~\ref{table:gamess_perf}.

\begin{table*}[ht]
\centering
\begin{tabular}{l|l|l|l|l}
                   & Nvidia A100 & AMD MI250 & Intel PVC      & Intel PVC \\
                   &               &             & (OpenCL backend) & (Level Zero backend) \\ \hline
Total SCF time (s) &    1.998      &    26.09    & 4.9  & 3.6  
\end{tabular}
    \caption{Comparison of GPU integral code performance across Intel, AMD, and Nvidia}
    \label{table:gamess_perf}
\end{table*}

Table~\ref{table:gamess_perf} shows that the execution time on the Nvidia A100 is shortest and on AMD MI250 the longest. The energy calculation of the GAMESS simulation can be split into two main parts: Fock build time (time for computation of electron repulsion integrals and Fock matrix), and DIIS time (time for solving a set of linear equations). The Fock build time is primarily hand-written HIP kernels. The DIIS time is primarily BLAS and LAPACK calls, including calls to the hipSOLVER function hipsolverDsyevd. Table~\ref{table:gamess_breakdown} shows the timing breakdown for each architecture. Compared to the A100 times, the execution times on Intel PVC with the Level Zero backend are 2-5x slower. Although the Fock build time for the Intel PVC with the OpenCL backend is only 3x slower than the A100 time, the times are 25-53x slower for the DIIS and hipsolverDsyevd times. Similarly, although the Fock build time for the MI250 time is only 1.4x slower than the A100 time, and the DIIS time without hipsolverDsyevd is 0.7x the A100 time, the hipsolverDsyevd time is 109x the A100 time.\pj{TODO: Still need to identify why CL is so much slower. But is the LZ number good or not - how is A100 vs. PVC in terms of peak perf?}

% Colleen needs to update this section

\begin{table*}[ht]
\centering
\begin{tabular}{l|l|l|l|l}
 &
  Fock time &
  DIIS time without hipsolverDsyevd &
  hipsolverDsyevd time &
  Remainder \\
 &
  (ratio over A100 time) &
  (ratio over A100 time) &
  (ratio over A100 time) &
  (ratio over A100 time) \\ \hline
Nvidia   A100       & 1.46 (1x) & 0.183 (1x) & 0.230 (1x) & 0.128 (1x) \\
AMD MI250           & 2.03 (1.4x)  & 0.12 (0.7x)  &  25.10 (109x) & 0.09 (0.7x) \\
Intel PVC   (OpenCL) & 4.29 (2.9x)  & 4.73 (25.8x) & 12.34 (53.7x) & 0.54 (4.2x) \\
Intel PVC   (LevelZero)    & 3.11 (2.1x) & 0.954 (5.2x) & 0.984 (4.3x) & 0.262 (2x)
\end{tabular}
    \caption{Timing breakdown of the GPU integral HIP code across Intel, AMD, and Nvidia}
    \label{table:gamess_breakdown}
\end{table*}


\section{Related Work}
\label{section:relatedWork}

The origin of \textit{chipstar} is on the HIPCL~\cite{HIPCL} prototype which first tested the concept of compiling HIP programs to fat binaries relying on OpenCL and SPIR-V. The \chipstar tool described in this article is a result of an almost a complete rewrite of the HIPCL code base and over approximately three years of continuous development work by multiple partners and HPC users. The HIPCL code base was initially forked to a separate code base to utilize the Level Zero~\cite{l0} low level API directly (HIPLZ~\cite{HIPLZ}) after which the OpenCL backend of HIPCL and the Level Zero backend of HIPLZ were merged to the same code base discussed in this article.
A large number of missing essential features have been implemented since the initial prototypes were published. This article thus significantly expands upon the original poster abstract that introduced the early-prototype-stage HIPCL and now presents a much more mature software stack usable for a wider range of real-world workloads.
%The direct Level Zero access is used as an additional backend for comparison purposes in this article, with the primary focus being on the OpenCL backend.

%However, at the time of this writing, the recommended path from CUDA/HIP to Level Zero goes through the OpenCL backend and PoCL's~\cite{poclIJPP} Level Zero backend since the OpenCL code path has matured longer and is somewhat more robust.

When comparing \chipstar to other HIP implementations, obviously the original ROCm, the AMD's official GPU software platform~\cite{ROCm}, is the baseline. ROCm consists of the general purpose programming API compilation and runtime support for HIP, and a set of libraries that support different degrees of compatibility with the CUDA platform. \chipstar is not a new backend in addition to the AMD GPU and NVIDIA GPU backends provided by the AMD's offering, but has an important technical difference: \chipstar aims to offer runtime portability by its open standard based fat binary, removing the need to recompile the input software per target vendor platform, which is the case with ROCm.
%\pj{Brice/Paulius: Can you check that this is (still) true?}

%HIP is very close to CUDA, and in fact AMD provides a source-to-source translation tool called HIPify that can automate the porting process. Interestingly, although heavily based on the NVIDIA-driven CUDA, AMD now promotes HIP as the primary C++ programming API for their GPU platforms. Since AMD GPUs have increased their market share and received major design wins in large HPC installations, HIP as such has risen in importance as an application-facing interface.

SYCLomatic~\cite{SYCLomatic} is a tool contributed by Intel Corporation for converting CUDA sources to the cross-vendor open standard SYCL~\cite{SYCL}. Similar to AMD's HIPify, but in contrast to \chipstar which aims for source-level compatibility, SYCLomatic is a source-to-source conversion tool, which has its good and bad sides. The most apparent implication of relying on source-to-source conversion is more about maintenance aspects than technical ones; it neccessitates the further development of the converted application to proceed using the SYCL API instead or in addition to CUDA. The main drawback is that in reality many code bases are difficult or impossible to convert solely to SYCL without having the CUDA version as a backup due to legacy, risk-management or technical reasons. The main benefit is that SYCL is an open standard, in constrast to CUDA, enabling more fair competition ground between hardware vendors. Thus, being able to target many platforms from a fat binary compiled from the unmodified CUDA/HIP source code base using a \chipstar-style open platform approach can have its benefits. Furthermore, since \chipstar is not a linkage-time or binary translation solution, but requires recompilation, it coincidentally also encourages utilizing and further developing the cross-vendor ecosystem APIs it relies upon. Furthermore, as of this writing SYCLomatic supports only CUDA, not HIP, while HIP has an increasing number of new applications implemented directly using it.

% https://github.com/vosen/ZLUDA
ZLUDA~\cite{ZLUDA} is a tool for running unmodified CUDA binaries on AMD GPUs. It works by reimplementing the \cuda driver API, and converting NVIDIA PTX~\cite{ptx} to the vendor-specific IRs. Since it is a ``drop-in solution'' that works at program loading/linkage time, it can execute unmodified CUDA fat binaries, which is very comfortable to the end users as it doesn't require access to the source code of the application. While we see ZLUDA as an excellent tool, it requires reverse engineering CUDA SDK's binary interfaces and keeping up-to-date with the NVIDIA PTX as it evolves. We believe, in the longer term, especially as more of the missing extensions we describe are adopted by OpenCL implementations, \chipstar can provide a more robust solution. 
%Of course only time will tell how well this turns out to be the case. 
In addition, ZLUDA also doesn't support HIP as an input and now only targets AMD GPUs, whereas a key goal of \chipstar is extensive cross-vendor portability.

%However, its developed has stalled and it only supports a limited subset of applications and only on the Intel devices supported by the Level Zero API. ZLUDA author claims in their web page that they can achieve performance benefits when running straight on top of the lower level \lz instead of the somewhat higher level OpenCL. Since \chipstar supports both, we were able to measure this difference accurately, and found it to be negligible\pj{to do actually}.
%A key benefit of skipping a cross-vendor standardized layer is that PTX has instructions which map directly to the Intel GPU instructions which are not exposed in OpenCL C.
%Although \chipstar uses the OpenCL runtime for portability, it targets SPIR-V instead of OpenCL C as the device-side programming language, thus this drawback does not appear with it. The potential overhead is first passing through LLVM IR, which might lose beneficial information, but that also is found not to be an issue according to the measurements presented in Section~\ref{TODO} \pj{to do actually}.

MCUDA~\cite{MCUDA} is the oldest tool we found for porting CUDA programs to non-NVIDIA platforms. MCUDA does source-to-source translation of kernels in a fashion that the translated kernels can execute efficiently on CPUs on a single CPU thread while respecting the barrier synchronization. In the case of \chipstar, since it uses OpenCL as its portability layer, it can similarly target also vectorized CPU execution through CPU-targeting OpenCL implementations such as the Intel OpenCL CPU driver and PoCL's CPU drivers~\cite{PoCL}. Both of them are capable of vectorizing work-items (CUDA/HIP threads) inside work-groups, which translates to implicit autovectorization of CUDA/HIP kernels across CUDA threads and provide the benefits of CPU execution such as easier kernel debugging.

Swan~\cite{Swan} is another early source-to-source tool for CUDA porting. It generates OpenCL code from CUDA, providing similar level of portability as \chipstar does. Another similar tool, CU2CL~\cite{CU2CL} was published in the same year as \cite{Swan}. Neither Swan nor CU2CL are maintained any longer.  In comparison to \chipstar, the main technical differences to these tools are that \chipstar utilizes the latest version of the OpenCL standard to support the newer CUDA/HIP features, uses SPIR-V as the intermediate language (no need to generate textual OpenCL C with its limitations) and it doesn't suffer from problems related to source-to-source translations as \chipstar provides source-level compatibility.

The closest comparable CUDA porting tool we could find is CUDA-on-CL~\cite{CUDAonCL}. Like \chipstar, it similarly compiles CUDA programs using Clang/LLVM-based compiler chain to binaries which then execute on OpenCL platforms. However, similarly to Swan and CU2L, it compiles device kernels to OpenCL C whereas \chipstar uses SPIR-V as the portable binary format. Other technical differences in \chipstar are related to the use of modern OpenCL standard features to implement some of the features of CUDA. These include using SVM to implement raw pointers and implementing warp-level primitives such as shuffles using the subgroup features. 




%%%%%%%%%%%%%%%%%%%%%%%%%%%%%%%%%%%%%%%%%%%%%%%%%%%%%%%%%%%%%%%%%%%%%%%%%

% This doesn't fit as the page limit is only 12pp. If we resubmit to another journal, it's interesting info to add and can be easily copy-pasted from the report:

%\section{Supporting Newer CUDA Features}
%\label{section:directCUDA}

%HIP is a subset of CUDA features, roughly at version 8~\pj{check this}. Thus, it doesn't include support some of the newer features which can utilize some of the more advanced capabilities of the NVIDIA GPU platforms. Some of these features are difficult to implement efficiently on other vendors' GPU features, and since GPU offloading is primarily done with performance improvements in mind, a functional, but inefficient implementation is less interesting.

%However, for the purpose of completeness, it is interesting to highlight some of the more useful newer features in later CUDA versions, and consider implementation strategies for future work.

%\pj{Discuss features specific to CUDA, from the doc I wrote in Parmance.}

% TODO: the OpenCL extensions identified and proposed.

%%%%%%%%%%%%%%%%%%%%%%%%%%%%%%%%%%%%%%%%%%%%%%%%%%%%%%%%%%%%%%%%%%%%%%%%%

\section{Conclusions}
\label{section:conclusions}

In this article, we presented \chipstar, a compilation flow and a runtime for CUDA/HIP applications using open cross-vendor supported standards. In comparison to previous tools, \chipstar's goal is on source-level compatiblity which we believe has longer-term robustness benefits in comparison to binary translation. Whereas relying on standardized APIs has its drawbacks in shorter term due to the cross-vendor ``democratic concensus'' requirement, in terms of building open and fair heterogeneous computing ecosystem of the future, we consider using and expanding open standards aiming at the portability layer level has also significant far-reaching value.

The performance of HIP benchmarks using \chipstar was shown to be on par or surpass their SYCL versions when using a compiler/runtime with similar components (OpenCL and SPIR-V). An example of the source-level compatibilty was provided with GAMESS, a code base with a significant number of kernel code lines. This demonstrates that \chipstar is a useful option for applications that are not feasible to port to more cross-vendor supported open standard input APIs such as SYCL or OpenMP.

In the future, as \chipstar focuses on the HIP/CUDA API/language support, the main aspect that requires more engineering work is to expand the directly supported set of core libraries in the CUDA and HIP ecosystems to cover more real-world applications. On the HIP/CUDA core front, we aim to focus on the more advanced recent features such as collaborative groups as well as the standard extensions to bridge the remaining gaps between CUDA/HIP and OpenCL/SPIR-V.

\bibliography{IEEEabrv,chipstar}
\bibliographystyle{IEEEtran}
\newpage

\pj{Notes:
\begin{itemize}
    \item I merged the hipBLAS description to GAMESS for now.
 %   \item I removed the debugging/profiling section. Not enough sensible content. We should state somewhere that thanks to using OpenCL we can use any tools that can profile OpenCL. Using the CPU target helps in debugging thanks to running gdb.
 %   \item Can we add more porting case studies which demonstrate something different than GAMESS? CP2K? Is the pruned-down Pytorch-HIP sensible or too pruned down? Exabiome?
    \item I removed the Libraries section as there was too little technical content. cuBLAS is too brief and highlights the negative fact that we support only a small subset of the NVIDIA or AMD ecosystem libraries.
\end{itemize}}

\pj{todo:
  \begin{itemize}
    \item We should add more pictures. It's too much of text without pics. Anyone any idea of what and where?
    \item The article still reads a bit as a ``technical report'' instead of a ``scientific paper''. For more "scientific content" we should add a bit more complex technical content of some more trickier aspects, e.g., about the Graph implementation using command buffers. Ideas?
\end{itemize}}

\end{document}


\end{document}
